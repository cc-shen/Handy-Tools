\documentclass[12pt]{article}
\usepackage[margin = 1.5in]{geometry}
\setlength{\parindent}{0in}
\usepackage{amsfonts, amssymb, amsthm, mathtools, tikz, qtree, float}
\usepackage{algpseudocode, algorithm, algorithmicx}
\usepackage[T1]{fontenc}
\usepackage[utf8]{inputenc}
\usepackage{DejaVuSans}
\usepackage{ae, aecompl, color}
\usepackage{wrapfig}
\usepackage{multicol, multicol, array}
\usepackage{imakeidx}
\makeindex[columns=2, title=Indices, intoc]

\usepackage[pdftex, pdfauthor={Charles Shen}, pdftitle={MATH 136: Linear Algebra I}, pdfsubject={Theorems and more from MATH 136: at the University of Waterloo}, pdfkeywords={course notes, notes, Waterloo, University of Waterloo}, pdfproducer={LaTeX}, pdfcreator={pdflatex}]{hyperref}
\usepackage{cleveref}

\DeclarePairedDelimiter{\set}{\lbrace}{\rbrace}
\renewcommand*\familydefault{\sfdefault}
\definecolor{darkish-blue}{RGB}{25,103,185}

\hypersetup{
  colorlinks,
  citecolor=darkish-blue,
  filecolor=darkish-blue,
  linkcolor=darkish-blue,
  urlcolor=darkish-blue
}

\theoremstyle{definition}
\newtheorem*{defn}{Definition}
\newtheorem*{theorem}{Theorem}
\newtheorem*{corollary}{Corollary}
\newtheorem{ex}{Example}[section]

\crefname{ex}{Example}{Example}

\setlength{\marginparwidth}{1.5in}
\newcommand{\lecture}[1]{
  \marginpar{{
    \footnotesize $\leftarrow$ \underline{#1}}
  }
}

\newcommand{\defnterm}[1]{
  \textbf{\textcolor{teal}{\index{#1}}}
}

\newcolumntype{L}[1]{>{\raggedright\let\newline\\\arraybackslash\hspace{0pt}}m{#1}}
\newcolumntype{C}[1]{>{\centering\let\newline\\\arraybackslash\hspace{0pt}}m{#1}}
\newcolumntype{R}[1]{>{\raggedleft\let\newline\\\arraybackslash\hspace{0pt}}m{#1}}

\allowdisplaybreaks

\makeatletter
\def\blfootnote{\gdef\@thefnmark{}\@footnotetext}
\makeatother

%%%%%%%%%%%%%%%%%%%%%
%% D O C U M E N T %%
%%%%%%%%%%%%%%%%%%%%%

\begin{document}
\let\ref\Cref
\pagenumbering{roman}

\title{\bf{MATH 136: Linear Algebra I}}
\date{Winter 2015, University of Waterloo \\ \center Quick reference sheet.}
\author{Charles Shen}

\blfootnote{Feel free to email feedback to me at
\href{mailto:ccshen902@gmail.com}{ccshen902@gmail.com}.}

\maketitle
\thispagestyle{empty}
\newpage
\tableofcontents
\newpage
\pagenumbering{arabic}

\section{Matrices}
\begin{defn}
  An \textbf{\textcolor{teal}{augmented matrix}} has the form $\left[A ~|~ \vec{b}~\right]$
\end{defn}

\begin{defn}
  A \textbf{\textcolor{teal}{homogeneous matrix}} has the form $\left[A ~|~ \vec{0}~\right]$ \\
  A homogeneous system is always consistent, $\vec{x} = \vec{0}$ is always a solution.
\end{defn}

\subsection{Elementary Row Operations}
\begin{enumerate}
  \item Add a multiple of one row to another: $R_{i} + cR_{j}$
  \item Multiple a row by a non-zero constant: $cR_{i}$
  \item Exchange two rows: $R_{i} \leftrightarrow R_{j}$
\end{enumerate}

\subsection{Reduced Row Echelon Form}
\begin{defn}
  A matrix $R$ is said to be in \textbf{\textcolor{teal}{Reduced Row Echelon Form}} (RREF) if:
  \begin{enumerate}
   \item All rows containing a non-zero entry are above rows which only contains zeros
   \item The first non-zero entry in each non-zero row is $1$, called a \textbf{\textcolor{teal}{leading one}} (or a pivot)
   \item The leading one in each non-zero row is to the right of the leading one in any row above it
   \item A leading one is the only non-zero entry in its column
  \end{enumerate}
\end{defn}

\begin{defn}
  The \textbf{\textcolor{teal}{rank}} of a matrix $A$ is the number of leading ones in the RREF of the matrix.
\end{defn}

\subsection{System Rank Theorem}
\begin{theorem}
  Let $A$ be the coefficient matrix of a system of $m$ linear equations in $n$ unknowns $\left[A ~|~ \vec{b}~\right]$
  \begin{enumerate}
    \item If the rank of $A$ is less than the rank of the augmented matrix $\left[A ~|~ \vec{b}~\right]$, then the system is inconsistent
    \item If the system $\left[A ~|~ \vec{b}~\right]$ is consistent, then the system contains $(n - rank(A))$ free variables
    \item $rank(A) = m$ iff the system $\left[A ~|~ \vec{b}~\right]$ is consistent for every $\vec{b} \in \mathbb{R}^{m}$
  \end{enumerate}
\end{theorem}

\section{Identities}
\subsection{Transpose}
\begin{defn}
  The \textbf{\textcolor{teal}{transpose}} of an $m \times n$ matrix $A$ is the $n \times m$ matrix $A^\intercal$ whose $ij$-th entry is the $ji$-th entry of $A$; that is:
  $$(A^\intercal)_{ij} = (A)_{ji}$$
\end{defn}
\begin{theorem}
  If $A$ and $B$ are $m \times n$ matrices and $c \in \mathbb{R}$, then:
  \begin{enumerate}
    \item $(A^\intercal)^\intercal = A$
    \item $(A + B)^\intercal = A^\intercal + B^\intercal$
    \item $(cA)^\intercal = cA^\intercal$
    \item $(AB)^\intercal = B^\intercal A^\intercal$
  \end{enumerate}
\end{theorem}

\section{Inverses}
\subsection{Inverse of a 2-by-2 Matrix}
\begin{align*}
  \text{Suppose } A &=
  \begin{bmatrix}
    a & b \\
    c & d
  \end{bmatrix} \\
  \text{Then } A^{-1} &= \frac{1}{ad - bc}
  \begin{bmatrix}
    d & -b \\
    -c & a
  \end{bmatrix}
\end{align*}

\subsection{Invertible Matrix Theorem}
\begin{theorem}
  For any $n \times n$ matrix $A$, the following are equivalent:
  \begin{enumerate}
    \item $A$ is invertible
    \item $rank(A) = n$
    \item RREF of $A = I$
    \item $A\vec{x} = \vec{b}$ is consistent for all $\vec{b} \in \mathbb{R}^{n}$
    \item The columns of $A$ terms are a linearly independent set
    \item The columns of $A$ spans $\mathbb{R}^{n}$
    \item The columns of $A$ forms a basis for $\mathbb{R}^{n}$
    \item The rows of $A$ forms a basis for $\mathbb{R}^{n}$
    \item $Col(A) = \mathbb{R}^{n}$
    \item $Null(A) = \{\vec{0}\}$
    \item $A^{\intercal}$ is invertible
    \item $Row(A) = \mathbb{R}^{n}$
    \item $Null(A^{\intercal}) = \{\vec{0}\}$
  \end{enumerate}
\end{theorem}

\clearpage
\printindex
\end{document}

%%%%%%%%%%%%%%%%%%%%%
%% D O C U M E N T %%
%%%%%%%%%%%%%%%%%%%%%
