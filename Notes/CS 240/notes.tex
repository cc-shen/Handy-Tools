\documentclass[12pt]{article}
\usepackage[margin = 1.5in]{geometry}
\setlength{\parindent}{0in}
\usepackage{amsfonts, amssymb, amsthm, mathtools, tikz, qtree, float}
\usepackage[lined]{algorithm2e}
\usepackage[T1]{fontenc}
\usepackage{ae, aecompl, color}
\usepackage[pdftex, pdfauthor={Charles Shen}, pdftitle={CS 240: Data Structures and Data Management}, pdfsubject={Lecture notes from CS 240: at the University of Waterloo}, pdfkeywords={course notes, notes, Waterloo, University of Waterloo}, pdfproducer={LaTeX}, pdfcreator={pdflatex}]{hyperref}
\usepackage{cleveref}
\usepackage{wrapfig}
\usepackage{multicol}

\DeclarePairedDelimiter{\set}{\lbrace}{\rbrace}

\definecolor{darkish-blue}{RGB}{25,103,185}

\hypersetup{
  colorlinks,
  citecolor=darkish-blue,
  filecolor=darkish-blue,
  linkcolor=darkish-blue,
  urlcolor=darkish-blue
}

\theoremstyle{definition}
\newtheorem*{defn}{Definition}
\newtheorem*{theorem}{Theorem}
\newtheorem*{corollary}{Corollary}
\newtheorem{ex}{Example}[section]

\crefname{ex}{Example}{Example}

\setlength{\marginparwidth}{1.5in}
\newcommand{\lecture}[1]{
  \marginpar{{
    \footnotesize $\leftarrow$ \underline{#1}}
  }
}

\newcommand{\includePicture}[3]{
  \begin{figure}[!ht]
  \centering
  \scalebox{#1}{\includegraphics{#2}}
  \caption{#3}
  \end{figure}
}

\allowdisplaybreaks

\makeatletter
\def\blfootnote{\gdef\@thefnmark{}\@footnotetext}
\makeatother

%%%%%%%%%%%%%%%%%%%%%
%% D O C U M E N T %%
%%%%%%%%%%%%%%%%%%%%%

\begin{document}
\let\ref\Cref

\title{\bf{CS 240: Data Structure and Data Management}}
\date{Spring 2016, University of Waterloo \\ \center Formulas, run time, and more.}
\author{Charles Shen}

\blfootnote{Feel free to email feedback to me at
\href{mailto:echen902@gmail.com}{echen902@gmail.com}.}

\maketitle
\newpage
\tableofcontents
\newpage

\section{Order Notation}
\subsection{Big O}
O-Notation (bound from above; worst case):
$$f(n) \in O(g(n))$$
if there exists constants $c, n_{0} > 0$ such that
$$0 \leq f(n) \leq cg(n) ~~~\forall n \geq n_{0}$$

\subsection{Big Omega}
$\Omega$-Notation (bound from below; best case):
$$f(n) \in \Omega(g(n))$$
if there exists constants $c, n_{0} > 0$ such that
$$0 \leq cg(n) \leq f(n) ~~~\forall n \geq n_{0}$$

\subsection{Theta Bound}
$\theta$-Notation (tight bound):
$$f(n) \in \theta(g(n))$$
if there exists constants $c_{1}, c_{2}, n_{0} > 0$ such that
$$c_{1}g(n) \leq f(n) \leq c_{2}g(n) ~~~\forall n \geq n_{0}$$

\subsection{Little o}
o-Notation (bound from above for all):
$$f(n) \in o(g(n))$$
if for all constants $c > 0$, there exists a constant $n_{0} > 0$ such that
$$0 \leq f(n) \leq cg(n) ~~~\forall n \geq n_{0}$$

\subsection{Little omega}
$\omega$-Notation (bound from below for all):
$$f(n) \in o(g(n))$$
if for all constants $c > 0$, there exists a constant $n_{0} > 0$ such that
$$0 \leq cg(n) \leq cf(n) ~~~\forall n \geq n_{0}$$

\subsection{Techniques for Order Notation}
Suppose that $f(n) > 0$ and $g(n) > 0$ for all $n \geq n_{0}$. \\
Suppose that
$$L = \lim_{n\to\infty}\frac{f(n)}{g(n)}$$
Then
\begin{align*}
f(n) \in
\begin{cases}
o(g(n)) ~~&\text{if } L = 0 \\
\theta(g(n)) ~~& \text{if } 0 < L < \infty \\
\omega(g(n)) ~~&\text{if } L = \infty
\end{cases}
\end{align*}

\subsection{Relationships Between Order Notations}
\begin{align*}
f(n) \in \theta(g(n)) &\Longleftrightarrow g(n) \in \theta(f(n))& \\
f(n) \in O(g(n)) &\Longleftrightarrow g(n) \in \Omega(f(n))& \\
f(n) \in o(g(n)) &\Longleftrightarrow g(n) \in \omega(f(n))& \\ \\
f(n) \in \theta(g(n)) &\Longleftrightarrow
  f(n) \in O(g(n)) \text{ and } f(n) \Omega(g(n))& \\
f(n) \in o(g(n)) &\Longleftrightarrow f(n) \in O(g(n)) \\
f(n) \in o(g(n)) &\Longleftrightarrow f(n) \not\in \Omega(g(n)) \\
f(n) \in \omega(g(n)) &\Longleftrightarrow f(n) \in \Omega(g(n)) \\
f(n) \in \omega(g(n)) &\Longleftrightarrow f(n) \not\in O(g(n))
\end{align*}

\subsection{Algebra of Order Notation}
``Maximum'' Rules: Suppose that $f(n) > 0$ and $g(n) > 0$ for all $n \geq n_{0}$, then:
\begin{align*}
O(f(n) + g(n)) &\ O(max\{f(n), g(n)\}) \\
\theta(f(n) + g(n)) &\ \theta(max\{f(n), g(n)\}) \\
\Omega(f(n) + g(n)) &\ \Omega(max\{f(n), g(n)\})
\end{align*}

Transitivity: If $f(n) \in O(g(n))$ and $g(n) \in O(h(n))$ then $f(n) \in O(h(n))$.

\section{Summation Formulas}
Arithmetic:
$$\sum_{i=0}^{n-1}(a + di) = na + \frac{dn(n-1)}{2} \in \theta(n^{2}) ~\text{for } d \not = 0$$
Geometric:
\begin{align*}
\sum_{i=0}^{n-1}ar^{i} =
\begin{cases}
a\frac{r^{n} - 1}{r - 1} &\in \theta(r^{n}) \text{ if } r > 1 \\
na &\in \theta(n) ~\text{ if } r = 1 \\
a\frac{1 - r^{n}}{1 - r} &\in \theta(1) ~\text{ if } 0 < r < 1
\end{cases}
\end{align*}
Harmonic:
$$H_{n} = \sum_{i=1}^{n}\frac{1}{i} \in \theta(\log n)$$ \\

\begin{flalign}
&\sum_{i=1}^{n}ir^{i} = \frac{nr^{n+1}}{r - 1} - \frac{r^{n+1} - r}{(r-1)^{2}} &
&\sum_{i=1}^{\infty}\frac{1}{i^{2}} = \frac{\pi}{6} & \nonumber \\
&\text{for } k \geq 0, \sum_{i=1}^{n}i^{k} \in \theta(n^{k+1}) & \nonumber \\
&n! \in \theta(\frac{n^{\frac{n+1}{2}}}{e^{n}}) &
&\log n! \in \theta(n \log n) & \nonumber
\end{flalign}

%%%%%%%%%%%%
%% REVIEW %%
%%%%%%%%%%%%

\section{Review}
\textbf{[A] Prove} $\frac{1}{n} \in o(1)$
\begin{flalign}
&\frac{1}{n} < c & \notag \\
&1 < cn & \notag \\
&n > \frac{1}{c} & \notag \\
&\text{Choose } n_{0} = \frac{2}{c} &\notag \\
&\text{Given $c > 0$, set } n_{0} = \frac{2}{c} &\notag \\
&\frac{1}{n} \leq \frac{1}{n_0} \leq \frac{1}{2/c} & \notag \\
&= \frac{c}{2} & \notag \\
&< c & \blacksquare \notag
\end{flalign}
\newline

\textbf{[B] Prove} $\frac{1}{n\sqrt{n}} \not\in O(\frac{1}{n^2}$
\begin{flalign}
&\exists c, n_{0} > 0 \text{ such that} & \notag \\
&\frac{1}{n\sqrt{n}} \leq \frac{c}{n^2} & \notag \\
&\frac{n^{2}}{n\sqrt{n}} \leq \frac{n^{2}c}{n^{2}} & \notag \\
&\frac{n}{\sqrt{n}} \leq c & \notag \\
&\sqrt{n} \leq c & \notag \\
& n \leq c^{2} & \notag \\
&\textbf{Contradiction. Thus } n > c^{2} &\blacksquare \notag
\end{flalign}
\newline

\textbf{[C]} Suppose you own \emph{n} electronic devices.
You have \emph{n} charger cables associated with each phone.
Each plug is slightly different, but you can't compare plugs with each other.
You can only find which charger fits with each phone by plugging it in. \\
Give a randomized $O(n\log n)$ algorithm:

\end{document}

%%%%%%%%%%%%%%%%%%%%%
%% D O C U M E N T %%
%%%%%%%%%%%%%%%%%%%%%
