\documentclass[12pt]{article}
\usepackage[margin = 1.5in]{geometry}
\setlength{\parindent}{0in}
\usepackage{amsfonts, amssymb, amsthm, mathtools, tikz, qtree, float}
\usepackage{algpseudocode, algorithm, algorithmicx}
\usepackage[T1]{fontenc}
\usepackage[utf8]{inputenc}
\usepackage{DejaVuSans}
\usepackage{ae, aecompl, color}
\usepackage{wrapfig}
\usepackage{multicol, multicol, array}
\usepackage{imakeidx}
\makeindex[columns=2, title=Indices, intoc]

\usepackage[pdftex, pdfauthor={Charles Shen}, pdftitle={SCI 206: The Physics of How Things Work}, pdfsubject={Lecture Notes from SCI 206: at the University of Waterloo}, pdfkeywords={course notes, notes, Waterloo, University of Waterloo}, pdfproducer={LaTeX}, pdfcreator={pdflatex}]{hyperref}
\usepackage{cleveref}

\DeclarePairedDelimiter{\set}{\lbrace}{\rbrace}
\renewcommand*\familydefault{\sfdefault}
\definecolor{darkish-blue}{RGB}{25,103,185}

\hypersetup{
  colorlinks,
  citecolor=darkish-blue,
  filecolor=darkish-blue,
  linkcolor=darkish-blue,
  urlcolor=darkish-blue
}

\theoremstyle{definition}
\newtheorem{ex}{Example}[subsection]

\crefname{ex}{Example}{Example}

\setlength{\marginparwidth}{1.5in}
\newcommand{\lecture}[1]{
  \marginpar{{
    \footnotesize $\leftarrow$ \underline{#1}}
  }
}

\newcommand{\defnterm}[1]{\textbf{\textcolor{teal}{#1}}\index{#1}}

\newcolumntype{L}[1]{>{\raggedright\let\newline\\\arraybackslash\hspace{0pt}}m{#1}}
\newcolumntype{C}[1]{>{\centering\let\newline\\\arraybackslash\hspace{0pt}}m{#1}}
\newcolumntype{R}[1]{>{\raggedleft\let\newline\\\arraybackslash\hspace{0pt}}m{#1}}

\allowdisplaybreaks

\makeatletter
\def\blfootnote{\gdef\@thefnmark{}\@footnotetext}
\makeatother

%%%%%%%%%%%%%%%%%%%%%
%% D O C U M E N T %%
%%%%%%%%%%%%%%%%%%%%%

\begin{document}
\let\ref\Cref
\pagenumbering{roman}

\title{\bf{SCI 206: The Physics of How Things Work}}
\date{Fall 2016, University of Waterloo \\ \center Notes written from Stefan Idziak's lectures.}
\author{Charles Shen}

\blfootnote{Feel free to email feedback to me at
\href{mailto:ccshen902@gmail.com}{ccshen902@gmail.com}.}

\maketitle
\thispagestyle{empty}
\newpage
\tableofcontents
\newpage
\pagenumbering{arabic}

\section{Motion}
\lecture{Sept. 12, 2016}
\subsection{Inertia}
\defnterm{Inertia} states that a body in motion tends to remain in motion; a body at rest tends to remain at rest. \\

At any particular moment, you're located at a \defnterm{position}---that is, it is a specific point in space. \\
Whenever a position is reported, it is always as a \defnterm{distance} and \defnterm{direction} from some reference point. \\

Position is an example of a vector quantity. \\
A \defnterm{vector quantity} consists of both a magnitude and a direction; the \defnterm{magnitude} tells you how much of the quantity there is, while the \defnterm{direction} tells you which way the quantity is moving. \\

\defnterm{Velocity} measures the rate at which your position is changing with time. \\
Its magnitude is your \defnterm{speed}, the distance you travel in a certain amount of time,
$$\text{speed} = \frac{\text{distance}}{\text{time}}$$
and its direction is the direction in which you're heading. \\
A velocity of zero is special, because it has no direction. \\

When your velocity is constant and doesn't changes, you \defnterm{coast}.

\subsection{Newton's Law}
\defnterm{Newton's First Law of Motion}: an object that is not subject to any outside forces moves at a constant velocity, covering equal distances in equal times along a straight-line path. \\

The outside influences referred in the first law are called \defnterm{forces}, a technical term for pushes and pulls. \\
An object that moves in accordance with Newton's first law is said to be inertial. \\

\defnterm{Mass} is the measure of your inertia, your resistance to changes in velocity.
Almost everything in the universe has mass. \\
Mass has no direction, so it's not a vector quantity. \\
It is a \defnterm{scalar quantity}---that is, a quantity that only has an amount. \\

When something pushes on you, your velocity changes; in other words, you accelerate. \\
\defnterm{Acceleration}, a vector quantity, measures the rate at which your velocity is changing with time. \\
Any changes in velocity is acceleration. \\

You accelerate in response to the \defnterm{net force} you experience---the sum of all the individual forces being exerted on you.
\begin{align*}
\text{acceleration} &= \frac{\text{net force}}{\text{mass}} \\
\text{net force} &= \text{mass} \cdot \text{acceleration} \\
F_{net} &= m \cdot a
\end{align*}
This relationship is known as \defnterm{Newton's Second Law of Motion};
the net force exerted on an object is equal to the object's mass times its acceleration.
The acceleration is in the same direction as the net force. \\

\defnterm{Newton's Third Law of Motion}: for every force that one object exerts on a second object, there is an equal but oppositely directed force that the second object exerts on the first object. \\

Forces that are directed exactly away from surfaces are called \defnterm{normal forces}, since the term \defnterm{normal} is used by mathematicians to describe something that points exactly away from a surface---at a right angle or perpendicular to that surface.

\subsection{Weight and Gravity}
\defnterm{Gravity} is the physical phenomenon that produces attractive forces between every pair of objects in the universe. \\

Earth exerts a downward force on any object near its surface.
That object is attracted directly toward the center of Earth with a force we call the object's \defnterm{weight}.
This weight is exactly proportional to the object's mass. \\
Weight is how hard gravity pulls on an object, and mass is how difficult an object is to accelerate.

We have
$$\text{weight} = \text{mass} \cdot \text{acceleration due to gravity}$$
In other words: you can lose weight by either reducing your mass or by going someplace, like a small planet, where gravity is weaker.

The Earth's acceleration due to gravity is $9.8N/kg$ or $9.8m/s^{2}$.

\subsection{The Velocity of a Falling Ball}
The current (present) velocity can be calculated as
\begin{align*}
  \text{present velocity} &= \text{initial velocity} + \text{acceleration} \cdot \text{time} \\
  v_{f} &= v_{i} + a \cdot t
\end{align*}

\subsection{The Position of a Falling Ball}
The average velocity is exactly halfway in between the two individual velocities:
$$\text{average velocity} = \text{initial velocity} + \frac{1}{2}\cdot\text{acceleration}\cdot\text{time}$$

The present position can be calculated as the following:
\begin{align*}
  \text{present position} &= \text{initial position} + \text{initial velocity}\cdot\text{time} +
  \frac{1}{2}\cdot\text{acceleration}\cdot\text{time}^{2} \\
  x_f &= x_i + v_i \cdot t + \frac{1}{2} \cdot a \cdot t^2
\end{align*}

\subsection{Tossing the Ball Upward}
The larger the initial velocity of the ball, the longer it rises and the higher it goes before its velocity is reduced to zero. \\
It then descends for the same amount of time it spent rising. \\
The higher the ball goes before it begins to descent, the longer it takes to return to the ground and the faster it's traveling when it arrives.

\subsection{Energy}
The capacity to make things happen is called \defnterm{energy}, and the process of making them happen is called \defnterm{work}. \\
Energy and work are both physical quantities, meaning that both are measurable. \\

Physical \defnterm{energy} is defined as the capacity to do work. \\
Physical \defnterm{work} refers to the process of transferring energy. \\

Energy is what's transferred, and work does the transferring.
The most important characteristic of energy is that it's conserved.
In physics, a \defnterm{conserved quantity} is one that can't be created or destroyed but that can be transferred between objects or, in the case of energy, be converted from one form to another. \\

When you throw a ball, it picks up speed and undergoes an increase in \defnterm{kinetic energy}, energy of motion that allows the ball to do work on whatever it hits. \\
When you list a rock, it shifts farther from Earth and undergoes an increase in \defnterm{gravitational potential energy}, energy stored in the gravitational forces between the rock and Earth that allows the rock to do work on whatever it falls on. \\
\defnterm{Potential energy} is energy stored in the forces between or within objects.

\subsection{Doing Work}
To do work on an object, you must push on it while it moves in the direction of your push. \\
This relationship can be expressed as the following:
\begin{align*}
  \text{work} &= \text{force} \cdot \text{distance} \\
  W &= F \cdot d
\end{align*}
\emph{If you're not pushing or it's not moving, then you're not working.}

\clearpage
\printindex
\end{document}

%%%%%%%%%%%%%%%%%%%%%
%% D O C U M E N T %%
%%%%%%%%%%%%%%%%%%%%%
