\documentclass[12pt]{article}
\usepackage[margin = 1.5in]{geometry}
\setlength{\parindent}{0in}
\usepackage{amsfonts, amssymb, amsthm, mathtools, tikz, qtree, float}
\usepackage{algpseudocode, algorithm, algorithmicx}
\usepackage[T1]{fontenc}
\usepackage[utf8]{inputenc}
\usepackage{DejaVuSans}
\usepackage{ae, aecompl, color}
\usepackage{wrapfig}
\usepackage{multicol, multicol, array}
\usepackage{imakeidx}
\makeindex[columns=2, title=Indices, intoc]

\usepackage[pdftex, pdfauthor={Charles Shen}, pdftitle={SCI 206: The Physics of How Things Work}, pdfsubject={Lecture Notes from SCI 206: at the University of Waterloo}, pdfkeywords={course notes, notes, Waterloo, University of Waterloo}, pdfproducer={LaTeX}, pdfcreator={pdflatex}]{hyperref}
\usepackage{cleveref}

\DeclarePairedDelimiter{\set}{\lbrace}{\rbrace}
\renewcommand*\familydefault{\sfdefault}
\definecolor{darkish-blue}{RGB}{25,103,185}

\hypersetup{
  colorlinks,
  citecolor=darkish-blue,
  filecolor=darkish-blue,
  linkcolor=darkish-blue,
  urlcolor=darkish-blue
}

\theoremstyle{definition}
\newtheorem{ex}{Example}[subsection]

\crefname{ex}{Example}{Example}

\setlength{\marginparwidth}{1.5in}
\newcommand{\lecture}[1]{\marginpar{{\footnotesize $\leftarrow$ \underline{#1}}}}

\newcommand{\defnterm}[1]{\textbf{\textcolor{teal}{#1}}\index{#1}}

\newcolumntype{L}[1]{>{\raggedright\let\newline\\\arraybackslash\hspace{0pt}}m{#1}}
\newcolumntype{C}[1]{>{\centering\let\newline\\\arraybackslash\hspace{0pt}}m{#1}}
\newcolumntype{R}[1]{>{\raggedleft\let\newline\\\arraybackslash\hspace{0pt}}m{#1}}

\allowdisplaybreaks

\makeatletter
\def\blfootnote{\gdef\@thefnmark{}\@footnotetext}
\makeatother

%%%%%%%%%%%%%%%%%%%%%
%% D O C U M E N T %%
%%%%%%%%%%%%%%%%%%%%%

\begin{document}
\let\ref\Cref
\pagenumbering{roman}

\title{\bf{SCI 206: The Physics of How Things Work}}
\date{Fall 2016, University of Waterloo \\ \center Notes written from Stefan Idziak's lectures.}
\author{Charles Shen}

\blfootnote{Feel free to email feedback to me at
\href{mailto:ccshen902@gmail.com}{ccshen902@gmail.com}.}

\maketitle
\thispagestyle{empty}
\newpage
\tableofcontents
\newpage
\pagenumbering{arabic}

%%%%%%%%%%%%%%%%%
%% M O T I O N %%
%%%%%%%%%%%%%%%%%

\section{Motion}
\subsection{Inertia}
\defnterm{Inertia} states that a body in motion tends to remain in motion; a body at rest tends to remain at rest. \\

At any particular moment, you're located at a \defnterm{position}---that is, it is a specific point in space. \\
Whenever a position is reported, it is always as a \defnterm{distance} and \defnterm{direction} from some reference point. \\

Position is an example of a vector quantity. \\
A \defnterm{vector quantity} consists of both a magnitude and a direction; the \defnterm{magnitude} tells you how much of the quantity there is, while the \defnterm{direction} tells you which way the quantity is moving. \\

\defnterm{Velocity} measures the rate at which your position is changing with time. \\
Its magnitude is your \defnterm{speed}, the distance you travel in a certain amount of time,
$$\text{speed} = \frac{\text{distance}}{\text{time}}$$
and its direction is the direction in which you're heading. \\
A velocity of zero is special, because it has no direction. \\

When your velocity is constant and doesn't changes, you \defnterm{coast}.

\subsection{Newton's Law}
\defnterm{Newton's First Law of Motion}: an object that is not subject to any outside forces moves at a constant velocity, covering equal distances in equal times along a straight-line path. \\

The outside influences referred in the first law are called \defnterm{forces}, a technical term for pushes and pulls. \\
An object that moves in accordance with Newton's first law is said to be inertial. \\

\defnterm{Mass} is the measure of your inertia, your resistance to changes in velocity.
Almost everything in the universe has mass. \\
Mass has no direction, so it's not a vector quantity. \\
It is a \defnterm{scalar quantity}---that is, a quantity that only has an amount. \\

When something pushes on you, your velocity changes; in other words, you accelerate. \\
\defnterm{Acceleration}, a vector quantity, measures the rate at which your velocity is changing with time. \\
Any changes in velocity is acceleration. \\

You accelerate in response to the \defnterm{net force} you experience---the sum of all the individual forces being exerted on you.
\begin{align*}
\text{acceleration} &= \frac{\text{net force}}{\text{mass}} \\
\text{net force} &= \text{mass} \cdot \text{acceleration}
\end{align*}
This relationship is known as \defnterm{Newton's Second Law of Motion};
the net force exerted on an object is equal to the object's mass times its acceleration.
The acceleration is in the same direction as the net force. \\

\defnterm{Newton's Third Law of Motion}: for every force that one object exerts on a second object, there is an equal but oppositely directed force that the second object exerts on the first object. \\

Forces that are directed exactly away from surfaces are called \defnterm{normal forces}, since the term \defnterm{normal} is used by mathematicians to describe something that points exactly away from a surface---at a right angle or perpendicular to that surface.

\subsection{Weight and Gravity}
\defnterm{Gravity} is the physical phenomenon that produces attractive forces between every pair of objects in the universe. \\

Earth exerts a downward force on any object near its surface.
That object is attracted directly toward the centre of Earth with a force we call the object's \defnterm{weight}.
This weight is exactly proportional to the object's mass. \\
Weight is how hard gravity pulls on an object, and mass is how difficult an object is to accelerate.

We have
$$\text{weight} = \text{mass} \cdot \text{acceleration due to gravity}$$
In other words: you can lose weight by either reducing your mass or by going someplace, like a small planet, where gravity is weaker.

The Earth's acceleration due to gravity is $9.8N/kg$ or $9.8m/s^{2}$.

\subsection{The Velocity of a Falling Ball}
The current (present) velocity can be calculated as
$$\text{present velocity} = \text{initial velocity} + \text{acceleration} \cdot \text{time}$$

\subsection{The Position of a Falling Ball}
The average velocity is exactly halfway in between the two individual velocities:
$$\text{average velocity} = \text{initial velocity} + \frac{1}{2}\cdot\text{acceleration}\cdot\text{time}$$

The present position can be calculated as the following:
$$\text{present position} = \text{initial position} + \text{initial velocity}\cdot\text{time} + \frac{1}{2}\cdot\text{acceleration}\cdot\text{time}^{2}$$

\subsection{Tossing the Ball Upward}
The larger the initial velocity of the ball, the longer it rises and the higher it goes before its velocity is reduced to zero. \\
It then descends for the same amount of time it spent rising. \\
The higher the ball goes before it begins to descent, the longer it takes to return to the ground and the faster it's travelling when it arrives.

\subsection{Energy}
The capacity to make things happen is called \defnterm{energy}, and the process of making them happen is called \defnterm{work}. \\
Energy and work are both physical quantities, meaning that both are measurable. \\

Physical \defnterm{energy} is defined as the capacity to do work. \\
Physical \defnterm{work} refers to the process of transferring energy. \\

Energy is what's transferred, and work does the transferring.
The most important characteristic of energy is that it's conserved.
In physics, a \defnterm{conserved quantity} is one that can't be created or destroyed but that can be transferred between objects or, in the case of energy, be converted from one form to another. \\

When you throw a ball, it picks up speed and undergoes an increase in \defnterm{kinetic energy}, energy of motion that allows the ball to do work on whatever it hits. \\
When you list a rock, it shifts farther from Earth and undergoes an increase in \defnterm{gravitational potential energy}, energy stored in the gravitational forces between the rock and Earth that allows the rock to do work on whatever it falls on. \\
\defnterm{Potential energy} is energy stored in the forces between or within objects.

\subsection{Doing Work}
To do work on an object, you must push on it while it moves in the direction of your push. \\
This relationship can be expressed as the following:
$$\text{work} = \text{force} \cdot \text{distance}$$
\emph{If you're not pushing or it's not moving, then you're not working.} \\

Energy has no direction.
It can be hidden as potential energy. \\
\emph{Kinetic energy} is the form of energy contained in an object's motion. \\
\emph{Potential energy} is the form of energy stored in the forces between or within objects.

\subsection{Gravitational Potential Energy}
$$\text{gravitational potential energy} = \text{mass} \cdot \text{acceleration due to gravity} \cdot \text{height}$$
\emph{The higher it was, the harder it hits}.

A ramp provides \defnterm{mechanical advantage}, the process whereby a mechanical device redistributes the amount of force and distance that go into performing a specific amount of mechanical work.

\subsection{The Seesaw}
Overall movement of an object from one place to another is called the \defnterm{translational motion}. \\
A seesaw does not experience translational motion, it does turn around the pivot in the centre, and thus it experiences a different kind of motion. \\
Motion around a fixed amount (which prevents translation) is called \defnterm{rotational motion}. \\

The concepts and laws of rotational motion have many analogies in the concepts and laws of translational motion.

\subsection{The Motion of a Dangling Seesaw}

\defnterm{Rotational inertia} states that a body that's rotating tends to remain rotating; a body that is not rotating tends to remain not rotating. \\
At any particular moment, the seesaw is oriented in a certain way---that is, it has an \defnterm{angular position}.
It describes the seesaw's orientation relative to some reference orientation. \\

\defnterm{Angular velocity} measures the rate at which the seesaw's angular position is changing with time. \\
Its magnitude is the \defnterm{angular speed}, the angle through which the seesaw turns in a certain amount of time,
$$\text{angular speed} = \frac{\text{change in angle}}{\text{time}}$$
and its direction is the axis about which that rotation proceeds. \\
The seesaw's \defnterm{axis of rotation} is the line in space about which the seesaw is rotating.

\defnterm{Torques} is a technical term for twists and spins; its SI unit is \emph{newton-meter} ($N \cdot m$). \\

\defnterm{Newton's First Law of Rotational Motion} \\
A rigid object that is not wobbling and is not subject to any outside torques rotates at a constant angular velocity, turning equal amounts in equal times about a fixed axis of rotation. \\

\subsection{The Seesaw's Centre of Mass}
There is a special point in or near a free object about which all of its mass is evenly distributed and about which it naturally spins---its \defnterm{centre of mass}. \\
The axis of rotation passes right through this point so that, as the free object rotates, the centre of mass doesn't move unless the object has an overall translational velocity. \\

The point around which all the physical quantities of rotation are defined is the \defnterm{centre of rotation} or pivot. \\
For a free object, the natural pivot point is its centre of mass. \\
For a constrained object, the best pivot point may be determined by the constraints. \\

\subsection{How the Seesaw Responds to Torques}
The \defnterm{rotational mass} is the measure of an object's \emph{rotational} inertia, its resistance to changes in its \emph{angular} velocity;
also known as \textbf{\textcolor{teal}{moment of inertia}}\index{moment of inertia|see {rotational mass}}. \\
An object's rotational mass depends both on its ordinary mass and how that mass is distributed within the object. \\
The SI unit for rotational mass is \emph{kilogram-meter\textsuperscript{2}} ($kg \cdot m^2$). \\

An object's rotational mass depends on how far its mass is from the axis of rotation, so its rotational mass may change when its axis of rotation changes, even if it's rotating about its centre of mass. \\
Less torque is required to spin a tennis racket about its handle (racket's mass is fairly close to the axis and the rotational mass is small) than to flip the racket head-over-handle (both the head and the handle are far away fro m the axis and the rotational mass is large). \\

When something exerts a torque on an object, its angular velocity changes; i.e. it undergoes angular acceleration. \\
The \defnterm{Angular acceleration} measures the rate at which the \emph{angular} velocity is changing with time;
The SI unit is \emph{radian per second\textsuperscript{2}} ($1/s^{2}$). \\
It's analogous to acceleration, which measures the rate at which an object's \emph{translational} velocity is changing with time. \\
Just as with acceleration, angular acceleration involves both a magnitude and a direction.
An object undergoes angular acceleration when its angular speed increases or decreases or when its angular velocity changes direction. \\
If an object experiences several torques at once, it undergoes angular acceleration in response to the \defnterm{net torque} it experiences, the sum of all the individual torques being exerted on it.
$$\text{angular acceleration} = \frac{\text{net torque}}{\text{rotational mass}}$$

The relation is also known as \defnterm{Newton's Second Law of Rotational Motion}.
$$\text{net torque} = \text{rotational mass} \cdot \text{angular acceleration}$$
\emph{Spinning a marble is much easier than spinning a merry-go-round}. \\

This law does not apply to nonrigid or wobbling objects, because nonrigid objects can change their rotational masses and wobbling ones are affected by more than one rotational mass simultaneously.

\subsubsection{Summary}
\begin{enumerate}
  \item Your angular position indicates exactly how you're oriented
  \item Your angular velocity measures the rate at which your angular position is changing with time
  \item Your angular acceleration measures the rate at which your angular velocity is changing with time
  \item For you to undergo angular acceleration, you must experience a net torque
  \item The more rotational mass you have, the less angular acceleration you experience for a given torque
\end{enumerate}

\subsection{Forces and Torques}
A force can produce a torque and a torque can produce a force. \\

Nothing happens if you push on the seesaw exactly where the pivot passes through it;
there's no angular acceleration! \\
If you move a little away from the pivot, you can get the seesaw rotating, but you have to push hard. \\
You do much better if you push on the end of the seesaw, where even a small force can start the seesaw rotating. \\
The shortest distance and the direction from the pivot to the place where you push on the seesaw is a vector quantity called the \defnterm{lever arm};
in general, the longer the lever arm, the less force it takes to cause a particular angular acceleration. \\
The torque is proportional to the length of the lever arm;
i.e. you obtain more torque by exerting force farther from the pivot or axis of rotation. \\

When your force is directed parallel to the lever arm, it produces no torque. \\
To produce a torque, your force must have a component that is perpendicular to the lever arm and only that perpendicular component contributes to the torque. \\

The torque produced by a force is equal to the lever arm times that force, where we include on the component of the force that is perpendicular to the lever arm.
$$\text{torque} = \text{lever arm} \cdot \text{force perpendicular to the lever arm}$$
\emph{When twisting an unyielding object, it helps to use a long wrench}.

\subsection{Balancing and Unbalancing the Seesaw}
When a child sits on one end of the seesaw, the child's weight produces a torque on the seesaw about its pivot.
Gravity is ultimately responsible for that torque, so it can be referred to as \emph{gravitational} torque. \\

A \emph{balanced} seesaw experiences no overall gravitational torque about its pivot. \\
If nothing twists it, the balanced seesaw is inertial---the net torque on it its zero. \\
When the children aren't fidgeting, the seesaw is rigid and wobble-free, so it rotates at constant angular velocity, in according to Newton's first law of rotational mass.
If it's motionless, it stays motionless.
It it's running, it continues turning at a steady pace. \\

Two children with unequal weight can also balance the seesaw by sitting at different distances from the pivot.
A heavier child should sit closer to the pivot, and vice versa. \\

With two children on the seesaw, it has a considerable weight but that weight produces no torque on the seesaw about its pivot. \\
Because the seesaw's \defnterm{centre of gravity}, the effective location of the seesaw's overall weight, is located at the pivot. \\
With the centre of gravity located at the pivot, gravity has no lever arm with which to produce a torque on the seesaw about the pivot. \\

Any object or system of objects has a centre of gravity---the effective location of its overall weight. \\
While centre of gravity is a gravitational concept and centre of mass is an inertial concept, the two conveniently coincide;
the seesaw's centre of gravity and its centre of mass are located at the same point.
That coincidence stems from the proportionally between an object's weight and its mass near Earth's surface. \\

\subsection{Levers and Mechanical Advantage}
The seesaw's mechanical advantage allows even the tiniest child at its end to list the heaviest adult sitting near its pivot.
Because the child travels much farther than the adult, the child's work on the seesaw equals the seesaw's work on the adult. \\
This effect---a small force exerted for a short distance elsewhere in that system---is an example of the mechanical advantage associated with levers. \\

Also, each child is using the seesaw board to exert a torque on the other child about the pivot, and that torque cancels the other child's gravitational torque. \\
This is \defnterm{Newton's Third Law of Rotational Mass}: if one object exerts a torque on a second object, then the second object will exert an equal but oppositely directed torque on the first object. \\

\emph{Note that} not all equal but oppositely directed torques are Newton's third law pairs. \\
On a balanced seesaw, each child experiences two torques that \emph{happen} to be equal but oppositely directed: a torque due to gravity and a torque due to the other child.
Those two torques form a matched pair because the seesaw is balanced, not because of Newton's third law. \\
If the seesaw were not balanced, those two torques would not be equal but oppositely directed and the children would both be experiencing angular acceleration.

\subsection{Friction}
\defnterm{Friction} is a phenomenon that opposes the relative motion of two surfaces in contact with each one another. \\
Two surfaces that are in \defnterm{relative motion} are travelling with different velocities so that a person standing still on one surface will observe the other surface as moving.
In opposing relative motion, friction exerts forces on both surfaces in directions that tend to bring them to a single velocity. \\

Frictional forces always oppose relative motion, but they vary in strength according to
\begin{enumerate}
  \item how tightly the two surfaces are pressed against one another \\
  i.e. the harder you press two surfaces together, the larger the frictional forces they experience.
  \item how slippery the surfaces are \\
  i.e. roughening the surfaces generally increases friction, while smoothing or lubricating them generally reduces it
  \item whether the surfaces are actually moving relative to one another
\end{enumerate}

\subsection{A Microscopic View of Friction}
All surfaces experience friction as they rub across one another at various contact points. \\
When the surfaces slide across one another, these contact points collide, producing sliding friction and wear. \\

Increasing the size or number of microscopic projections (contact points) by roughening the surfaces generally leads to more friction. \\
Increasing the number of contact points by squeezing the two surfaces more rightly together also leads to more friction.
The microscopic projections simply collide more often. \\

A useful rule of thumb is that the frictional forces between two firm surfaces are proportional to the forces pressing those two surfaces together. \\

Friction also causes wear when the colliding contact points break one another off.
With time, this wear can remove large amounts of material so that even seemingly indestructible stone steps are gradually worn away by foot traffic. \\
The best way to reduce wear between two surfaces is to polish them so that they are extremely smooth.
The smooth surfaces will still touch at contact points and experience friction as they slide across one another, but their contact points will be broad and round and will rarely break one another off during a collision.

\subsection{Static Friction, Sliding Friction, and Traction}
When two surfaces are moving across one another, \defnterm{sliding friction} acts to stop them from sliding. \\

But even when those surfaces have the same velocity, \defnterm{static friction} may act to keep them form starting to slide across one another in the first place. \\
Static friction always exerts a frictional force that exactly balances your push.
However, the force that static friction can exert is limited. \\

By exerting more horizontal force on it than static friction can exert in the other direction;
the net force is then no longer zero, and the object accelerates. \\
Once the object is moving, static friction is replaced by sliding friction.
Because sliding friction acts to bring the object back to reset, and the object must be kept pushing to keep it moving. \\
As the object is being pushed, the contact points between the surfaces no longer have time to settle into one another, and they consequently experience weaker horizontal forces. \\
Thus, force of sliding friction is generally weaker than that of static friction. \\

Both types of frictions (sliding and static) are incorporated in the concept of \defnterm{traction}, the largest amount of frictional force that one surface can obtain from another surface at any given moment. \\
When an object is at rest, its traction is equal to the maximum amount of static friction. \\
When the object is moving, its traction reduces to the amount of force that sliding friction exerts. \\

Another difference between static and sliding friction is that sliding friction wastes energy. \\
It converts useful, \defnterm{ordered energy}, work or energy that can easily do work, into relatively useless \defnterm{thermal energy}, a disordered energy that's associated with temperature. \\
Sliding friction makes things hotter by turning work into thermal energy. \\

In the case of a file cabinet, work becomes thermal energy as your slide the cabinet across the floor at \emph{constant} velocity.
Since the cabinet isn't accelerating, you and the floor are pushing the cabinet equally hard in opposite directions. \\
Since you're pushing the cabinet forward as it moves forward, you're doing positive work on the cabinet.
Sliding friction from the floor, however, is pushing the cabinet backward as it moves forward, so sliding friction is doing negative work on the cabinet. \\
Overall, sliding friction is removing energy from the cabinet just as fast you're adding it. \\

Static friction doesn't waste energy as thermal energy;
it enables objects to do work on one another.

\subsection{Friction and Energy}
Sliding friction converts work into thermal energy. \\

Recall that energy is the capacity to do work, and it is transferred between objects by doing that work. \\
Energy is a conserved physical quantity---it cannot be created or destroyed;
it can only be transferred between objects or converted from one form to another.  \\
Energy can be thought of the \emph{conserved quantity of doing}. \\

Energy's two basic forms are kinetic (energy contained in the motions of objects) and potential (energy stored in the forces between or within those objects). \\
The individual forms of energy are \emph{not} conserved quantities;
only \emph{total} energy---the sum of all the individual forms---is conserved. \\

Thermal energy is a mixture of kinetic and potential energies. \\
The kinetic energy and potential energies in thermal energy are contained entirely within those objects. \\
Any object has \defnterm{internal energy}, energy held entirely within that object by its individual particles and forces between those particles;
\emph{thermal energy} is the portion of internal energy that's associated with temperature. \\
Thermal energy makes particle in the object jiggle randomly and independently;
at any moment, each particle has its own tiny supply of potential and kinetic energies, and this dispersed and disordered energy is collectively referred to as thermal energy. \\

As you push a file cabinet across the floor, you do work on it, but it doesn't pick up speed.
Instead, sliding friction converts your work into thermal energy, and the cabinet and floor become hotter as the energy you transfer to them disperses among their particles. \\
But although sliding friction easily turns work into thermal energy, there's no easy way to turn thermal energy back into work. \\
Disorder not only makes things harder to use, but it also is difficult to undo.
Suppose a mug is dropped on the floor and it shatters into a thousand pieces, the mug is still there;
however, it's disordered and thus much less useful.

\subsection{Wheels}
As a wheel spins, it lowers a portion of its surface onto the surface, leaves it there briefly so that it may experience taking its place. \\
Due to the touch-and-release character of rolling, there is no sliding friction between the wheel and the surface. \\

As the wheel rotates, its hub slides across the stationary axle at its centre. \\
This sliding friction wastes energy and causes wear to both hub and axle. \\
While the narrow hub moves relatively slowly across the axle so that the work and wear done each second are small, sliding friction is undesirable. \\

A solution is to insert rollers between the hub and axle. \\
The result is a roller bearing, a mechanical device that eliminates sliding friction between a hub and an axle. \\
A complete bearing consists of two rings separated by rollers that keep those rings from rubbing against one another.
The bearing's outer inner ring is attached to the stationary axle, while its outer ring is attached to the spinning wheel hub.

\subsection{Power and Rotational Work}
\defnterm{Power} is the physical quantity that measures the rate at which a car's engine does work---the amount of work it does in a certain amount of time, or
$$\text{power} = \frac{\text{work}}{\text{time}}$$
The SI unit of power is the \textbf{joule per second}, also called \textbf{watt}. \\

The car's engine does work on each its powered wheels in the context of \emph{rotational} motion, where work involves torque and angle. \\
The engine does work on a wheel by exerting a torque on the wheel as the wheel rotates through an angle in the direction of the torque. \\

To do work on an object via rotational motion, you must twist it as it rotates through an angle in the direction of your twist. \\
Work can be calculated as:
$$\text{work} = \text{torque} \cdot \text{angle}(\text{in radians})$$
\emph{If you're not twisting or it's not turning, then you're not working}. \\

The engine does work on its wheels via rotational motion, those wheels do work on the car via translational motion. \\
Each wheel grips the pavement with static friction, its contact point with the ground can't move.
Instead, the rotating wheel pushes the axle and car forward as they move forward.
So energy that flows into the wheels via rotational motion returns to the car via translational motion and keeps the car moving. \\

Power can be supplied by translational motion or rotational motion:
$$\text{power} = \text{force} \cdot \text{velocity}$$
$$\text{power} = \text{torque} \cdot \text{angular velocity}$$

\subsection{Kinetic Energy}
Car's brakes are designed to stop the car by turning its kinetic energy into thermal energy.
Brake rubs stationary brake pads against spinning metal discs. \\

A way to calculate kinetic energy is
$$\text{kinetic energy} = \frac{1}{2} \cdot \text{mass} \cdot \text{speed}^{2}$$

\emph{Racing around at twice the speed takes four times the energy}. \\
The dramatic increase in kinetic energy that results from a modest increase in speed explains why high-speed crashes are far deadlier than those at lower speeds. \\

Rotating objects have kinetic energy as well. \\
The kinetic energy of rotational motion depends on inertia and speed;
however, it's the rotational inertia and rotational speed that matter
$$\text{kinetic energy} = \frac{1}{2} \cdot \text{rotational mass} \cdot \text{angular speed}^{2}$$
\emph{It takes a very energetic person to spin his wheels twice as fast}.

\subsection{Coasting Forward: Linear Momentum}
A bumper car's translational inertia makes it hard to start or stop, and its rotational inertia makes it difficult to spin or stop from spinning. \\
There are two other conserved quantities besides energy in nature: \emph{linear momentum} and \emph{angular momentum}. \\

When bumper cars crash into one another, they exchange more than just energy. \\
Energy is direction-less since it's a scalar quantity. \\
The direction of travel after a collision is a conserved vector quantity called linear momentum. \\
\defnterm{Linear momentum}, or \defnterm{momentum}, is the measure of an object's translational impetus or tenacity---its determination to keep moving the way it's currently moving. \\
In other words, momentum is the \emph{conserved quantity of moving};
it can neither be created nor destroyed, only transferred between objects. \\
It can be calculated as such
$$\text{momentum} = \text{mass} \cdot \text{velocity}$$
\emph{It's hard to stop a fast-moving truck}. \\
The faster an object is moving or the more mass it has, the more momentum it has in the direction of its velocity. \\
The SI unit of momentum is \emph{kilogram-meter-per-second} ($kg \cdot m / s$).

\subsection{Exchanging Momentum in a Collision: Impulses}
Momentum is transferred to an object by giving it an \defnterm{impulse}, a force exerted on it for a certain amount of time. \\
When the motor and floor pushes your bumper car forward for a few seconds, they give your car an impulse and transfer momentum to it.
This impulse is the change in your car's momentum \\
Impulse can be calculated as the following
\begin{align*}
\text{impulse} &= \text{force} \cdot \text{time} \\
\Delta p &= F \cdot t
\end{align*}

The more force or the longer that force is exerted, the larger the impulse and the more an object's momentum changes. \\
Like momentum itself, an impulse is an vector quantity. \\

Forces that colliding objects exert on one another as they exchange momentum are called \defnterm{impact forces}. \\
Bumper cars have soft rubber bumpers instead of hard steel bumpers because a collision would last only an instant and would involve enormous impact forces. \\

A moving object carries only momentum, not force. \\
A coasting object is free of any net force.
However, when that object hits an obstacle, the two exchange momentum via impulses, and impulses involve forces.

\subsection{Spinning in Circles: Angular Momentum}
Another conserved vector quantity during a collision is \emph{angular momentum}. \\
\defnterm{Angular momentum} is the measure of an object's rotational impetus or tenacity, its determination to keep rotating the way it's currently rotating. \\
Angular momentum is the \emph{conserved quantity of turning}. \\
It can be calculated as the following
$$\text{angular momentum} = \text{rotational mass} \cdot \text{angular velocity}$$

Angular momentum has the same direction as the angular velocity;
the faster an object is spinning or the larger its rotational mass, the more angular momentum it has in the direction of its angular velocity. \\
The SI unit of angular momentum is \emph{kilogram-meter\textsuperscript{2} per second} ($kg \cdot m^{2} / s$). \\

Angular momentum is also a conserved quantity;
it cannot be created or destroyed, only transferred between objects. \\
For an object to start spinning, something must be transfer angular momentum to it, and the object will then continue to spin until it transfer this angular momentum elsewhere.

\subsection{Glancing Blows: Angular Impulses}
Angular momentum is transferred to an object by giving it an \defnterm{angular impulse}, which is a torque exerted on it for a certain amount of time. \\
Angular impulse is the change in an object's angular momentum and it can be found via
$$\text{angular impulse} = \text{torque} \cdot \text{time}$$
\emph{To get a merry-go-round spinning rapidly, you must twist it hard and for a long time}. \\

An angular impulse is a vector quantity and has the same direction as the torque. \\
As with linear momentum, sudden transfers of angular momentum can break things, so bumper cars are designed to limit their \emph{impact torques} to reasonable levels. \\

Angular momentum is conserved because of Newton's third law of rotational motion. \\
When an object exerts a torque on another object for a certain amount of time, the other object exerts an equal but oppositely directed torque on first object for exactly the same amount of time. \\

By reducing rotational mass, say pulling arms in when spinning, a skater will spin more rapidly since the said skater experiences zero net torque and angular momentum must remain constant. \\
If the skater spreads out his/her arms, there's more rotational mass and thus the spinning slows down.

\subsection{Potential Energy, Acceleration, and Force}
A \defnterm{gradient}, a vector quantity, characterizes how some physical quantity changes gradually with position.
It points in the direction of the fastest increase in its physical quantity, and its magnitude is the rate of that increase. \\

Suppose a car going up a ramp as a potential energy gradient;
that is, the car's potential energy increases as the car moves up the ramp gradually. \\
This gradient, known as \defnterm{potential gradient}, points uphill---in the direction of the quickest potential energy increase---and its magnitude is the increase in the car's potential energy caused by a small uphill step divided by the length of that step. \\
The steeper the ramp, the larger the car's potential gradient. \\

Note that the car's potential gradient and its acceleration point in exactly opposite directions. \\
The car's potential gradient points in the direction that \emph{increases} its potential energy as quickly as possible, whereas car accelerates in the direction that \emph{reduces} its potential energy as quickly as possible. \\
The negative of the car's uphill potential gradient is the downhill force on the car
$$\text{force} = -\text{potential gradient}$$
\emph{An object accelerates in the direction that reduces its potential energy as quickly as possible}.

%%%%%%%%%%%%%%%%%%%%%%%
%% R E S O N A N C E %%
%%%%%%%%%%%%%%%%%%%%%%%

\newpage
\section{Resonance}
\subsection{Stretching a Spring}
There's a relationship between the length of a spring and the forces it is exerting on its ends. \\
The spring scale, which uses a spring, determines how much upward force it is exerting on an object by measuring the length of its spring. \\

If a spring is neither stretched nor compressed, it exerts no forces on its ends. \\
If no other forces act this relaxed spring, each end is in \defnterm{equilibrium}---it experiences zero net force.
Equilibrium occurs whenever the forces acting on an object sum to zero so that the object does not accelerate. \\
So the relaxed spring is at its \defnterm{equilibrium length}, its natural length when no forces are exerted on its ends.
The spring tries to return to this equilibrium length no matter how you distort it. \\

Suppose the left end of a spring is fixed to a wall and its right end to a move-able bar.
When the bar isn't pulling or pushing on the springs right end, that end is in equilibrium at a particular location–its \defnterm{equilibrium position}.
Since the spring's right end naturally returns to this equilibrium position if the bar disturbs it and then lets go, the end is in a \defnterm{stable equilibrium}. \\
Now suppose that the bar moves the spring's right end to the right.
The stretched spring now exerts a leftward force on the right end, trying to restore the spring to its equilibrium position, called \defnterm{restoring force}. \\
The spring exerts restoring forces on both of its end, pulling them equally hard in opposite directions.;
however, the left end is held in place by the wall. \\
The farther the bar stretches the spring to the right, the stronger its restoring forces on the right end becomes.
The restoring force is exactly proportional to how far the bar has stretched the spring. \\
If the bar compresses the spring, the restoring force now exerts a rightward force on its right end. \\

Whenever a spring is distorted away from its equilibrium length, it exerts a restoring force on its move-able end that is proportional to the distortion and points in the opposite direction;
this is known as \defnterm{Hooke's Law} and the law can be presented as
$$\text{restoring force} = - \text{spring constant} \cdot \text{distortion}$$
\emph{The stiffer the spring and the farther you stretch it, the harder it pulls back}. \\
The \defnterm{spring constant} is a measure of the spring's stiffness.
The larger the constant---that is, the firmer the spring---the larger the restoring force the spring exerts for a given distortion. \\

Hooke's Law is not limited to only the behaviour of coil springs. \\
Many objects respond to distortion with restoring forces that are proportional to how far you've distorted them away from their equilibrium lengths---or their equilibrium shapes.
\defnterm{Equilibrium shape} is the shape an object adopts when it's not subject to any outside forces. \\

There's a limit to Hooke's law;
if you distort an object too far, it will usually begin to exert less force than Hooke's law demands. \\
This is due to the object has exceeded its \defnterm{elastic limit} and it will have deformed it permanently in the process. \\
If you stretch a spring too far, it won't return to its equilibrium length when you release it. \\

Distorting a spring requires work;
when it's being pulled, there's work being done as some energy is transferred to the spring. \\
This energy is \defnterm{elastic potential energy}. \\
If you reverse the motion, the spring returns most of this energy to your hand, and the remainder is converted to thermal energy by frictional effects inside the spring itself. \\
A spring that is distorted away from its equilibrium shape always contains elastic potential energy.

\subsection{Natural Resonances}
Practical clocks are based on a particular type of repetitive motion called a \defnterm{natural resonance}.
In a natural resonance, the energy in an isolated object or system of objects causes it to perform a certain motion over and over again. \\
An object that has been displaced from its stable equilibrium accelerates toward that equilibrium but then overshoots;
it coasts right through equilibrium and must turn around to try again.
As long as it has excess energy, this object continues to glide back and forth through its equilibrium and thus exhibits a natural resonance.

\subsection{Pendulums and Harmonic Oscillators}
A pendulum's centre of gravity is directly below its pivot, it is in a stable equilibrium.
Its centre of gravity is then as low as possible, so displacing it raises its gravitational potential energy and a restoring force begins pushing it back toward that equilibrium position. \\
This restoring force is almost exactly proportional to how far the pendulum is from equilibrium. \\
The pendulum swings back and forth about its equilibrium position in a repetitive motion called an \defnterm{oscillation}. \\
As it swings, its energy alternates between potential and kinetic forms.
This repetitive transformation of excess energy from one form to another is part of any oscillation and keeps the oscillator---the system experiencing the oscillation---moving back and forth until that excess energy is either converted into thermal energy or transferred elsewhere. \\

The pendulum is a \defnterm{harmonic oscillator}---the simplest and best understood mechanical system in nature---because its restoring forces is proportional to its displacement from equilibrium. \\
As a harmonic oscillator, the pendulum undergoes \defnterm{simple harmonic motion}, a regular and predictable oscillation that keeps passage of time with remarkable accuracy. \\

The \defnterm{period} of any harmonic oscillator, the time it takes to complete one full cycle of its motion, depends only on how stiffly its restoring force pushes it back and forth, and on how strongly its inertia resists the accelerations of that motion. \\
\defnterm{Stiffness} is the measure of how sharply the restoring force increases as the oscillator is displaced from equilibrium;
stiff restoring forces are associated with firm or hard objects, while less stiff restoring forces are associated with soft objects. \\
The stiffer the restoring force, the more forcefully it pushes the oscillator back and forth and the shorter the oscillator's period. \\
The larger the oscialltor's mass, the less it accelerates and the longer its period. \\

The most important characteristic of a harmonic oscillator is that its period \emph{doesn't} depend on \defnterm{amplitude}, its furthest displacement from equilibrium. \\
The insensitivity to amplitude is a consequence of its special restoring force, a restoring force that is proportional to its displacement from equilibrium. \\
The harmonic oscillator completes a large cycle of motion just as quickly as it completes a small cycle of motion. \\

In a pendulum, increasing its mass doesn't increase its period.
Because increasing the mass also increases its weight and therefore stiffens its restoring force.
These two changes compensate for one another perfectly so that the pendulum's period remains unchanged. \\
The period does depend on its length and on gravity. \\
When a pendulum's length is reduced, the restoring force is stiffed and hence shortens its period.\\
When you strength gravity, you increase the pendulum's weight, stiffen its restoring force, and reduce its period. \\
Note that materials expand with increasing temperature, so a simple pendulum slows down as it heats up.
A better, more accurate pendulum is thermally compensated by using several different materials with different coefficients of loudness expansion to ensure that its centre of mass remains at a fixed distance from its pivot.

\subsection{Pendulum Clocks}
The top of the pendulum has a two-pointed anchor that controls the rotation of a toothed wheel;
this mechanism is called \defnterm{escapement}. \\
A weighted cord wrapped around the toothed wheel's shaft exerts a torque on that wheel, so that the wheel would spin if the anchor weren't holding it in place. \\
Each time the pendulum reaches the end of a swing, one point of the anchor releases the toothed wheel while the other point catches it. \\
The wheel turns slowly as the pendulum rocks back and forth, advancing one tooth for each full cycle of the pendulum.
This wheel turns a series of gears, which slowly advances the clock's hands. \\

The toothed wheel also keeps the pendulum moving by giving the anchor a tiny forward push each time the pendulum completes a swing.
This work on the anchor and pendulum replaces energy lost to friction and air resistance. \\
This energy comes from the weighted cord, which releases gravitational potential energy as its weight descends.
When you wind the clock, you rewind this cord around the shaft, lifting the weight and replenishing its potential energy. \\

A clock works best when its pendulum swings with almost perfect freedom.
Any outside force---even the push from the toothed wheel---will influence the pendulum's period. \\
A good pendulum clock uses an aerodynamic pendulum and low-friction bearings. \\

If the pendulum is displaced too far, it becomes an \defnterm{anharmonic oscillator}---its restoring force ceases to be proportional to its displacement from equilibrium, and its period begins to depend on its amplitude.

\subsection{Balance Clocks}
A small wheel attached to a coil spring is a harmonic oscillator known as a \defnterm{balance ring}, used in most mechanical clocks and watches.
Its period is determined only by the stiffness of the coil spring and the rotational mass of the wheel. \\
It resembles a tiny metal bicycle wheel, supported at its centre of mass/gravity by an axle and a pair of bearings. \\
Any friction in the bearings is exerted so close to the ring's axis of rotation that it produces little torque and the ring turns extremely easily.
The ring pivots about its own center of gravity so that its weight produces no torque on it. \\

The only thing exerting a torque on the balance ring is a tiny coil spring. \\
One end of this spring is attached to the ring, while the other is fixed to the body of the clock.
When the spring is not distorted, it exerts on torque on the ring and the ring is in equilibrium.\\
Since this restoring torque is proportional to the ring's rotation away from a stable equilibrium, the balance ring and coil spring form a harmonic oscillator. \\

The period of the ring depends on the \emph{torsional} stiffness of the coil spring, that is, on how rapidly the spring's torque increases as you twist it, and on the balance ring's \emph{rotational} mass. \\
The balance ring's period doesn't depend on the amplitude of its motion. \\
Gravity exerts no torque on the balance ring, so this timekeeper works anywhere and in any orientation. \\

As the balance ring rocks back and forth, it tips a lever that controls the rotation of a toothed wheel.
An anchor attached to the lever allows the wheel to advance one tooth for each complete cycle of the balance ring's motion. \\
Gears connected to the tooth wheel to the clock's hands, which slowly advance as the wheel turns. \\
The balance clock can't draw energy from a weighted cord.
Instead, it has a main spring that exerts a torque on the toothed wheel. \\
The main spring is a coil of elastic metal that stores energy when you wind the clock.
Its energy keeps the balance ring rocking steadily back and forth and also turns the clock's hands.
The main spring unwinds as the toothed wheel turns, the clock occasionally needs winding.

\subsection{Electronic Clocks}
The potential accuracy of pendulum and balance clocks is limited by friction, air resistance, and thermal expansion to about 10 s per year. \\

Many modern clocks use quartz oscillators as their timekeepers. \\
A quartz oscillator is made from a single crystal of quartz.
A quartz crystal oscillates strongly after being struck. \\
It is a harmonic oscillator because it acts like a spring with a metal cylinder at each end.
The two cylinders oscillate in and out symmetrically about their combined center of mass, with a period determined only by the cylinders' masses and the spring's stiffness. \\
In a quartz crystal, the spring is the crystal itself and the cylinders' are its two halves.
Since their restoring forces are proportional to displacement from equilibrium, both systems are harmonic oscillators. \\

A quartz crystal's restoring forces are extremely stiff due to its hardness.
A tiny distortion leads to large restoring forces. \\
Since the period of a harmonic oscillator decreases as its spring becomes stiffer, a typical quartz oscillator has an extremely short period. \\
Its motion is usually called a \defnterm{vibration} rather than an oscillation because vibration implies a fast oscillation in a mechanical system.
\emph{Oscillation} itself is a more general term for any repetitive process, and it can even apply to such non-mechanical processes as electric or thermal oscillations. \\

A fast oscillator is characterized by its \defnterm{frequency}, the number of cycles it completes in a certain amount of time. \\
The SI unit of frequency is the \emph{cycle per second}, also called the \defnterm{hertz} (Hz)
$$\text{frequency} = \frac{1}{\text{period}}$$

Quartz clocks are normally electronic. \\
The crystal's vibrations are too fast and too small for most mechanical devices to follow. \\
A quartz crystal is intrinsically electronic itself;
it responds mechanically to electrical stress and electrically to mechanical stress. \\
Crystalline quartz is known as a \emph{piezoelectric} material and is ideal for electronic clocks.

\subsection{Sound and Music}
In air, \defnterm{sound} consists of density waves, patterns of compression and rarefaction that travel outward rapidly from their source. \\
When a sound passes by, the air pressure in your ear fluctuates up and down about normal atmospheric pressure. \\

When the fluctuations are repetitive, you hear a \emph{tone} with a pitch equal to the fluctuation's frequency.
\defnterm{Pitch} is the frequency of a sound.

\subsection{A Violin's Vibrating String}
The violin subjects its strings to \defnterm{tension}, outward forces that act to stretch it, and this tension gives each string an equilibrium shape---a straight line. \\
Tension exerts a pair of outward forces on each piece of the string;
its neighbouring pieces are pulling that piece toward them.
Since the string's tension is uniform, these two outward forces sum to zero;
they have equal magnitudes but point in opposite directions. \\
With zero net force on each of its pieces, the straight string is in equilibrium. \\

When the string is curved, the pairs of outward forces no longer sum to zero;
they still have equal magnitudes and pointing in different directions. \\
Each piece experiences a small net force. \\
The net forces on its pieces are restoring forces because they act to straighten the string.
If the string is distorted and then released, these restoring forces will cause the string to vibrate about its straight equilibrium shape in a natural resonance. \\
The more the string is curved, the stronger the restoring forces on its pieces become.
The restoring forces are \defnterm{springlike forces}---they increase in proportion to the string's distortion---so the string is a form of harmonic oscillator. \\

A string can bend and vibrate in many distinct \defnterm{modes}, or basic patterns of distortion, each with its own period of vibration. \\
The period of each vibrational mode is independent of its amplitude. \\

A violin has one simplest vibration---its \defnterm{fundamental vibrational mode}.
The string's kinetic energy peaks as it rushes through its straight equilibrium shape, and its elastic potential energy peaks as it stops to turn around. \\
The string's midpoint travels the farthest (the \defnterm{vibrational antinode}), while its ends remain fixed (the \defnterm{vibrational nodes}). \\
Its vibrational period depends only on the stiffness of its restoring forces and on its inertia.
Either stiffening the violin string or reducing its mass will quicken its fundamental vibration and increase its fundamental pitch. \\

A string's fundamental pitch also depends on its length.
Shortening the string both stiffens it and reduces its mass, so its pitch increases.
A shorter string curves more sharply when it's displaced from equilibrium and therefore subjects its pieces to larger net forces. \\

A \defnterm{mechanical wave}, the natural motions of an extended object about its stable equilibrium shape or situation. \\
An \emph{extended} object is one like a string, stick, or lake surface that has many parts that move with limited independence. \\
The string's fundamental mode is a particularly simple wave, a \defnterm{standing wave}, which is a wave with fixed nodes and antinodes.
A standing wave's basic shape doesn't change with time;
it merely scales up and down rhythmically at a particular frequency and amplitude (its peak extend of motion). \\
An important characteristic is that the standing wave doesn't travel along the string. \\
Not only does a standing wave extends along the string, its associated oscillation is \emph{perpendicular} to the string and therefore \emph{perpendicular} to the wave itself.
A wave in which the underlying oscillation is perpendicular to the wave itself is called a \defnterm{transverse wave}.

\subsection{The Violin String's Harmonics}
A violin string also has \defnterm{higher-order vibrational modes} in which the string vibrates as a chain of shorter strings arcing in alternate directions.
Each of these higher-order modes is another standing wave, with a fixed shape that scales up and down rhythmically at its own frequency and amplitude. \\
A string's vibrational frequency is inversely proportional to its length, so halving its length doubles its frequency. \\
Frequencies that are integer multiples of the fundamental pitches are called \defnterm{harmonics}, so a half-string vibration occurs at the second harmonic pitch and is called the \emph{second harmonic mode}. \\

The string often vibrates in more than one mode at the same time.
For instance, a string vibrating in its fundamental mode can also vibrate in its second harmonic and emit two tones at once. \\
\defnterm{Timbre} is the mixture of the fundamental tones and the harmonics in an instrument. \\
When a violin string is vibrating in several modes at once, its shape and motion are complicated.
The individual standing waves add on top of one another, a process known as \defnterm{superposition}. \\
Each vibrational mode has its own amplitude and therefore its own loudness contribution to the string's timbre. \\

While the waves have virtually no effect on one another, the string's overall distorted shape is the superposition of the individual wave shapes that changes substantially with time. \\
The different harmonic waves vibrate at different frequencies, and their superposition changes as they change. \\
The string's overall wave is not a standing wave, and its features can even move along the string.

\subsection{Bowing and Plucking the Violin String}
The bow exerts frictional forces on the strings as it moves across them. \\
The bow hairs grab the string and push it forward with static friction.
The string's restoring forces eventually overpowers the static friction, and the string starts sliding backward across the hairs. \\
Since the hairs exerts little sliding friction, the string completes half a vibrational cycle with ease.
When it stops to reverse direction, the hairs grab the string again and begin pushing it forward and this process is repeated over and over. \\

Each time the bow pushes the string forward, it does work on the string and adds energy to the string's vibrational modes.
This is called \defnterm{resonant energy transfer}, in which a modest force doing work in synchrony with a natural resonance can transfer a large amount of energy to that resonance. \\
Similar rhythmic pushes can cause other objects to vibrate strongly. \\
A wine glass' response to a certain tone is also an example of \defnterm{sympathetic vibration}, the transfer of vibrational energy between two systems that share a common vibrational frequency. \\

Bowing the string nearer its middle reduces the string's curvature, weakening its harmonic vibrations and giving it a more mellow sound. \\
Bowing the string nearer its end increases the string's curvature, strengthening its harmonic vibrations and giving it a brighter sound. \\

Sometimes a tone from an instrument or sound system will cause some object in the room to begin vibrating loudly. \\
The object has a natural resonance at the tone's frequency, and sympathetic vibration is transferring energy to the object. \\
Energy moves easily between two objects that vibrate at the same frequency. \\

Resonance energy transfer make it possible for sound to shatter a crystal wineglass. \\
When the sound pushes on the glass rhythmically, the sound slowly transfer energy to the glass, until it finally shatters.

\subsection{An Organ Pipe's Vibrating Air}
A pipe organ uses vibrations to create sound. \\
Its vibration takes place in the air itself.
The pipe is open at each end and filled with air.
Since that air is protected by rigid walls of the pipe, its pressure can fluctuate up and down relative to the atmosphere pressure and it can exhibit natural resonances. \\

In its fundamental vibrational mode, air moves alternately toward and away from the pipe's center.
As air moves toward the pipe's center, the density there rises and a pressure imbalance develops.
Since the pressure at the center is higher than its ends, air accelerates \emph{away} from the center. \\
The air eventually stops moving inward and moves outward.
As air moves away from the center, the density in the center drops and a reversed pressure imbalance occurs. \\
Air now accelerates \emph{towards} the center, and the cycle repeats. \\
The air's kinetic energy peaks each time it rushes through that equilibrium and its pressure potential energy peaks each time it stops to turn around. \\

The air column is a harmonic oscillator, where its vibrational frequency depends only on the stiffness of its restoring forces and on its inertia. \\
Stiffening the air column or reducing its mass will quicken its vibration and increase its pitch.
These characteristics depend on the length of the pipe. \\
A shorter pipe holds less air mass and it offers stiffer opposition to any movements of air in and out of that pipe.
With less room in the shorter pipe, the pressure inside it rises and falls more abruptly, hence stiffer restoring forces on the moving air. \\
An organ pipe's vibrational frequency is inversely proportional to its length. \\
The mass of vibrating air in a pipe also increases with the air's average density, so a change in temperature or weather will alter the pipe's pitch. \\

The fundamental vibrational mode of air in the organ is a standing wave. \\
Air in the pipe is an extended object with a stable equilibrium, and the disturbance associated with its fundamental vibrational mode has a basic shape that doesn't change with time;
it merely scales up and down rhythmically. \\

The shape of the wave in the pipe's air has back-and-forth compression and rarefaction, not with side-to-side displacements. \\
All wave's associated oscillation is \emph{along} the pipe and therefore \emph{along} the wave itself.
A wave in which the underlying oscillation is parallel to the wave itself is called a \defnterm{longitudinal wave}. \\
Waves in the air, including those inside organ pipes and other wind instruments and those in the open air, are all longitudinal waves. \\

\subsection{Playing an Organ Pipe}
The organ uses resonant energy transfer to make the air in a pipe vibrate.
As the air flows across the opening, it's easily deflected to one side or the other and tends to follow any air that's already moving into or out of the pipe. \\
If the air inside the pipe is vibrating, the new air will follow it in perfect synchrony and strength the vibration. \\

In its fundamental vibrational mode, the pipe's entire column of air vibrates together. \\
in the higher-order vibrational modes, this air column vibrates as a chain of shorter air columns moving in alternate direction.
If the pipe has a constant width, these vibrations occur at harmonics of the fundamental.
When the air column vibrates as two half-columns, its pitch is exactly twice that of the fundamental mode;
three third-columns lead to three times the fundamental, and so on. \\

The standing waves superpose and the fundamental and harmonic tones are produced together;
the sake of the organ pipe and the place where air is blown across it determine the pipe's harmonic content and thus its timbre.

\subsection{A Drum's Vibrating Surface}
A drumhead has a stable equilibrium and springlike restoring forces, its overtone vibrations \emph{aren't} harmonics. \\
A drumhead is effectively two-dimensional or surface-like, it doesn't divide easily into pieces that resemble the entire drumhead. \\

Since striking a drumhead causes it to vibrate in several modes at once, the drum emits several pitches simultaneously.
The amplitude of each mode, and consequently its loudness, depends on \emph{how hard} you hit the drumhead and also on \emph{where} you hit it.
Circular modes if hit at its centre;
non-circular if hit near its edge. \\

\subsection{Sound in Air}
Air is in a stable equilibrium when its density is uniform (neglecting gravity). \\
If disturbed, the resulting pressure imbalances will provide springlike restoring forces.
These forces, along with air's inertia, lead to rhythmic vibrations---the vibrations of harmonic oscillators. \\

In open air, the most basic vibrations are waves that move steadily in a particular direction and are therefore called \defnterm{travelling waves};
those waves are longitudinal. \\
A basic travelling sound wave consists of an alternating pattern of high-density regions, called \defnterm{crests}, and low-density regions, \defnterm{troughs}.
The highs of any wave is a crest and the lows of any wave is a trough. \\
The shortest distance between two adjacent crests (or troughs) is the \defnterm{wavelength}. \\

A standing wave's crests and troughs flip back and forth in place, crests becoming troughs and troughs becoming crests. \\
A travelling wave's crests and troughs move steadily in a particular direction at a particular speed. \\
The speed and direction of travel is the travelling wave's \defnterm{wave velocity}. \\
The wave speed can be calculated via the following
$$\text{wave speed} = \text{wavelength} \cdot \text{frequency}$$
\emph{Broad waves that vibrate quickly travel fast.} \\

The \defnterm{speed of sound} in air is about $331 m/s$ under standard conditions are sea level ($0^{\circ}C, 101,325~Pa$ pressure). \\

When a marching band steps quickly towards and away from the the listener, the listener hears its music shifted up or down in pitch.
Known as the \defnterm{Doppler effect}, this frequency shift occurs because the listener encounters sound wave crests at a rate that's different from the rate at which those crests were created. \\
If two are moving away from one another, the listener encounters the crests at a decreased rate and the pitch decreases.


%%%%%%%%%%%%%%%%%%%%%
%% P R E S S U R E %%
%%%%%%%%%%%%%%%%%%%%%

\newpage
\section{Pressure}
\subsection{Air and Air Pressure}
Air has weight and mass, and it has no fixed shape or size so it can occupy a wide range of \defnterm{volumes}. \\
Air is \defnterm{compressible}, so a certain mass of air can be squeezed into almost any space. \\
Air is a \defnterm{gas}, a substance consisting of tiny, individual particles that travel around independently, hence the flexibility of size and shape.
These particles are atoms and molecules.
An \defnterm{atom} is the smallest portion of an element that retains all the chemical characteristics of that element;
a \defnterm{molecule}, an assembly of at least two atoms, is the smallest portion of a chemical compound that retains all the characteristics of that compound. \\
A molecule's atom are held together by \defnterm{chemical bonds}, linkages formed by electromagnetic forces between the atoms. \\
Some gases are called \defnterm{inert gases} due to their chemical inactivity, like argon, neon, helium, krypton, and xenon. \\

Air's individual particles exhibit fast and energetic \defnterm{thermal motion};
thermal energy keeps them moving, spinning, and ricocheting off one another at bullet-like speeds of roughly $500m/s$. \\
Their frequent collisions prevent them from making much progress in any particular direction and, between collisions, they travel in nearly straight line paths because gravity doesn't have time to make them fall very far.
This vigorous thermal motion spreads the air particles part, so they don't accumulate on the ground. \\

The average force the air exerts on each unit of surface area is \defnterm{pressure}:
$$\text{pressure} = \frac{\text{force}}{\text{surface area}}$$
The SI unit of pressure is \emph{newton per meter\textsuperscript{2}}, also known as \defnterm{pascal} ($Pa$). \\
One pascal is a small pressure;
the air around you has a pressure of about $100,000~Pa$, so it exerts a force of about $100,000~N$ on a 1-m\textsuperscript{2} surface.

\subsection{Pressure, Density, and Temperature}
Air pressure is produced by bouncing air particles, so it depends on how often, and how hard, those particles hit a particular region of surface. \\
If air particles are packed more tightly, then the rate at which they hit a surface increases. \\
Thus, air's pressure is proportional to its \defnterm{density}, its mass per unit of volume:
$$\text{density} = \frac{\text{mass}}{\text{volume}}$$
The SI unit of density is \emph{kilogram per meter\textsuperscript{3}} ($kg/ms^3$). \\

The rate at which air particles hit a surface can also be increased by speeding the particles up. \\
The hotter the air, the more thermal energy it contains and the faster its particles move.
Thermal energy is a combination of \defnterm{internal kinetic energy} in the particles' random thermal motion and \defnterm{internal potential energy} stored as part of that random thermal motion. \\
Nearly all of air's thermal energy is internal kinetic energy since air particles are essentially independent (except during collisions). \\
Calm air's pressure is proportional to the average kinetic energy of its particles. \\
Even when air is moving as wind and has non-internal kinetic energy in its overall motion, its pressure is still proportional to the average \emph{internal kinetic} energy of its particles. \\

\defnterm{Temperature} measures the average internal kinetic energy per particles;
the hotter the air, the larger is the average kinetic energy per particle and the greater the air's pressure. \\
This is why the tire's air pressure increases on hot days and when it heats up during highway driving. \\

The common \defnterm{Celsius}($^{\circ}C$) and \defnterm{Fahrenheit}($^{\circ}F$) are not the most convenient scales for relating air's temperature to air's pressure. \\
Instead, it is the \defnterm{absolute temperature scales}, in which the zero of the temperature scale is \defnterm{absolute zero}, the temperature at which an object contains zero thermal energy. \\
At absolute zero ($-273.15~^{\circ}C$, air contains no internal kinetic energy and has no pressure.
In this scale, air's pressure is proportional to its temperature;
the SI scale of absolute temperature is the \defnterm{Kelvin} scale (K) and that 0 K is $-273.15~^{\circ}C$. \\

Air pressure is proportional to both the air's density and its absolute temperature:
$$\text{pressure} \propto \text{density} \cdot \text{absolute temperature}$$
This relationship does \emph{not} work when comparing pressures of two different gasses (i.e. different chemical compositions).

\subsection{Earth's Atmosphere}
Most of the mass of Earth's atmosphere is contained in a layer less than 6 km thick.
The atmosphere stays on Earth's surface due to gravity;
although the particles moves too fast for gravity to affect their motions significantly over the short term, gravity works slowly to keep them relatively near Earth's surface. \\
Occasionally, light and fast moving particles in the atmosphere---hydrogen and helium---escape Earth's gravity and drift off into interplanetary space. \\

While gravity pulls the atmosphere downward, air pressure pushes the atmosphere upward.
As the air particles try to fall to Earth's surface, their density increases and so does their pressure.
It's this air pressure that supports the atmosphere and prevents it from collapsing into a thin pile on the ground. \\

The atmosphere's density and pressure increase gradually with distance in the downward direction, the atmosphere has a downward \defnterm{density gradient} and a downward \defnterm{pressure gradient}. \\
The pressure of the surrounding air is always referred to as \defnterm{atmospheric pressure}.

\subsection{The Lifting Force on a Balloon: Buoyancy}
The air in Earth's atmosphere is a \defnterm{fluid}, a shapeless substance with mass and weight.
This  air has a pressure and exerts forces on the surfaces it touches;
that pressure is greatest near the ground and decreases with increasing altitude. \\
Air pressure and its variation with altitude allow air to lift hot-air and helium balloons though an effect known as \emph{buoyancy}. \\

An object partially or wholly immersed in a fluid is acted on by an upward \defnterm{buoyant force} equal to the weight of the fluid it displaced, this is known as \defnterm{Archimedes' Principle}. \\
Archimedes' principle applies to objects floating or submerged in any fluid, water, or oil. \\

Without gravity, the forces would cancel each other perfectly because of the pressure of a stationary fluid would be uniform throughout.
But gravity causes a stationary fluid's pressure to increase with distance in the downward direction;
i.e., the fluid has a downward pressure gradient. \\

When nothing is moving, the air pressure beneath an object is always greater than the air pressure above it.
Thus, the object experience an upward overall force from the air---a buoyant force. \\

Whether an object will float in a fluid can also be viewed in terms of density. \\
An object that has an average density greater than that of the surrounding fluid sinks, while one that has a lesser average density floats.

\subsection{Hot-Air Balloons}
A gas that has a lower density at atmospheric pressure is hot air. \\
Filling a balloon with hot air takes fewer particles than filling it with cold air, since each hot air particle is moving faster and contributes more to the overall pressure than does a cold-air particle.
A hot-air balloon contains fewer air particles, has less mass, and weighs less than it would if it contained cold air. \\
The balloon has an average density less than of the surrounding air, so the buoyant force it experiences is larger than its weight and thus it goes up. \\
Since the air pressure inside a hot-air balloon is the same as the air pressure outside the balloon, the air has no tendency to move in and or out and the balloon doesn't need to be sealed.
So there's a large propane burner heating up the air that fills the balloon from beneath the balloon's open end.
The heated air expands to fill more volume at the same pressure and some of it flows out of the balloon's open bottom. \\

The hotter the air, the lower its density and the less the balloon weighs. \\
If the pilot raises the air's temperature, particles leave the balloon, the balloon's weight decreases, and the balloon rises. \\
If the pilots let the air to cool, particles enter the envelope, the balloon's weight increases, and the balloon descends. \\

As the balloon ascended, the air becomes thinner and the pressure decreases both inside and outside the balloon.
Thus, the buoyant force decreases more rapidly than the balloon's weight decreases. \\
Eventually, the balloon reaches a \defnterm{flight ceiling} where the air becomes too thin to lift the balloon any higher. \\

A balloon's fabric envelope ages quickly at high temperatures, the balloon's air mustn't be heated above about $120~^{\circ}C$ ($248~^{\circ}F$).
Lower air temperatures prolong the life of the envelope;
the pilot tends to try to reduce the air temperature required by reducing the weight of the balloon's cargo.

\subsection{Helium Balloons}
The number of particles per unit of volume is the \defnterm{particle density}:
$$\text{particle density} = \frac{\text{particles}}{\text{volume}}$$

Hot air has a smaller particle density than cold air, since since they contain similar particles, hot air is lifted up by the buoyant force. \\

Air and helium have the same particle densities whenever their pressures and temperatures are equal, but helium atoms are much lighter than air particles.
A gas particle's contribution to the pressure does \emph{not} depend on its mass or weight, hence air and helium maintain the same densities. \\
At a particular temperature, each particle in a gas has the same average internal kinetic energy in its translational motion, \emph{regardless} of its mass. \\
Since helium weights less, the average helium atom moves much faster and bounces more often.
Thus, the lighter but faster-moving helium atoms are just as effective at creating pressure as heavier but slower-moving air particles. \\

If the helium atoms inside a balloon spreads out until the pressures and temperatures inside and outside the balloon are equal, the particle densities inside the outside the balloon will also be equal. \\
Since helium atoms are lighter than air, the balloon weights less and it will be lifted upward by the buoyant force. \\

The pressure of a gas has the following relation:
$$\text{pressure} \propto \text{particle density} \cdot \text{absolute temperature}$$
This property holds \emph{regardless} of the gas's chemical composition;
this is called the \defnterm{ideal gas law}.
This law relates pressure, particle density, and absolute temperature for a gas in which the particles are perfectly independent. \\
The constant of proportionality is the \defnterm{Boltzmann constant}, with a measured value of $1.381 \times 10^{-23}Pa\cdot m^{3}/(\text{particle}\cdot K)$. \\
With the constant, the ideal gas law has the following equation:
$$\text{pressure} = \text{Boltzmann constant} \cdot \text{particle density} \cdot \text{absolute temperature}$$
\emph{Don't incinerate a spray can. A hot dense gas tends to burst its container}. \\

Note that a balloon's lifting capacity is the different between the upward buoyant force it experiences and its downward weight.
So a gas, like Hydrogen, will not lift a balloon twice as fast since both balloons experience the same buoyant force. \\

Helium gas is obtained as a by-product of natural gas production from underground reservoirs in the United States, where it formed through the gradual radioactive decay of uranium and other unstable elements. \\
Once the underground stores are consumed, helium will become a relatively rare and expensive gas.

%%%%%%%%%%%%%%%%%%%%%%%%%%%
%% A I R   C L E A N E R %%
%%%%%%%%%%%%%%%%%%%%%%%%%%%
\newpage
\section{Air Cleaners}
\subsection{Electric Charge and Freshly Laundered Clothes}
Static electricity start with \defnterm{electric charge}, an intrinsic property of matter. \\
Electric charge is present in many of the \defnterm{subatomic particles} from which matter is constructed, and these particles incorporate their charges into nearly everything. \\
Since electric charge has a lot of influence on the objects that contain it, those objects are referred as \defnterm{electric charges} or \defnterm{charges}. \\

Charges exert forces on one another, which are the forces observed with static electricity. \\
There are two types of electrical charges, \emph{positive} and \emph{negative}. \\
Opposite charges \emph{attract} and similar charges \emph{repel}.
These forces between stationary electric charges are called \defnterm{electrostatic forces}. \\

Like momentum, angular momentum, and energy, electric charge is a conserved physical quantity---it cannot be created or destroyed, only transferred. \\
Positive charge and negative charge are not separate entities---they are just positive and negative amounts of the same physical quantity: electric charge. \\
Positive charges have positive amounts of electric charge, while negative charges have negative amounts. \\
The SI unit of electric charge is the \defnterm{coulomb} (C). \\

An atom consists of a dense central core or \defnterm{nucleus}, containing positively charged \defnterm{protons} and uncharged \defnterm{neutrons}, surrounded by a diffuse cloud of negatively charged \defnterm{electrons}. \\
The \defnterm{net electric charge} of an object is the sum of all its positive and negative amounts of charge. \\
An object is \defnterm{neutral} if it has zero net charge. \\

Electric charge is \defnterm{quantized}, that is, charge always appears in integer multiples of the \defnterm{elementary unit of electric charge} which is about $1.6 \times 10^{-19}C$ and is the magnitude of the charge found on most subatomic particle. \\
An electron has $-1$ elementary unit of charge, while a proton has $+1$ elementary unit of charge. \\
Electron is negatively charged, but it is also the primary constituents of electric currents in wires, hence we have deal with negative amount of charge flowing through wires.

\subsection{Coulomb's Law and Static Cling}
Forces between charges weaken with distance. \\
Forces between two electric charges are inversely proportional to the square of their separation. \\
Forces between electric charges are proportional to the amount of each charge. \\
Force between two charges can be represented as
$$\text{force} = \frac{\text{Coulomb constant} \cdot \text{charge}_{1} \cdot \text{charge}_{2}}{(\text{distance between charges})^{2}}$$
\emph{When there are enough like charges packed close together on your hair, it'll stand up.} \\
This relationship is \defnterm{Coulomb's law};
and the \defnterm{Coulomb constant} is about $8.988 \times 10^{9}N\cdot m^{2}/C^{2}$ and is one of the physical constants found in nature. \\

An \defnterm{electric polarization} is when an object remains neutral overall but has a positively charged region and a negatively charged region. \\
For example, a wall is neutral and a sock is negatively charged;
then the region closet to the sock is positively charged while the region farthest from the sock is negatively charged.

\subsection{Transferring Charge: Sliding Friction or Contact?}
When two different materials come in contact with one another, a few electrons normally shift from one surface to the other. \\
The transfer results from the chemical differences between the two touching surfaces and the associated change in an electron's potential energy when it shifts. \\
Some surfaces are ``hungrier'' for electrons than others, and whenever two dissimilar surfaces touch, the hungrier surface steals a few electrons from its ``less hungry'' partner.
This has to do with \defnterm{chemical potential energy}, energy stored in the chemical forces that bind together a material's constituent atoms and electrons. \\

To hold onto an electron, a surface maintains negative chemical energies such that it would take additional energy to free those electrons.
Some surfaces reduce the electron chemical potential energies more than others and thus bind their electrons more tightly.
If an electron on one surface can reduce its chemical potential energy by shifting to the other surface, it will accelerate toward that ``hungrier'' surface and eventually stick there. \\

As electrons accumulate on the lower energy surface, they begin to repel any electrons that try to follow and the transfer process soon comes to a halt.  \\
It stops when the electrons reach equilibrium---when the forward chemical force an electron experiences is exactly balanced by the backward electrostatic force. \\

It is possible for certain surfaces to exchange \defnterm{ions}, that is, electrically charged atoms, molecules, or small particles, along with electrons and acquire net charges as a result.

\subsection{Producing High Voltage}
Objects accumulate \defnterm{electrostatic potential energy};
it is present whenever opposite charges have been pulled apart or like charges have been pushed together. \\

\defnterm{Voltage} is electrostatic potential energy available per unit of electric charge at a given location:
$$\text{voltage} = \frac{\text{electrostatic potential energy}}{\text{charge}}$$
Voltage is difficult to conceptualize because you cannot see charge or sense its stored energy. \\
The SI unit of voltage is the joule per coulomb, commonly called \defnterm{volt} (V). \\
When the voltage is positive, positive charge can release electrostatic potential energy by escaping to a distant neutral place, where voltage is zero. \\
When the voltage is negative, negative needs energy to escape to a distance neutral place, where the voltage is zero.

\subsection{Accumulating Huge Static Charges}
Touching two different materials together causes a small transfer of charge from one surface to the other and that separating those oppositely charged surfaces produces elevated voltages and perhaps sparks. \\

A famous static machine is the Van de Graaff generator, a machine deliberately accumulate separated charge to produce extraordinary high voltages. \\
It uses a rubber belt to lift positive or negative charges onto a metal sphere until the magnitude of that sphere's voltage reaches hundreds of thousands or even millions of volts. \\
The belt is typically motor-driven to carry negative charges from its base to its spherical metal top.
Once inside the sphere, the belt's negative charges flow outward onto the sphere's surface, where they can be as far apart as possible.
There they remain until something releases them.

\subsection{Controlling Static Electricity}
The basic solution to static charge is mobility;
if charges can move freely, they'll eliminate static electricity all by themselves. \\
Opposite charges attract, so any separated positive and negative charges will join up as soon as they're allowed to move. \\

Materials such as metals that permit free charge movement are called \defnterm{electrical conductors}. \\
Those such as plastic, hair, and rubber that prevent free charge movement are called \defnterm{electrical insulators}.
Since charge movement eliminates static electricity, our troubles with static stem mostly from insulators. \\

The simplest way to reduce static electricity is to turn the insulators into conductors.
Even slight conductors, ones that just barely let charges move, will gradually get rid of any accumulation of separated charge. \\
Hence one of the main goals of fabric softeners, dryer sheets, and hair conditioners is to turn insulating materials into slight electrical conductors. \\
Those items use roughly the same chemical: a positively charged detergent molecule. \\

A detergent molecule is a long molecule that is electrically charged at one end and electrically neutral at the other end. \\
Its charged end clings electrostatically to opposite charges and is chemically ``at home'' in water. \\
Its neutral end is oil-like, slippery and ``at home'' in oils and greases. \\
This duality makes detergents very good at cleaning. \\

Cleaning agents shouldn't cling to the materials they're cleaning, so it's important that the two do not have opposite charges. \\
Fabrics and hair generally become negatively charged when wet, so negatively charged detergent molecules clean much better than positively charged ones.

%%%%%%%%%%%%%%%%%%%%%%%%%%%%%%%
%% X E R O X   M A C H I N E %%
%%%%%%%%%%%%%%%%%%%%%%%%%%%%%%%
\newpage
\section{Xerox Machines}
\subsection{Xerography: Using LIght to Print Copies}
The image that a xerographic copier prints on a sheet of paper begins as a pattern of tiny black particles or \defnterm{toner} on a smooth, light-sensitive surface.
The copier uses static electricity and light reflected from the original document to arrange this toner on the surface and then carefully transfers the toner to the paper. \\

At the heart of the xerographic copier is a thin, light-sensitive surface made from a \defnterm{photoconductor}, a normally insulating material that becomes a conductor while exposed to light. \\
Light from the original document is allowed to determine the pattern of static electricity on the photoconducting surface and consequently the placement of toner on the piece of paper. \\

Each coping cycle begins in the dark with the copier spraying negative charges onto its photoconductor.
On the other side of the photoconductor is a grounded metal surface---grounded in the sense that it's electrically connected to Earth so that charges are free to flow between the two. \\
As negative charges land on the open surface of the photoconductor, they attract positive charges onto the metal surface beneath it. \\
When the charge-spraying process is complete, the open surface of the photoconductor is uniformly coated with negative charges while the underlying metal surface is uniformly coated with positive charges. \\
After, the copier uses a lens to cast a sharp image of the original document onto the photoconducting surface;
the light hits the photoconductor only in places corresponding to the white parts of the original document. \\
When exposing the photoconductor to light, charges move through any regions of the photoconductor that are exposed to light, leaving those regions electrically neutral.
The result is a \defnterm{charge image}, a pattern of electric charge on the photoconductor's surface that exactly matches the pattern of ink on the original document. \\

To develop the charge image into a visible one, the copier exposes the photoconductor to positively charged toner particles.
This toner is a fine, insulating plastic powder containing a coloured pigment (usually black). \\

The photoconductor now carries a black image of the original document, an image that the copier must transfer to the paper. \\
Before the transfer, the copier first weakens the photoconductor's grip on the toner by exposing it to light from a charge erase lamp.
This light eliminates the photoconductor's charge and leaves the positively charged toner particles clinging only loosely to its surface. \\
The copier then transfers the toner image to a blank sheet of paper by pressing that paper (lightly) against the photoconductor while spraying negative charge onto the paper's back. \\
The positively charged toner is attracted to the negatively charged paper, permanently fusing the toner onto the paper. \\
Once the image has been transferred to the paper, the copier cleans it photoconducting surface in preparation for the next copy; a second charge erase lamp eliminates any remaining charge, and a brush or squeegee mops up any residual toner.

\subsection{Discharges and Electric Fields}
At the start of the copy cycle, the xerographic copier coats its photoconducting surface uniformly with electric charges.
Because this process is done in the dark, while the surface is an electrical insulator, the charges must be sprayed onto it like paint. \\
The copier's sprayer is a \defnterm{corona discharge}, a gentle sustained spark that forms in the air near a needle or fine wire that's kept at high voltage. \\

A corona \emph{discharge} is a type of \defnterm{discharge}, a flow of electric charge through a gas. \\
Air is normally an insulator because its atoms and molecules are neutral and can't convey charge from one place to another.
However, by seeding air liberally with individual charged particles, the copier manages to turn that air into a conductor and then to produce a discharge in it. \\

Free charges are hard to come by in the air, the copier begins with just a few charged particles and uses them to generate more. \\
The copier uses electrostatic forces to accelerate those initial charges to enormous speeds and lets them smash into air's neutral particles.
When hit hard enough, a neutral air particle breaks into oppositely charged fragments and thus adds two more free charges to the air. \\
A cascade of collisions ensues, and the air ``breaks down,'' transforming from an insulator to a conductor. \\
The copier then uses this conducting air to spray the photoconductor with charges. \\

To parlay the initial charges into the vast numbers it needs, the copier must accelerate them aggressively.
The neutral air particles are so densely packed that it's difficult for the charged ones to pick up much speed before they hit something and slow down.
To give each initial charge a good shot at breaking the first neutral particle it hits, the copier must accelerate that charge very quickly. \\
The copier accelerates its charge using strong electrostatic forces. \\
Suppose we view the electrostatic force on our charge as a result of its one interaction with something local: an \defnterm{electric field}, an attribute of space that exerts an electrostatic force on a charge.
The surrounding charges create the electric field and that field pushes on our charge. \\
The electrostatic force on our charge depends on the charge's location in space and time, so the value of the electric field also depends on space and time. \\

The electric field is a \defnterm{field}, a structure that associates a physical quantity with each point in space and time. \\
The electric field is actually a \defnterm{vector field}, a structure that associates a \emph{vector} quantity with each point in space and time.
At each point in the electric field, its magnitude is the amount of electrostatic force it exerts per unit of electric charge and its direction is the direction in which it pushes a positive charge. \\

So, the electrostatic force on a charge is exerted by the electric field itself, not by the source of the electric field.
We can write the relationship as the following:
$$\text{electrostatic force} = \text{charge} \cdot \text{electric field}$$
\emph{Charged lint accelerates quickly in a region full of static electricity,} where the electrostatic force is in the direction of the electric field. \\
Note that a particle carrying a negative amount of charge (an electron) experiences a force opposite the electric field. \\
The SI unit of electric field is \emph{newton per coulomb} ($N/C$). \\

The Xerographic copier employs a very strong electric field to ``break down'' the air so that it can operate its discharge.
That field accelerates charges so rapidly that collision cascades occur and fill the air with free charges.

\subsection{Conductors and Voltage Gradients}
A copier's precharging system uses the gentle corona discharge that develops in the strong electric field just outside a fine high-voltage wire.
This discharge ferries charges to the photoconductor surface and coats it uniformly. \\

\textbf{Voltage and charge on a conduction object}:
\emph{With its charges at equilibrium, a homogeneous conducting object has a single uniform voltage and the net charge anywhere in its interior is zero.} \\

\textbf{Electric field in a conducting object}:
\emph{With its charges at equilibrium, a homogeneous conducting object has zero electric field in its interior}. \\

Although the voltage is uniform \emph{in} and \emph{on} the copier's fine conducting wire, it varies rapidly with position \emph{outside} that wire. \\
Accompanying this large spatial variation in voltage is a strong electric field, called a \defnterm{voltage gradient}.
A spatial variation in voltage can be thought as a ``slope'' in the voltage;
positive charge accelerates swiftly down a large voltage gradient toward a lower voltage. \\

Both electric fields and voltage gradients cause charges to accelerate, so voltage gradient is actually an electric field:
$$\text{electric field} = \text{voltage gradient} = \frac{\text{voltage drop}}{\text{distance}}$$
\emph{Charges rush down steep drops in voltage, much as bicycles rush down steep drops in hill height}, where the electric field points in the direction of the most rapid voltage decrease. \\
The SI unit of electric field is also \emph{volt per meter} ($V/m$).

\subsection{Fine Wires and High Voltages: Corona Discharges}
Suppose on a dry winter day, you coat yourself with positive charges and raise your voltage to $30,000V$ by scuffing your rubber-soled shoes across an acrylic carpet. \\
As you then approach a grounded door knob at $0V$, the voltage different between the doorknob and your hand will be $30,000V$.
When your hand is about 1cm from the doorknob, the electric field will reach $30,000V$ per centimeter and the air will break down with a brilliant spark. \\
Because your hand and the doorknob are similar in size and shape, the voltage changes smoothly between them.
It varies steadily from $0V$ on the doorknob to $30,000V$ on your hand, so the voltage graident or electric field is nearly \emph{uniform}. \\

However, when two objects differ significant in size, the \emph{larger} object dominates voltages in the space between them! \\
Suppose you hold a long pin in your hand as you approach the doorknob, the doorknob will control the voltage most of the way to the pin and nearly all the increase in voltage will occur just outside the pin's point.
Rather than being uniform, the voltage gradient or electric field will be strongest near that point. \\

The copier makes good use of a nonuniform electric field. \\
Its fine high-voltage wire is nearly surrounded by a much larger metal shroud.
The wire is so thin that its influence fades just a hair's breadth from its surface and the grounded shroud dominates voltage almost all the way to the wire. \\

The discharge that forms near the fine wire is a special, self-regulating one---a \defnterm{corona discharge}. \\
While most discharges can't control how many free charges they produce, a corona discharge automatically maintains a steady production.
Because free charges form only in the strong electric field near its thin conductor, their production rate is very sensitive to charges in that conductor's effective thickness.
If there are too many free charges in the air near the conductor, their ability to conduct electricity effectively thickens the conductor, weakens the electric filed, and slows the production of free charges. \\
The discharge is correcting its own mistake. \\

Due to this stabilizing effect, the air in a corona discharge maintains a steady electrical conductivity that's ideal for charging a copier's photoconductor.

\subsection{Getting Ready to Copy: Charging by Induction}
A corona discharge does more than just turn air into a conductor;
it also produces an outward spray of electric charges.
Those charges are pushed outward by the electric field surrounding the corona wire. \\
Since the copier's corona wire has a negative voltage, the surrounding electric field points toward that wire.
Because negative charges accelerate opposite an electric field, the copier's corona produces a shower of outgoing negative charges.
They spray onto the photoconducting surface as it moves steadily past the corona, and the photoconductor thus acquires a uniform coating of negative charges. \\

As each negative charge lands, it draws a positive charge onto the grounded metal surface beneath the photoconductor and the attraction between those two opposite charges holds them firmly in place.
While the photoconductor's open surface is acquiring its uniform negative charge, the metal layer underneath is acquiring an equivalent positive charge. \\
This process, whereby a grounded conductor acquires a charge through the attraction of nearby opposite charge, is called ``charging by induction.'' \\

The induced positive charge on the metal side of the photoconductor is important to the xerographic process for several reasons:
\begin{enumerate}
  \item[1.] it lowers the electrostatic potential energy of the negative charge so that the surface's negative voltage isn't as enormous
  \item[2.] without that positive layer nearby, repulsion between like charges would tend to push negative charges on the open surface toward the edges of the photoconductor and distort the resulting images
  \item[3.] the positive charge layer gives the negative charge layer somewhere to go when the photoconductor is exposed to light. \\
  Whenever light from the original document turns a patch of the photoconductor into a conductor, the negative and positive charge layers rush together and cancel. \\
  The resulting uncharged portion of the photoconductor subsequently attracts no toner and produces a white patch on the finished copy
\end{enumerate}

So the copier uses a corona discharge to coat a photoconducting surface with negative charge and then selectively erase portions of that charge layer with light from the original document. \\
The remaining charged patches on the photoconductor attract positively charged black toner, which is then transferred permanently to the paper.

\subsection{Capacitors}
When it's in the dark, the copier's photoconductor system is an example of a \defnterm{capacitor}, a device that stores separated positive and negative charge. \\
Most capacitors consist of two oppositely charged surfaces separated by a thin insulating layer.
To make adding or removing charge easy, those surfaces are usually made of metals or other electrical conductors. \\
However, the copier's capacitor has only one metal surface on its insulating photoconductor layer;
the other surface is open and nonconducting.
The copier uses its corona discharge to put charge on that open surface, and it uses light to remove that charge! \\

The two conducting surfaces of an ordinary capacitor are called \emph{plates}.
When one plate is positively charged and the other is negatively charged, the attraction of opposite charges on those plates offsets the repulsion of like charges on each plate (hence the capacitor remains neutral). \\

A capacitor can be ``charged'' by transferring (positive) charge from its negative plate to its positive one.
Since each charge you move experiences an electrostatic force in the opposite direction, you must do work on that charge as you push it from the negative plate to the positive plate.
This work is stored in the capacitor as electrostatic potential energy, and that stored energy is released when you let the separated charge get back together. \\
Since (positive) charge has more electrostatic energy on the positive plate than on the negative plate, the voltage of the positive plate is higher than the voltage on the negative one.
This difference of voltage between plates is proportional to the separated charge on them;
the more separated charge the capacitor is holding, the larger the voltage difference. \\

The voltage different also depends on the physical structure of the capacitor. \\
\emph{Increasing} the surface area of each plate \emph{decreases} the repulsion of its like charges. \\
\emph{Thinning} the insulating layer between the plates \emph{increases} the attraction of their opposite charges. \\
Both changes lower the separated charge's electrostatic potential energy and, thus, the voltage different difference between the plates.
The bigger and closer the plates, the less energy it takes to store separated charge on them. \\

Changes that allow the capacitor to store separated charge more easily increase its \defnterm{capacitance}, the amount of separated charge the capacitor holds divided by the voltage difference between its plates. \\
The SI units for capacitance is \emph{coulomb per volt}, also called the \emph{farad} ($F$). \\
A capacitor with a farad of capacitance stores an incredible amount of separated charge, even at a low voltage difference, but a capacitor with a billionth of a farad of capacitance is much more common. \\

If a capacitor has separated charge, it will have a voltage difference between its two plates.

%%%%%%%%%%%%%%%%%%%%%%%%%%%
%% E L E C T R I C I T Y %%
%%%%%%%%%%%%%%%%%%%%%%%%%%%
\newpage
\section{Electricity}
\subsection{Electricity and the Flashlight's Electric Circuit}
A basic flashlight has just three components---a battery,  a light-bulb, and a switch.
When the switch is on, the strips transfer energy from the batteries to the bulb. ]]

When the flashlight is turned on, electricity conveys energy from the batteries to the bulb. \\
An \defnterm{electric current}, a current of electric charges, flows through these components, carrying the energy with it. \\
As long as the flashlight is turned on, charges flow around these components in a loop;
receiving energy from the batteries and delivering it to the bulb, over and over again.
On route, the charges carry this energy mostly as electrostatic potential energy. \\
The looping path taken by charges is called an \defnterm{electric circuit}. \\

A circuit has no beginning or end, so charges cannot accumulate in one place, where their mutual repulsion would eventually stop them from flowing. \\
Circuits explain the need at least two wires in the power cord of any home appliance: one wire carries charges to the appliance to deliver energy, and the other wire carries those charges back to the power company to receive some more. \\

As part of one conducting path between the batteries and the bulb, the switch can make or break the flashlight's circuit. \\
When the flashlight is on, the switch completes the loop so that charges can flow continuously around the \defnterm{closed circuit}. \\

When the flashlight is turned off, the switch breaks the loop to form an \defnterm{open circuit}. \\
The loop now has a gap in it (at the switch) and the circuit can no longer carry a continuous current. \\

Another type of circuit is a \defnterm{short circuit}, it forms when the two separate paths connecting the batteries to the bulb accidentally touch one another. \\
This unintended contact creates a new, shorter loop around which the charges can flow. \\
Because the bulb is expected to extract energy from the charges, it's designed to impede their  ow and to convert their electrostatic potential energy into thermal energy and light. \\
This opposition to the flow of electricity is called \defnterm{electrical resistance}. \\
Since the shortened loop offers little resistance, most of the charges flow through it, bypassing the bulb. \\
Since the bulb is the only part of the flashlight that's designed to consume electric energy, a short circuit leaves the charges without a safe place to get rid of their electrostatic potential energy.
They deposit it dangerously in the batteries and the metal paths, making them hot.

\subsection{The Electric Current in the Flashlight}
Each of the tiny charged particles flowing through the flashlight's circuit carries with it just a single elementary unit of electric charge and a minuscule amount of electrostatic potential energy. \\
however, those charges flow in astonishing numbers, so they convey a considerable amount of energy per second---the quantity we know as power and measured in \emph{watts} ($W$). \\

There are too many elementary charges to count, so it's much better to measure the circuit's \defnterm{current}, that is, the amount of charge passing a particular point in the circuit per unit of time. \\
The SI unit of current is \emph{coulomb per second} or more commonly known as \emph{ampere} ($A$);
one ampere is 1 C per second. \\
We can determine how much power is reaching the bulb by multiplying the current by its electrostatic energy per coulomb---voltage. \\

Fictitiously, charge flows around the circuit clockwise, from the battery chain's positive terminal, through the bulb's filament through the switch, and into the battery chain's negative terminal. \\
In reality, the electric current is carried by \emph{negatively} charged electrons heading in the opposite (counterclockwise) direction;
so electrons come out from the negative terminal and goes back in at the positive terminal of the battery chain. \\

This doesn't affect the wire as charges pass through actually. \\
If negative charge flow in one direction, the side of its direction becomes negatively charged after while the other end is positively charged. \\
If positive charge flow in the opposite direction, the side of its direction becomes positively charged after while the other end is negatively charged.

\subsection{Batteries}
Two ways to think about a battery:
\begin{itemize}
  \item[1.] a battery is a type of pump;
  it ``pumps'' charge from low voltage to high voltage. \\
  A battery increases a charge's electrostatic potential energy by pushing it up a voltage gradient.

  \item[2.] a battery is a chemically powered machine;
  it uses chemical forces to transfer charges from its negatively terminal to its positive terminal. \\
  As positive charge accumulate on the battery's positive terminal, the voltage there rises, and as negative charges accumulate on the battery's negative terminal, the voltage there drop. \\
  Since battery does work transferring charges from low voltage to high voltage, it is converting its chemical potential energy into electrostatic potential energy in these separated charges.
\end{itemize}

A battery's rated voltage reflects its chemistry, specifically the amount of chemical potential energy it has available for each charge transfer. \\
As the voltage difference between its terminals increases, so does the energy required for each charge transfer. \\
Eventually, the electrostatic forces opposing the next charge transfer exactly balance the chemical forces promoting it and the battery is then in \emph{equilibrium}. \\

As the battery consumes its chemical potential energy, its ability to pump charges diminishes;
i.e. the battery's increasing disorder reduces it voltage. \\

The more batteries in a chain, the more energy a charge receives overall and the more the voltage will increase from the chain's negative terminal to its positive terminal. \\
If a battery is reversed in a chain, the reversed battery will extract energy from any charge passing through it. \\
While the chain may still pump charge from its negative terminal to its positive terminal, the overall voltage will be reduced. \\

As the reversed battery extract energy from the charges passing through it, some of that extracted energy is \emph{converted into} chemical potential energy.
Thus the reversed battery is recharging. \\
Battery chargers push current backward through a battery to restore chemical potential energy in a rechargeable battery. \\
Note: nonchargeable batteries may overhead and explode during recharging!

\subsection{Bulbs and Metal Strips}
A battery gives charges electrostatic potential energy by pushing them \emph{up} a voltage gradient, a bulb releases that electrostatic potential energy by letting charges slide \emph{down} another voltage gradient. \\

While a conductor has a uniform voltage when its charges are in \emph{equilibrium}, the charges in the bulb are in equilibrium only when the flashlight is turned off. \\
When it is turned on, there's a voltage difference between the two ends of the filament and the filament's charges immediately begin accelerating down the voltage gradient towards the lower voltage end. \\

If the filament were a prefect conductor of electricity, each charge would accelerate steadily down the voltage gradient and convert its electrostatic potential energy into kinetic energy. \\
However, the filament has a large electrical resistance (it impedes the flow of current). \\
Each charge bounces its way down the voltage gradient, colliding frequently with the filament's tungsten atoms and giving up kinetic energy with each collision. \\

Each metal strip in the flashlight has a small electrical resistance.
So, the charges need a small voltage gradient to keep them moving forward, and they emerge from the strip at a slightly lower voltage than where they entered.
The lost energy becomes thermal energy, slightly heating the metal strip.

\subsection{Voltage, Current, and Power in Flashlights}
A bulb consumes power because the current passing through it slides down the (voltage) gradient and hence a drop in voltage. \\
This \defnterm{voltage drop} measures the electrostatic potential energy each unit of charge loses, calculated by
$$\textbf{power consumed} = \text{voltage drop} \cdot \text{current}$$
\emph{A large electric current dropping from high voltage to low voltage is like the torrent of water dropping from high to lower over Niagara Falls: both release a a lot of power}. \\

A battery chain produces electric power because current passing through it is pushed up a voltage gradient and thus a rise in voltage. \\
This \defnterm{voltage rise} measure the electrostatic potential energy each unit of charge gains, calculated by
$$\text{power provided} = \text{voltage rise} \cdot \text{current}$$
\emph{Raising a large current from low voltage to high voltage is like pumping a huge stream of water from low to high to fight a fire at the top of a skyscraper: both require a lot of power}.

\subsection{Choosing the Bulb: Ohm's Law}
The relationship between current and voltage drop is the result of collisions. \\
Charges effectively stop each time they crash into metal atoms, so they need the push of an electric field to keep them moving forward.
Doubling that electric field doubles each charge's average speed and, because the number of mobile charges in the filament is fixed, also doubles the overall current flowing through the filament. \\
Since the electric field that propels this current is the filament's voltage gradient, doubling the voltage drop through filament doubles the current as well. \\

\defnterm{Ohm's law} is the observation that current is proportional to the voltage drop and inversely proportional to electrical resistance:
$$\text{current} = \frac{\text{voltage drop}}{\text{electrical resistance}}$$
$$\text{voltage drop} = \text{current} \cdot \text{electrical resistance}$$
\emph{Long, skinny jumper cables have large resistances. When you connect them to a battery to jumpstart your car, they'll carry a relatively small current and your current probably won't start}. \\
The SI unit of electrical resistance is \emph{ohm} ($\Omega$). \\

Ohm's law applies to nearly everything that is \defnterm{ohmic}---can be characterized by an electrical resistance. \\
Ohm's law are common in modern technology, often as simple electronic components known as resistors.
A \defnterm{resistor} carries a current equal to its voltage drop divided by its resistance, and it experiences a voltage drop equal to the current it carries times its resistance. \\

An object's electrical resistance is usually dependent on temperature. \\
So an object's resistance increases with temperature, because rising temperature increases the collision frequency.

\subsection{LED Flashlights, Series and Parallel Circuits}
LEDs are more energy efficient than bulbs and last much longer (almost forever). \\
LEDs are semiconductors that are \emph{nonohmic} because they don't obey Ohm's law and cannot be characterized by resistance. \\
An LED is often paired with a resistor because small changes in voltage drop dramatically change the current passing through it.
The resistor conducts enough amount of current to power the LED properly. \\

Current flows sequentially through a resistor and an LED before returning to the battery;
this is a \defnterm{series circuit} since the same current flows sequentially through each component. \\
Since the resistor and LED are in series, they must share that overall voltage drop. \\
They also carry the same current, the resistor's ohmic behaviour limits the current passing through the entire circuit. \\
Each component takes enough of the overall \emph{voltage} (drop) to allow it to conduct the \emph{same} current as the other component. \\

If there are multiple LEDs, the current is divided from the batteries into parts and sending one part through each of the LEDs. \\
The arrangement where the \emph{current} is divided into parts that flow \emph{simultaneously} through each component is known as a \defnterm{parallel circuit}. \\
Even though each component carries a fraction of the overall current, all the components experience the same voltage!

\subsection{Button-Shaped Refrigerator Magnets}
Two types of \defnterm{magnetic poles} that exert \defnterm{magnetostatic forces} on one another.
The word \emph{pole} is to  distinguish magnetism from electricity;
\defnterm{poles} are magnetic, charges are electric. \\
The two poles are \emph{north} and \emph{south}, and are opposite of one another. \\
Both poles carry just one physical quantity: \defnterm{magnetic pole}.
North poles carry \emph{positive} amount of magnetic pole, while south poles carry \emph{negative} amount. \\
Like poles repel each other and opposite poles attract. \\
The magnetostatic forces between two poles grow weaker as they move apart and are inversely proportional to the square of the distance between them. \\

Subatomic particles that carry pure positive or negative electric charges are common.
However, particles that carry pure north or south magnetic poles have never been found, called \defnterm{magnetic monopoles}.
Without monopoles, there are no magnetic sparks. \\

While isolated magnetic poles can't be found, we find pairs of poles that consist of \emph{equal} north and south poles, spatially separated from one another in an arrangement called a \defnterm{magnetic dipole}.
The dipole has zero \defnterm{net magnetic pole}. \\

\defnterm{Coulombs's Law for Magnetism}: \\
The magnitudes of the magnetostatic forces between two magnetic poles are equal to the permeability of free space times the produce of the two poles divided by $4\pi$ times the square of the distance separating them. \\
If the poles are like, then the forces are repulsive. \\
If the poles are opposite, then the forces are attractive.
$$\text{force} = \frac{\text{permeability of free space} \cdot \text{pole}_{1} \cdot \text{pole}_{2}}{4\pi \cdot (\text{distance between poles})^{2}}$$
\emph{Don't hold two strong magnetic poles near one another unless you're prepared to be pushed around hard as they attract or repel each other}. \\
The \defnterm{permeability of free space} is $4\pi \times 10^{-7} N/A^{2}$. \\
The magnetic forces follow Newton's third law.

\subsection{The Refrigerator: Iron and Steel}
The steel on the refrigerator is composed of microscopic magnets, each with a matched north pole and soul pole.
Those magnets are normally oriented semi-randomly so the refrigerator exhibits no overall magnetism. \\
When a magnet comes near the fridge, its tiny magnets evolve in size, shape, and orientation. \\
Opposite poles shift close to the magnet's pole and like poles shift farther from the magnet's pole. \\
The steel develops a \defnterm{magnetic polarization} and consequently attracts the pole of the button magnet. \\

The reason why steel can develop strong magnetic polarization is because it's a \defnterm{ferromagnetic} material---it is actively and unavoidably magnetic on the size scale of atoms. \\

Subatomic particles---electrons, protons, and neutrons---have magnetic dipoles, particularly the electrons, and the atoms they form often display magnetism. \\
While most atoms are intrinsically magnetic, most materials are not.
Because another round of pairing and cancelling occurs when atoms assemble into materials. \\
Few materials avoid this cancellation and remain magnetic at the atomic scale. \\
The most important of theses are ferromagnets, a class of magnetic materials that include ordinary steel and iron. \\
When examined, it is composed of many microscopic regions, or \defnterm{magnetic domains}, that are naturally magnetic and cannot be demagnetized. \\
Within a single domain, all the atomic-scale magnetic dipoles are aligned and together they give the overall domain a substantial net magnetic dipole. \\

In magnetic domains, magnetic interaction orient nearby domains so that their magnetic dipoles oppose one another and cancel.
Those magnets balance one another so well so that the overall steel appears nonmagnetic. \\

When a strong magnetic pole comes near the steel, the steel undergoes magnetization and becomes \defnterm{magnetized}. \\
Note that the atoms don't move;
the change is purely a reorientation of the atomic-scale magnetic dipoles.

\subsection{Plastic Sheet Magnets and Credit Cards}
When a magnet is removed from the fridge, it undergoes demagnetization and becomes (almost) \defnterm{demagnetized}. \\
Chemical forces can make it difficult for domains to grow or shrink and the domains get stuck during the process. \\
Adjacent domains are separated by \defnterm{domain walls}, boundary surfaces between one direction of magnetic orientation and another.
Those domain walls must move if the domains are to change size, but impurities and flaws can interact with a domain wall to keep it from moving. \\
Hence, the steel fails to demagnetize itself completely (can be removed typically with heat or mechanical shock). \\

A \defnterm{soft magnetic material} is one that demagnetize itself easily when all nearby poles are removed, e.g. chemically pure iron. \\
A \defnterm{hard magnetic material} is one that does not demagnetize itself easily and that tends to retain whatever domain structure is imposed on it by its most recent exposure to strong nearby poles. \\
A \defnterm{permanent magnet} retains its present magnetization almost indefinitely. \\

A hard magnetic material's ability to ``remember'' its magnetization can be useful for saving information. \\
Once magnetized in a particular manner so as to represent a piece of information, the material will retain its magnetization and the associated information until something magnetizes it differently.

\subsection{Compass}
Earth itself has a magnetic dipole and that dipole affects the orientation of a compass needle. \\
The Earth's \emph{north} geographic pole is a \emph{south} magnetic pole.
Attraction from that south magnetic pole is what draws the compass' north magnetic pole toward the north. \\

The full story of magnetic interaction between the Earth's poles and the compass, suppose the needle interacts with a \defnterm{magnetic field}, an attribute of space that exerts a magnetostatic force on a pole. \\
So the needle responds to the local magnetic field, a field that's created by all the surrounding magnetic poles. \\
Note that A magnetic field can be created by things other than pole. \\
The magnetic field at a given location measures the magnetostatic force that a unit of pure north pole would experience if it were placed at that point
$$\text{magnetostatic force} = \text{force} \cdot \text{magnetic field}$$
\emph{If you place a strong magnet in a strong magnetic field, expect to be pushed around}, where the magnetosatic force is in the direction of the magnetic field. \\

A compass needle aligns with the local magnetic field and Earth's magnetic field is uniform in the vicinity of the compass which balances the northward push and the southward push. \\
When the needle is aligned with a nonuniform field---its north pole pointing in the same direction as the local field---the forces on its two opposite poles won't balance and it will experience a net force in the direction of increasing field. \\
If aligned against, it will point in the direction of the decreasing field.

\subsection{Iron Fillings and magnetic Flux Lines}
With iron fillings sprinkle into the magnetic filed, the iron particles magnetize along the local field and then stick together, north pole to south pole, in long strands that delineate the magnetic field. \\
The strands follow along the local magnetic field direction and have a density proportional to that local field, these lines are called \defnterm{magnetic flux lines}. \\

A magnetic field tends to point away from the north poles and toward south poles. \\
In permanent magnets, every magnetic flux line begins at a north pole and ends at a south pole. \\
Also, flux lines never start at or end on anything other than a magnetic pole.
So if you follow a magnetic flux line with a compass, you either reach a south pole or walk forever!

\subsection{Electric Doorbells and Electromagnets}
When you press the doorbell button, current flows through a coil of wire and the resulting magnetic field yanks an iron rod into the center of that coil. \\
About the time the rod reaches the center, part of it hits the first chime. \\
When you then open the switch, stopping the current and its magnetism, a spring pushes the iron rod back out of the coil and it hits the second chime. \\

So, electric current can \emph{produce} magnetic forces! \\
Moreover, electric currents \emph{are} magnetic! \\

\textbf{First Connection Between Electricity and Magnetism}: \\
Moving electric charge produces a magnetic field. \\

Suppose we use iron powder to disclose the magnetic flux lines surrounding a lone, straight, current-carrying wire. \\
Those flux lines circle the wire like concentric rings, growing more widely separated as the distance from the wire increases. \\
The wire is an \defnterm{electromagnet}, a device that becomes magnetic when it carries an electric current (the field is strongest near the surface of the wire). \\
Since an electromagnet has no true magnetic poles, the flux lines can't stretch from north pole to south pole.
Instead, each flux line of an electromagnet is a closed loop. \\

The magnetic field around a current-carrying wire is fairly weak, however, and a practical doorbell winds that wire into a coil to concentrate and strengthen its field. \\
The field produced by a two-pole (button) magnet is almost identical to the magnetic field of the coil.
It's as though the coil has a north pole at one end and a south pole at the other. \\
Since there are no true poles, the flux lines don't end anywhere.
Instead, they continue right through the middle of the coil and form complete loops. \\

While current is flowing through the coil and the iron rod is inside it, the two objects act as a single powerful electromagnet. \\
The magnetic field surrounding the pair is the sum of the coil's modest magnetic field and the magnetized iron's much strong field. \\
In effect, the current in the coil magnetizes the iron and the iron then creates most of the surrounding magnetic field.

\subsection{Direct Current Power Distribution}
Batteries have limited uses and are expensive for larger power consumption. \\
A most cost-effective source for electricity was coal- or oil-powered electric generators back in the days. \\

\defnterm{Direction current} (DC) is current that always flow in one direction and the voltage difference is constant \\

To deliver electricity to places/things farther away, thicker wires were required to allow them to carry current more easily (wires have resistance). \\
We want to minimize the amount of power that are wasted as thermal energy (from collisions), this can be determined with Ohm's law:
\begin{align*}
\text{power consumed} &= \text{voltage drop} \cdot \text{current} \\
&= (\text{current} \cdot \text{electrical resistance}) \cdot \text{current} \\
&= \text{current}^{2} \cdot \text{electrical resistance}
\end{align*}
So delivering more current actually increases the wasted power! \\

We can also decrease current while increasing voltage difference.
So while less current flowed, the larger voltage drop left the delivered power unchanged. \\

Materials that lose their electrical resistance at extremely low temperatures and become perfect electrical conductors, they are \defnterm{superconductors}.
But these superconductors are too impractical for power distribution systems.

\subsection{Introducing Alternating Current}
A problem with distributing electrical power via direction current is that there is no easy way to transfer from one DC circuit to another.
DC power distribution therefore wastes a lot of power in the wires connecting everything together since safety requires that the entire circuit use low voltages and large currents. \\

An \defnterm{alternating current} (AC) is a current that periodically reverses direction---it alternates. \\
AC makes it easy to transfer power from one AC circuit to another so that different paths of the AC power-distribution system can operate at different voltages with different currents. \\
Also, the wires that carry the power long distance are part of a high-voltage, low-current circuit and therefore waste little power. \\

This alternating current is caused by alternating voltage drop (like a wave), a voltage drop that periodically reverses direction. \\

In the states, an AC outlet offers three connections: \emph{hot, neutral, \emph{and} ground}. \\
The absolute voltage of \emph{neutral} remains near 0 V, while the absolute voltage of \emph{hot} alternates above and below 0 V.
\emph{Ground}, which is an optional safety connection, also remains near 0 V absolute. \\
So current flows from \emph{hot} to \emph{neutral} when \emph{hot} has a positive voltage.
And from \emph{neutral} to \emph{hot} when \emph{hot} has a negative voltage. \\

The \emph{hot} voltage varies sinusoidally---the trig sine function with respect to time.
In the states, AC voltage reverse every 120th of a second, yielding 60 full cycles of reversal (back and forth) each second -> 60 Hz. \\
In Europe, the reversals occur every 100th of a second, so AC voltage completes 50 full cycles of reversal each second -> 50 Hz. \\

Power consumption in simple ohmic devices is used to define an effective voltage for AC electric power.
An outlet's nominal AC voltage---technically, its \defnterm{root mean square voltage} (RMS)---is defined to be equal to the DC voltage that would cause the same average power consumption in an ohmic device. \\
So, 120-V AC power delivers the same average power as 120-V DC power. \\

The reversal of AC powers aren't without consequence:
\begin{enumerate}
  \item[1.] some electrical and most electronic devices are sensitive to the direction of current flow and must handle the reversals carefully
  \item[2.] the power available from an ordinary AC outlet rises and falls with each voltage reversal and is momentarily zero at the reversal itself
\end{enumerate}

\subsection{Magnetic Induction}
An advantage in alternating current was that its power could be transformed---it could be passed via electromagnetic action from one circuit to another by a transformer. \\
A \defnterm{transformer} is a deice that uses two important connections between electricity and magnetism to convey power from one AC circuit to another:
\begin{enumerate}
  \item[1.] moving electric charge creates magnetic fields
  \item[2.] magnetic fields that change with time create electric fields;
  this allows magnetism to produce electricity! \\
  This is the \textbf{second connection between electricity and magnetism}.
\end{enumerate}

The process, whereby a time-changing magnetic field initiates or influences an electric current, is called \defnterm{magnetic induction}. \\

So a transformer uses electricity to produce magnetism to produce electricity. \\
However, the transformer moves that power from the current in one coil of wire through a magnetic field to the current in a second coil of wire.

\subsection{Alternating Current and a Coil of Wire}
Because currents are magnetic, the coil becomes an electromagnet. \\
Since the current passing through the coil reverses periodically, so does its magnetic field. \\
Moreover, because a magnetic field that changes with time produces an electric field, the coil's alternating magnetic field produces an alternating electric field. \\
As the coil's current increases, the induced electric field pushes that current backward and thereby opposes its increase. \\
As the coil's current decreases, the induced electric pushes that current forward and thereby opposes its decrease. \\

\textbf{Lenz's Law}: \\
When a changing magnetic field induces a current in a conductor, the magnetic field from that current opposes the change that induced it. \\

A wire coil's natural opposition to current change makes it useful in electrical equipment and electronics, where it's called an \defnterm{inductor}. \\

The coil's induced electric field do positive or negative work on the current and thereby shift that current from one voltage to another. \\
The coil's overall voltage shift, from one end of the coil to the other, is known as its \defnterm{induced emf} (electromagnetic force);
so current enters the coil at one voltage and exits at another voltage. \\

Current flows through the coil from higher voltage to lower voltage only half the time. \\
The other half, current flows from lower voltage to higher voltage. \\

When current flows toward lower voltage, the induced emf removes electrostatic potential energy from the current. \\
When current flows toward higher voltage, the induced emf returns electrostatic potential energy to the current. \\
When it's not in the current, it ``missing'' energy is in the coil's magnetic field. \\
Magnetic fields contain energy:
$$\text{energy} = \frac{\text{magnetic field}^{2} \cdot \text{volume}}{2 \cdot \text{permeability of free space}}$$
\emph{Strong permanent magnets store so much magnetic energy that they can be dangerous if you break them. The pieces will flip around violently and you may get pinched}. \\

The coil in the transformer stores it briefly in the magnetic field and then return it to the current. \\
Storing energy when magnitude of current increases---the field strengthens and the current loses voltage. \\
The coil returns energy while the magnitude of the current decreases---the field weakens and the current gains voltage. \\
Because the coil's self-induced emf is responsible for bouncing the energy back to the current, it's frequently called a \defnterm{back emf}.

\subsection{Two Coils Together: A Transformer}
A single coil experiences only self-inductance.
Any energy removed from the coil's current by its induced emf must eventually returned to that \emph{same} current;
it has nowhere else to go. \\
When two coils share the same electromagnetic environment, they experience mutual inductance and can exchange energy via magnetic induction.
Energy removed from one coil's current by its induced emf can be given to the other coil's current by its induced emf. \\

So the transformer transfer electric power from one circuit to another.
It consists of two coils, primary and secondary, wrapped around a magnetizable core that enhances magnetic induction and allows them to share the same electromagnetic environment \\
When AC flows through the primary, it produces an induced electric field that affects both coils and both coils develop induced emfs.
The induced emf in the primary coil removes energy from its AC while the induced emf in the secondary coil gives that energy to its AC. \\



%%%%%%%%%%%%%%%%%%%
%% R O C K E T S %%
%%%%%%%%%%%%%%%%%%%
\newpage
\section{Rockets - Special Lecture} \lecture{Mon. Oct. 17, 2016}
\textbf{Chris A. Hadfield} --- Colonel, Astronaut \\
A rocket goes forward by transfer of momentum.
The combustion chamber (contains fuel and oxygen) generates momentum and it transfer to the rocket itself to move itself forward. \\
This is due to Newton's Third Law of Motion. \\
Note: it is \emph{wrong} to think that rocket push against anything to go forward. \\

In 1957, a rocket was launched that had enough speed to go into space and stay in orbit. \\

Hypergolic fuel is used in small rockets because the its components cause an explosion (i.e. ignites) as soon as they come into contact with each other. \\
This allows reliable thrust from the rockets. \\

Space is \emph{arbitrarily} decided to be 100km above ground. \\

\subsection{Components of a Rocket}
The basic components are
\begin{enumerate}
  \item[1.] A payload (i.e. thing you carry)

  \item[2.] A rocket

  \item[3.] Where the stuff comes out

  \item[4.] Steering
\end{enumerate}

For the space shuttle, we have multiple rockets and the payload is the shuttle itself.

\subsection{Accelerating a Rocket}
The fuel and oxygen that we need to shoot out the nozzle (where we shoot mass out) needs to come out at such a momentum such that a rocket can be taken into space. \\
Remember, a rocket is \emph{very} heavy! The fuel and oxygen we carry also adds to that mass! \\

For a space shuttle, the two side rockets only burn about two minutes to push the shuttle above the air. \\
Then the side rockets are detached and the orange (big) rocket and the shuttle's internal rockets do the rest of the job! \\

Stages of \textbf{Saturn V}:
\begin{itemize}
    \item First Stage \\
    \textbf{Power}: Five F-1 engines with combined thrust of 7.5 million pounds \\
    \textbf{Propellants}: RP-1 kerosene, 214,2000 gallons. Liquid oxygen, 346,400 gallons. \\
    Fueled weight of stage, 5,028,000 pounds

    \item Second Stage \\
    \textbf{Power}: Five J-2 engines with a combined thrust of 1,000,000 pounds \\
    \textbf{Propellants}: Liquid hydrogen, 267,700 gallons. Liquid oxygen, 87,400 gallons. \\
    Fueled weight of stage,

    \item Third Stage \\
    \textbf{Power}:  \\
    \textbf{Propellants}:  \\
    Fueled weight of stage,

    \item Instrument Unit

    \item Apollo Spacecraft
\end{itemize}

\subsection{Coming Back}
The usual design uses (air) friction to slow down as we re-enter orbit. \\
The shuttle can land like a plane with little engines on board to control the shuttle. \\
Other rockets come in akin to a meteorite and opens up a parachute as it gets closer to the ground.

\subsection{Future Rockets}
The speed of a rocket is limited by how fast the mass can come out from the chemical reactions (burning fuel). \\
An idea is to use electromagnets to use charged particles to accelerate the rate that rockets can go at, i.e. using ions to boost rockets.

%%%%%%%%%%%%%%%%%%%%%%%%%%
%% RADIO AND TELEVISION %%
%%%%%%%%%%%%%%%%%%%%%%%%%%
\section{Radio and Television}


%%%%%%%%%%%%%%%%%%%%%%%%%%%%%%%%%%%%
%% PHYSICS AND THE ORIGIN OF LIFE %%
%%%%%%%%%%%%%%%%%%%%%%%%%%%%%%%%%%%%
\newpage
\section{Physics and the Origin of Life}
\lecture{November 7, 2016}
\textbf{Special Lecture} \\

\subsection{Introduction}
Humans wonder:
\begin{itemize}
  \item Who are we? [Nature of life]
  \item Where do we come from? [Origin of life]
  \item Are we alone? [Extraterrestrial life]
  \item What is this place? [Survey of the universe]
  \item How does it all work? [Main physical]
  \item Why is there something instead of nothing?
  \item What principles unify everything?
  \item What is the ultimate nature of reality?
  \item What mysteries remain?
\end{itemize}

\subsection{Energy}
\textbf{Energy} is the ``lifeblood'' of the universe. \\
The story of everything that was, is, or will be, is the story of \textbf{energy transformations}. \\

Einstein discovered two \textbf{clues}:
\begin{itemize}
  \item First clue: $E = mc^{2}$
  \item \textbf{All} forms of energy has \textbf{mass}
  \item A hot cup of coffee (thermal energy) has \textbf{more mass} than the same cup of coffee, cold!
  \bigskip
  \item Conversely, mass \emph{itself} has \textbf{energy}
  \item Life on Earth survives by \emph{literally} eating the Sun!
  \bigskip
  \item Second clue: $G_{ab} + \Lambda g_{ab} = \frac{8\pi G}{c^{4}}T_{ab}$
  \item \textbf{Mass-energy} is inextricably woven into the very fabric of \textbf{space \& time} itself
  \item This is a powerful clue in our search for a \textbf{unified ``theory of everything''}
\end{itemize}

Remarkably, it appears as if energy is \textbf{conserved}:
\begin{itemize}
  \item If energy \textbf{decreases} here, it \textbf{increases} there, so that \textbf{total = constant}
  \item Energy can \textbf{move},
  \item If so, the universe contains the \textbf{same amount of energy today as it did at the Big Bang}
  \item Energy is \textbf{eternal}.
  We can't ``use'' energy, or ``waste'' energy.
  Energy will \textbf{never run out}.
  \item What's running out is \textbf{useful} energy (called ``\textbf{free energy}'')
  \item If the universe were to reach \textbf{equilibrium} (no \textbf{gradients} of temperature, pressure, density, ...) energy could no longer \textbf{flow/transform}.
  ``\textbf{Lifeblood}'' would ``\textbf{freeze}.''
\end{itemize}

Energy will never run out, but it ``\textbf{runs down}'':
\begin{itemize}
  \item The \textbf{quantity} of energy is constant, but the \textbf{quality} continually decreases
  \item High quality/``ordered'' energy $\Rightarrow$ Lower quality/``disordered'' energy
  \item ``\textbf{Disorder}'' == ``\textbf{entropy}''
  \item The entropy/``disorder'' of the universe is continually increasing
  \item Called ``\textbf{Second Law of Thermodynamics}''
\end{itemize}

\subsection{Entropy}
The flow to disorder is the ``force'' that \textbf{animates} living organisms, and possibly \textbf{created} life in the first place. \\

Living organisms \textbf{do not consume energy}. \\
Energy In $=$ Energy Out. \\

They \textbf{consume order}, so as to maintain their own high degree of order, against their tendency towards disorder and death. \\

\textbf{Diffusion of particles} is an example of entropy/``disorder'' increasing. \\
Diffusion happens \textbf{spontaneously}: there are simply \textbf{many more ways} for the particles to be \textbf{spread out}, than \textbf{localized}. \\
This spreading continues until \textbf{equilibrium} is reached. \\

\textbf{Diffusion of energy} is another example of entropy/``disorder'' increasing. \\
Thermal energy \textbf{spontaneously} flows from hot to cold, \textbf{spreading out} the energy amongst more particles, until thermal \textbf{equilibrium} is reached and \textbf{net energy flow creases} (``death''). \\

\textbf{Summary}: the entropy/``disorder'' of the universe \textbf{spontaneously} increases, \textbf{on balance}. \\
``\textbf{Loophole}'': ``disorder'' can \textbf{decreases locally} (order \textbf{increases} locally) \emph{provided} disorder increases \textbf{more} elsewhere, so that the \textbf{net} ``disorder'' increases. \\
This is how life is possible: living organisms \textbf{spontaneously extract order (the flow to) disorder}. \\
This is the \textbf{essence of life}. \\


\subsection{Order from Disorder}
Machines exists all around us that \textbf{spontaneously} extract \textbf{order} from (the flow to) \textbf{disorder}. \\

Thermal energy \textbf{spontaneously} flows from ``Hot'' to ``Cold'', increasing the ``disorder'' of the universe. \\

Along the way it heats the Gas in the cylinder. \\
Hot gas atoms bombard the piston, which absorbs \textbf{only right-moving kinetic energy (a simple ``one-way valve'')}. \\

Some of the \textbf{disordered} $KE$ \\

This \textbf{ordered/high quality} energy is used to move people, build cities, and create order of all kinds, but must be accompanied by \emph{greater} \textbf{destruction of order} elsewhere. \\

Early on, humans mastered burning fuels to \textbf{produced heat} (wood, coal, oil, etc.), but only recently figured out how to burn fuels to \textbf{create order}. \\

A \emph{human} is the \textbf{same sort of machine}:
\begin{itemize}
  \item It takes in high quality/ordered energy (food)
  \item It ``burns'' it and expels low quality/disordered energy (excrement \& body heat)
  \item IN the process, it constructs highly ordered materials (DNA, cells, etc.) and extracts highly ordered energy for locomotion, consciousness, etc.
  \item Once you have the machine, you just need to ``feed'' it;
  it will \textbf{function spontaneously}
  \item What animates living organisms is simply the \textbf{spontaneous} dispersion of energy \& particles from more ordered to less ordered forms
  \item A human \textbf{consumes order} to maintain its relatively low entropy/highly ordered form
\end{itemize}

\subsection{Human as Piston Engine?}
ATP is the \textbf{universal} energy currency of \textbf{all} life on Earth.
Animals, plants, etc. \\

It provides the \textbf{high quality energy} needed for synthesis of proteins and membranes, all cell movements and functions, cellular division, etc. \\

\textbf{Electrochemical battery!} E.g.: Drop a cube of \textbf{iron} (Fe) into a \textbf{copper} (Cu) sulphate solution. \\
Electrons \textbf{spontaneously} flow from the \textbf{Fe} to the \textbf{Cu\textsuperscript{2+} ions}, in \textbf{random} directions. \\
IDEA: \emph{force} electrons to flow in \textbf{one direction only, exactly} like we extracted \textbf{ordered KE} in the piston engine!
Then use electric current to ``charge'' ADP up to ATP. \\

Nature has evolved a \textbf{similar battery}, using \textbf{glucose \& oxygen}:
\begin{itemize}
  \item Glucose gets ``snipped'' into two \textbf{pyruvate} ions (act like \textbf{Fe} to \textbf{donate} electrons)
  \item Oxygen has a strong \textbf{affinity} for electrons (acts like \textbf{Cu\textsuperscript{2+}})
  \item The \textbf{spontaneous flow} of electrons from pyruvate to oxygen (``burning sugar'') is used to charge ADP to ATP
\end{itemize}

\subsection{The Complexity Problem}
\textbf{Problem}: The ADP to ATP energy machinery is \textbf{very complex}!
\begin{itemize}
  \item The electron flow drives a \textbf{proton pump} that forces protons across a membrane, creating a \textbf{proton gradient} that \textbf{stores potential energy} (like a \textbf{battery}) for later use
  \item This store of energy is \textbf{tapped} by allowing protons to flow back across the membrane through a machine called the \textbf{ATP synthase}
  \item The \textbf{electric field} in the proton gradient \textbf{accelerates} the protons to large KE.
  They collide with angled vanes attached to a shaft, \textbf{forcing it to rotate}
  \item Like a rotary motor, the turning shaft \textbf{mechanically} converts ADP to ATP.
  It's very much a \textbf{physical ``gear and wheels'' machine}, not traditional ``chemistry''
  \item This type of ``energy machinery'' is \textbf{universal to all life}, \textbf{as universal as the genetic code itself}
  \item It exists inside \textbf{mitochondria} in cells
\end{itemize}

\subsection{Spontaneous Emergence of Order/Complexity}
The universe began in a hot dense state (Big Bang) with (almost) no order/structure. \\
Ever since then it has been \textbf{expanding} and \textbf{cooling}, causing various structures to ``\textbf{condense out}.'' \\

At each stage, \textbf{local} order/structure \textbf{increases}, but \textbf{net} \\

%%%%%%%%%%%%%%%%%%%%%%
%% ALTERNATE ENERGY %%
%%%%%%%%%%%%%%%%%%%%%%
\newpage
\section{Alternate Energy}
Honda Prius has a Hydrogen fuel cell.
\subsection{Generator}
A generator contains two coils with a spinning magnet in-between them. \\
As the magnet spins, the changing magnetic field produces electric current, so we generate electric power. \\
Lenz's law: harder to turn when there's a larger load. \\

The crank generator converts mechanical power into electrical power as you turn the crank. \\

A typical example of a generator is a windmill;
as it turns, it turns the mechanical energy into electrical energy. \\

In an electric current and alternating current, every time the direction of the current changes, the field of the magnet changes as well. \\
So an electric motor uses this to generate power. \\

The magnetic motor works with both AC and DC. \\
It turns faster with more current;
slower with less current. \\

\subsection{Gasoline Engine}
Gasoline engine phases: \\
\begin{itemize}
  \item Induction
    \begin{itemize}
      \item Piston is at the very top, the exhaust valve is closed
      \item As the piston is moved down, we're sucking a mixture of air and gasoline into the engine
    \end{itemize}
  \item Compression
    \begin{itemize}
      \item The cylinder compresses the air fuel mixture, increasing the temperature
      \item The magnet will cause a spark with burns the air and gasoline
      \item The burn (chemical energy) pushes the piston down. Converts chemical into kinetic (the piston moving)
    \end{itemize}
  \item Power
  \item Exhaust
    \begin{itemize}
      \item The exhaust valve is open, letting out the exhaust out
    \end{itemize}
\end{itemize}

Four-stroke engines required two full rotations to generate power. \\
Two-stroke only needs one rotation. \\

The piston changes the up and down direction of the crank shaft. \\
The shaft turns the gears. \\

Having more pistons (that goes at different times) allow a smoother generation of power. \\
One piston generates power, and another piston generates power as the first exhausts. \\

There are ``dead'' positions where we need to start it before it can generate power. \\

To turn off the car, turn off the gasoline, so we can't generate more power. \\

There are energy wasted in a card:
\begin{itemize}
  \item Drag (air resistance). Can be overcome with sleeker cars (aerodynamic)
  \item Gasoline Engine, the engine itself is inefficient.
  \item Braking (friction), turns that energy into heat
\end{itemize}

ABS pulses the brakes when it recognizes the car is skidding. \\
Penalty is that the stopping distance increases.

\subsection{Hybrid Engine}
\textbf{Full-Power}: gas engine and electric motor, drains batteries. \\

\textbf{Braking}: motor/generator acts as generator, slows car and charges the batteries. \\
The kinetic energy gets converted into electric energy as the car slows down (decreasing energy). \\
There's also a friction brake for faster brakes. \\

\textbf{Highway Driving}: gas engine pushes car and charges batteries with generator. \\
No benefits of a hybrid engine due to constant speed on the highway. \\

Another benefit is it allows a smaller gasoline engine. \\
Accelerating can use both the gasoline engine and electric engine. \\

Electric motor has a constant torque which allows better acceleration than gas engine (spins faster for higher acceleration).

\subsection{Hydrogen Fuel Cell}
Not primary source of energy. \\
Need to make Hydrogen somehow. \\
Behaves like a battery. \\

A battery, recharge by adding hydrogen. \\

The hydrogen ions split into hydrogen and free electrons;
oxygen comes in and uses those to make water. \\

If running out of Hydrogen, we charge up the car again. \\

A current through the water separates it into Hydrogen and Oxygen. \\

Faster to charge, replace the chemical. \\
Electric engine could take a long time to recharge.

\subsection{Solar Energy}
Heat. \\
Generate Steam. \\
Concentrate Solar Energy. \\

Parabolic mirrors focus the solar energy at one area. \\
This can create very (extremely) high heat at a particular location (super heats water to generate steam). \\
This has some environmental concerns: heat attracts bugs which attracts birds and birds get incinerated as they fly through the focal point. \\

Solar cells uses sunlight to make electricity.

%%%%%%%%%%%%%
%% QUANTUM %%
%%%%%%%%%%%%%
\newpage
\section{Quantum Mechanics}

\clearpage
\printindex
\end{document}

%%%%%%%%%%%%%%%%%%%%%
%% D O C U M E N T %%
%%%%%%%%%%%%%%%%%%%%%
