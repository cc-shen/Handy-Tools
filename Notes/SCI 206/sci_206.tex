\documentclass[12pt]{article}
\usepackage[margin = 1.5in]{geometry}
\setlength{\parindent}{0in}
\usepackage{amsfonts, amssymb, amsthm, mathtools, tikz, qtree, float}
\usepackage{algpseudocode, algorithm, algorithmicx}
\usepackage[T1]{fontenc}
\usepackage[utf8]{inputenc}
\usepackage{DejaVuSans}
\usepackage{ae, aecompl, color}
\usepackage{wrapfig}
\usepackage{multicol, multicol, array}
\usepackage{imakeidx}
\makeindex[columns=2, title=Indices, intoc]

\usepackage[pdftex, pdfauthor={Charles Shen}, pdftitle={SCI 206: The Physics of How Things Work}, pdfsubject={Lecture Notes from SCI 206: at the University of Waterloo}, pdfkeywords={course notes, notes, Waterloo, University of Waterloo}, pdfproducer={LaTeX}, pdfcreator={pdflatex}]{hyperref}
\usepackage{cleveref}

\DeclarePairedDelimiter{\set}{\lbrace}{\rbrace}
\renewcommand*\familydefault{\sfdefault}
\definecolor{darkish-blue}{RGB}{25,103,185}

\hypersetup{
  colorlinks,
  citecolor=darkish-blue,
  filecolor=darkish-blue,
  linkcolor=darkish-blue,
  urlcolor=darkish-blue
}

\theoremstyle{definition}
\newtheorem{ex}{Example}[subsection]

\crefname{ex}{Example}{Example}

\setlength{\marginparwidth}{1.5in}
\newcommand{\lecture}[1]{
  \marginpar{{
    \footnotesize $\leftarrow$ \underline{#1}}
  }
}

\newcommand{\defnterm}[1]{\textbf{\textcolor{teal}{#1}}\index{#1}}

\newcolumntype{L}[1]{>{\raggedright\let\newline\\\arraybackslash\hspace{0pt}}m{#1}}
\newcolumntype{C}[1]{>{\centering\let\newline\\\arraybackslash\hspace{0pt}}m{#1}}
\newcolumntype{R}[1]{>{\raggedleft\let\newline\\\arraybackslash\hspace{0pt}}m{#1}}

\allowdisplaybreaks

\makeatletter
\def\blfootnote{\gdef\@thefnmark{}\@footnotetext}
\makeatother

%%%%%%%%%%%%%%%%%%%%%
%% D O C U M E N T %%
%%%%%%%%%%%%%%%%%%%%%

\begin{document}
\let\ref\Cref
\pagenumbering{roman}

\title{\bf{SCI 206: The Physics of How Things Work}}
\date{Fall 2016, University of Waterloo \\ \center Notes written from Stefan Idziak's lectures.}
\author{Charles Shen}

\blfootnote{Feel free to email feedback to me at
\href{mailto:ccshen902@gmail.com}{ccshen902@gmail.com}.}

\maketitle
\thispagestyle{empty}
\newpage
\tableofcontents
\newpage
\pagenumbering{arabic}

%%%%%%%%%%%%%%%%%
%% M O T I O N %%
%%%%%%%%%%%%%%%%%

\section{Motion}
\subsection{Inertia}
\defnterm{Inertia} states that a body in motion tends to remain in motion; a body at rest tends to remain at rest. \\

At any particular moment, you're located at a \defnterm{position}---that is, it is a specific point in space. \\
Whenever a position is reported, it is always as a \defnterm{distance} and \defnterm{direction} from some reference point. \\

Position is an example of a vector quantity. \\
A \defnterm{vector quantity} consists of both a magnitude and a direction; the \defnterm{magnitude} tells you how much of the quantity there is, while the \defnterm{direction} tells you which way the quantity is moving. \\

\defnterm{Velocity} measures the rate at which your position is changing with time. \\
Its magnitude is your \defnterm{speed}, the distance you travel in a certain amount of time,
$$\text{speed} = \frac{\text{distance}}{\text{time}}$$
and its direction is the direction in which you're heading. \\
A velocity of zero is special, because it has no direction. \\

When your velocity is constant and doesn't changes, you \defnterm{coast}.

\subsection{Newton's Law}
\defnterm{Newton's First Law of Motion}: an object that is not subject to any outside forces moves at a constant velocity, covering equal distances in equal times along a straight-line path. \\

The outside influences referred in the first law are called \defnterm{forces}, a technical term for pushes and pulls. \\
An object that moves in accordance with Newton's first law is said to be inertial. \\

\defnterm{Mass} is the measure of your inertia, your resistance to changes in velocity.
Almost everything in the universe has mass. \\
Mass has no direction, so it's not a vector quantity. \\
It is a \defnterm{scalar quantity}---that is, a quantity that only has an amount. \\

When something pushes on you, your velocity changes; in other words, you accelerate. \\
\defnterm{Acceleration}, a vector quantity, measures the rate at which your velocity is changing with time. \\
Any changes in velocity is acceleration. \\

You accelerate in response to the \defnterm{net force} you experience---the sum of all the individual forces being exerted on you.
\begin{align*}
\text{acceleration} &= \frac{\text{net force}}{\text{mass}} \\
\text{net force} &= \text{mass} \cdot \text{acceleration} \\
F_{net} &= m \cdot a
\end{align*}
This relationship is known as \defnterm{Newton's Second Law of Motion};
the net force exerted on an object is equal to the object's mass times its acceleration.
The acceleration is in the same direction as the net force. \\

\defnterm{Newton's Third Law of Motion}: for every force that one object exerts on a second object, there is an equal but oppositely directed force that the second object exerts on the first object. \\

Forces that are directed exactly away from surfaces are called \defnterm{normal forces}, since the term \defnterm{normal} is used by mathematicians to describe something that points exactly away from a surface---at a right angle or perpendicular to that surface.

\subsection{Weight and Gravity}
\defnterm{Gravity} is the physical phenomenon that produces attractive forces between every pair of objects in the universe. \\

Earth exerts a downward force on any object near its surface.
That object is attracted directly toward the centre of Earth with a force we call the object's \defnterm{weight}.
This weight is exactly proportional to the object's mass. \\
Weight is how hard gravity pulls on an object, and mass is how difficult an object is to accelerate.

We have
$$\text{weight} = \text{mass} \cdot \text{acceleration due to gravity}$$
In other words: you can lose weight by either reducing your mass or by going someplace, like a small planet, where gravity is weaker.

The Earth's acceleration due to gravity is $9.8N/kg$ or $9.8m/s^{2}$.

\subsection{The Velocity of a Falling Ball}
The current (present) velocity can be calculated as
\begin{align*}
  \text{present velocity} &= \text{initial velocity} + \text{acceleration} \cdot \text{time} \\
  v_{f} &= v_{i} + a \cdot t
\end{align*}

\subsection{The Position of a Falling Ball}
The average velocity is exactly halfway in between the two individual velocities:
$$\text{average velocity} = \text{initial velocity} + \frac{1}{2}\cdot\text{acceleration}\cdot\text{time}$$

The present position can be calculated as the following:
\begin{align*}
  \text{present position} &= \text{initial position} + \text{initial velocity}\cdot\text{time} +
  \frac{1}{2}\cdot\text{acceleration}\cdot\text{time}^{2} \\
  x_f &= x_i + v_i \cdot t + \frac{1}{2} \cdot a \cdot t^2
\end{align*}

\subsection{Tossing the Ball Upward}
The larger the initial velocity of the ball, the longer it rises and the higher it goes before its velocity is reduced to zero. \\
It then descends for the same amount of time it spent rising. \\
The higher the ball goes before it begins to descent, the longer it takes to return to the ground and the faster it's travelling when it arrives.

\subsection{Energy}
The capacity to make things happen is called \defnterm{energy}, and the process of making them happen is called \defnterm{work}. \\
Energy and work are both physical quantities, meaning that both are measurable. \\

Physical \defnterm{energy} is defined as the capacity to do work. \\
Physical \defnterm{work} refers to the process of transferring energy. \\

Energy is what's transferred, and work does the transferring.
The most important characteristic of energy is that it's conserved.
In physics, a \defnterm{conserved quantity} is one that can't be created or destroyed but that can be transferred between objects or, in the case of energy, be converted from one form to another. \\

When you throw a ball, it picks up speed and undergoes an increase in \defnterm{kinetic energy}, energy of motion that allows the ball to do work on whatever it hits. \\
When you list a rock, it shifts farther from Earth and undergoes an increase in \defnterm{gravitational potential energy}, energy stored in the gravitational forces between the rock and Earth that allows the rock to do work on whatever it falls on. \\
\defnterm{Potential energy} is energy stored in the forces between or within objects.

\subsection{Doing Work}
To do work on an object, you must push on it while it moves in the direction of your push. \\
This relationship can be expressed as the following:
\begin{align*}
  \text{work} &= \text{force} \cdot \text{distance} \\
  W &= F \cdot d
\end{align*}
\emph{If you're not pushing or it's not moving, then you're not working.} \\

Energy has no direction.
It can be hidden as potential energy. \\
\emph{Kinetic energy} is the form of energy contained in an object's motion. \\
\emph{Potential energy} is the form of energy stored in the forces between or within objects.

\subsection{Gravitational Potential Energy}
\begin{align*}
\text{gravitational potential energy} &= \text{mass} \cdot \text{acceleration due to gravity} \cdot \text{height} \\
U &= m \cdot m \cdot h \\
The~higher~it~was,&~the~harder~it~hits.
\end{align*}

A ramp provides \defnterm{mechanical advantage}, the process whereby a mechanical device redistributes the amount of force and distance that go into performing a specific amount of mechanical work.

\subsection{The Seesaw}
Overall movement of an object from one place to another is called the \defnterm{translational motion}. \\
A seesaw does not experience translational motion, it does turn around the pivot in the centre, and thus it experiences a different kind of motion. \\
Motion around a fixed amount (which prevents translation) is called \defnterm{rotational motion}. \\

The concepts and laws of rotational motion have many analogies in the concepts and laws of translational motion.

\subsection{The Motion of a Dangling Seesaw}

\defnterm{Rotational inertia} states that a body that's rotating tends to remain rotating; a body that is not rotating tends to remain not rotating. \\
At any particular moment, the seesaw is oriented in a certain way---that is, it has an \defnterm{angular position}.
It describes the seesaw's orientation relative to some reference orientation. \\

\defnterm{Angular velocity} measures the rate at which the seesaw's angular position is changing with time. \\
Its magnitude is the \defnterm{angular speed}, the angle through which the seesaw turns in a certain amount of time,
$$\text{angular speed} = \frac{\text{change in angle}}{\text{time}}$$
and its direction is the axis about which that rotation proceeds. \\
The seesaw's \defnterm{axis of rotation} is the line in space about which the seesaw is rotating.

\defnterm{Torques} is a technical term for twists and spins; its SI unit is \emph{newton-meter} ($N \cdot m$). \\

\defnterm{Newton's First Law of Rotational Motion} \\
A rigid object that is not wobbling and is not subject to any outside torques rotates at a constant angular velocity, turning equal amounts in equal times about a fixed axis of rotation. \\

\subsection{The Seesaw's Centre of Mass}
There is a special point in or near a free object about which all of its mass is evenly distributed and about which it naturally spins---its \defnterm{centre of mass}. \\
The axis of rotation passes right through this point so that, as the free object rotates, the centre of mass doesn't move unless the object has an overall translational velocity. \\

The point around which all the physical quantities of rotation are defined is the \defnterm{centre of rotation} or pivot. \\
For a free object, the natural pivot point is its centre of mass. \\
For a constrained object, the best pivot point may be determined by the constraints. \\

\subsection{How the Seesaw Responds to Torques}
The \defnterm{rotational mass} is the measure of an object's \emph{rotational} inertia, its resistance to changes in its \emph{angular} velocity;
also known as \textbf{\textcolor{teal}{moment of inertia}}\index{moment of inertia|see {rotational mass}}. \\
An object's rotational mass depends both on its ordinary mass and how that mass is distributed within the object. \\
The SI unit for rotational mass is \emph{kilogram-meter\textsuperscript{2}} ($kg \cdot m^2$). \\

An object's rotational mass depends on how far its mass is from the axis of rotation, so its rotational mass may change when its axis of rotation changes, even if it's rotating about its centre of mass. \\
Less torque is required to spin a tennis racket about its handle (racket's mass is fairly close to the axis and the rotational mass is small) than to flip the racket head-over-handle (both the head and the handle are far away fro m the axis and the rotational mass is large). \\

When something exerts a torque on an object, its angular velocity changes; i.e. it undergoes angular acceleration. \\
The \defnterm{Angular acceleration} measures the rate at which the \emph{angular} velocity is changing with time;
The SI unit is \emph{radian per second\textsuperscript{2}} ($1/s^{2}$). \\
It's analogous to acceleration, which measures the rate at which an object's \emph{translational} velocity is changing with time. \\
Just as with acceleration, angular acceleration involves both a magnitude and a direction.
An object undergoes angular acceleration when its angular speed increases or decreases or when its angular velocity changes direction. \\
If an object experiences several torques at once, it undergoes angular acceleration in response to the \defnterm{net torque} it experiences, the sum of all the individual torques being exerted on it.
$$\text{angular acceleration} = \frac{\text{net torque}}{\text{rotational mass}}$$

The relation is also known as \defnterm{Newton's Second Law of Rotational Motion}.
$$\text{net torque} = \text{rotational mass} \cdot \text{angular acceleration}$$
\emph{Spinning a marble is much easier than spinning a merry-go-round}. \\

This law does not apply to nonrigid or wobbling objects, because nonrigid objects can change their rotational masses and wobbling ones are affected by more than one rotational mass simultaneously.

\subsubsection{Summary}
\begin{enumerate}
  \item Your angular position indicates exactly how you're oriented
  \item Your angular velocity measures the rate at which your angular position is changing with time
  \item Your angular acceleration measures the rate at which your angular velocity is changing with time
  \item For you to undergo angular acceleration, you must experience a net torque
  \item The more rotational mass you have, the less angular acceleration you experience for a given torque
\end{enumerate}

\subsection{Forces and Torques}
A force can produce a torque and a torque can produce a force. \\

Nothing happens if you push on the seesaw exactly where the pivot passes through it;
there's no angular acceleration! \\
If you move a little away from the pivot, you can get the seesaw rotating, but you have to push hard. \\
You do much better if you push on the end of the seesaw, where even a small force can start the seesaw rotating. \\
The shortest distance and the direction from the pivot to the place where you push on the seesaw is a vector quantity called the \defnterm{lever arm};
in general, the longer the lever arm, the less force it takes to cause a particular angular acceleration. \\
The torque is proportional to the length of the lever arm;
i.e. you obtain more torque by exerting force farther from the pivot or axis of rotation. \\

When your force is directed parallel to the lever arm, it produces no torque. \\
To produce a torque, your force must have a component that is perpendicular to the lever arm and only that perpendicular component contributes to the torque. \\

The torque produced by a force is equal to the lever arm times that force, where we include on the component of the force that is perpendicular to the lever arm.
$$\text{torque} = \text{lever arm} \cdot \text{force perpendicular to the lever arm}$$
\emph{When twisting an unyielding object, it helps to use a long wrench}.

\subsection{Balancing and Unbalancing the Seesaw}
When a child sits on one end of the seesaw, the child's weight produces a torque on the seesaw about its pivot.
Gravity is ultimately responsible for that torque, so it can be referred to as \emph{gravitational} torque. \\

A \emph{balanced} seesaw experiences no overall gravitational torque about its pivot. \\
If nothing twists it, the balanced seesaw is inertial---the net torque on it its zero. \\
When the children aren't fidgeting, the seesaw is rigid and wobble-free, so it rotates at constant angular velocity, in according to Newton's first law of rotational mass.
If it's motionless, it stays motionless.
It it's running, it continues turning at a steady pace. \\

Two children with unequal weight can also balance the seesaw by sitting at different distances from the pivot.
A heavier child should sit closer to the pivot, and vice versa. \\

With two children on the seesaw, it has a considerable weight but that weight produces no torque on the seesaw about its pivot. \\
Because the seesaw's \defnterm{centre of gravity}, the effective location of the seesaw's overall weight, is located at the pivot. \\
With the centre of gravity located at the pivot, gravity has no lever arm with which to produce a torque on the seesaw about the pivot. \\

Any object or system of objects has a centre of gravity---the effective location of its overall weight. \\
While centre of gravity is a gravitational concept and centre of mass is an inertial concept, the two conveniently coincide;
the seesaw's centre of gravity and its centre of mass are located at the same point.
That coincidence stems from the proportionally between an object's weight and its mass near Earth's surface. \\

\subsection{Levers and Mechanical Advantage}
The seesaw's mechanical advantage allows even the tiniest child at its end to list the heaviest adult sitting near its pivot.
Because the child travels much farther than the adult, the child's work on the seesaw equals the seesaw's work on the adult. \\
This effect---a small force exerted for a short distance elsewhere in that system---is an example of the mechanical advantage associated with levers. \\

Also, each child is using the seesaw board to exert a torque on the other child about the pivot, and that torque cancels the other child's gravitational torque. \\
This is \defnterm{Newton's Third Law of Rotational Mass}: if one object exerts a torque on a second object, then the second object will exert an equal but oppositely directed torque on the first object. \\

\emph{Note that} not all equal but oppositely directed torques are Newton's third law pairs. \\
On a balanced seesaw, each child experiences two torques that \emph{happen} to be equal but oppositely directed: a torque due to gravity and a torque due to the other child.
Those two torques form a matched pair because the seesaw is balanced, not because of Newton's third law. \\
If the seesaw were not balanced, those two torques would not be equal but oppositely directed and the children would both be experiencing angular acceleration.

\subsection{Friction}
\defnterm{Friction} is a phenomenon that opposes the relative motion of two surfaces in contact with each one another. \\
Two surfaces that are in \defnterm{relative motion} are travelling with different velocities so that a person standing still on one surface will observe the other surface as moving.
In opposing relative motion, friction exerts forces on both surfaces in directions that tend to bring them to a single velocity. \\

Frictional forces always oppose relative motion, but they vary in strength according to
\begin{enumerate}
  \item how tightly the two surfaces are pressed against one another \\
  i.e. the harder you press two surfaces together, the larger the frictional forces they experience.
  \item how slippery the surfaces are \\
  i.e. roughening the surfaces generally increases friction, while smoothing or lubricating them generally reduces it
  \item whether the surfaces are actually moving relative to one another
\end{enumerate}

\subsection{A Microscopic View of Friction}
All surfaces experience friction as they rub across one another at various contact points. \\
When the surfaces slide across one another, these contact points collide, producing sliding friction and wear. \\

Increasing the size or number of microscopic projections (contact points) by roughening the surfaces generally leads to more friction. \\
Increasing the number of contact points by squeezing the two surfaces more rightly together also leads to more friction.
The microscopic projections simply collide more often. \\

A useful rule of thumb is that the frictional forces between two firm surfaces are proportional to the forces pressing those two surfaces together. \\

Friction also causes wear when the colliding contact points break one another off.
With time, this wear can remove large amounts of material so that even seemingly indestructible stone steps are gradually worn away by foot traffic. \\
The best way to reduce wear between two surfaces is to polish them so that they are extremely smooth.
The smooth surfaces will still touch at contact points and experience friction as they slide across one another, but their contact points will be broad and round and will rarely break one another off during a collision.

\subsection{Static Friction, Sliding Friction, and Traction}
When two surfaces are moving across one another, \defnterm{sliding friction} acts to stop them from sliding. \\

But even when those surfaces have the same velocity, \defnterm{static friction} may act to keep them form starting to slide across one another in the first place. \\
Static friction always exerts a frictional force that exactly balances your push.
However, the force that static friction can exert is limited. \\

By exerting more horizontal force on it than static friction can exert in the other direction;
the net force is then no longer zero, and the object accelerates. \\
Once the object is moving, static friction is replaced by sliding friction.
Because sliding friction acts to bring the object back to reset, and the object must be kept pushing to keep it moving. \\
As the object is being pushed, the contact points between the surfaces no longer have time to settle into one another, and they consequently experience weaker horizontal forces. \\
Thus, force of sliding friction is generally weaker than that of static friction. \\

Both types of frictions (sliding and static) are incorporated in the concept of \defnterm{traction}, the largest amount of frictional force that one surface can obtain from another surface at any given moment. \\
When an object is at rest, its traction is equal to the maximum amount of static friction. \\
When the object is moving, its traction reduces to the amount of force that sliding friction exerts. \\

Another difference between static and sliding friction is that sliding friction wastes energy. \\
It converts useful, \defnterm{ordered energy}, work or energy that can easily do work, into relatively useless \defnterm{thermal energy}, a disordered energy that's associated with temperature. \\
Sliding friction makes things hotter by turning work into thermal energy. \\

In the case of a file cabinet, work becomes thermal energy as your slide the cabinet across the floor at \emph{constant} velocity.
Since the cabinet isn't accelerating, you and the floor are pushing the cabinet equally hard in opposite directions. \\
Since you're pushing the cabinet forward as it moves forward, you're doing positive work on the cabinet.
Sliding friction from the floor, however, is pushing the cabinet backward as it moves forward, so sliding friction is doing negative work on the cabinet. \\
Overall, sliding friction is removing energy from the cabinet just as fast you're adding it. \\

Static friction doesn't waste energy as thermal energy;
it enables objects to do work on one another.

\subsection{Friction and Energy}
Sliding friction converts work into thermal energy. \\

Recall that energy is the capacity to do work, and it is transferred between objects by doing that work. \\
Energy is a conserved physical quantity---it cannot be created or destroyed;
it can only be transferred between objects or converted from one form to another.  \\
Energy can be thought of the \emph{conserved quantity of doing}. \\

Energy's two basic forms are kinetic (energy contained in the motions of objects) and potential (energy stored in the forces between or within those objects). \\
The individual forms of energy are \emph{not} conserved quantities;
only \emph{total} energy---the sum of all the individual forms---is conserved. \\

Thermal energy is a mixture of kinetic and potential energies. \\
The kinetic energy and potential energies in thermal energy are contained entirely within those objects. \\
Any object has \defnterm{internal energy}, energy held entirely within that object by its individual particles and forces between those particles;
\emph{thermal energy} is the portion of internal energy that's associated with temperature. \\
Thermal energy makes particle in the object jiggle randomly and independently;
at any moment, each particle has its own tiny supply of potential and kinetic energies, and this dispersed and disordered energy is collectively referred to as thermal energy. \\

As you push a file cabinet across the floor, you do work on it, but it doesn't pick up speed.
Instead, sliding friction converts your work into thermal energy, and the cabinet and floor become hotter as the energy you transfer to them disperses among their particles. \\
But although sliding friction easily turns work into thermal energy, there's no easy way to turn thermal energy back into work. \\
Disorder not only makes things harder to use, but it also is difficult to undo.
Suppose a mug is dropped on the floor and it shatters into a thousand pieces, the mug is still there;
however, it's disordered and thus much less useful.

\subsection{Wheels}
As a wheel spins, it lowers a portion of its surface onto the surface, leaves it there briefly so that it may experience taking its place. \\
Due to the touch-and-release character of rolling, there is no sliding friction between the wheel and the surface. \\

As the wheel rotates, its hub slides across the stationary axle at its centre. \\
This sliding friction wastes energy and causes wear to both hub and axle. \\
While the narrow hub moves relatively slowly across the axle so that the work and wear done each second are small, sliding friction is undesirable. \\

A solution is to insert rollers between the hub and axle. \\
The result is a roller bearing, a mechanical device that eliminates sliding friction between a hub and an axle. \\
A complete bearing consists of two rings separated by rollers that keep those rings from rubbing against one another.
The bearing's outer inner ring is attached to the stationary axle, while its outer ring is attached to the spinning wheel hub.

\subsection{Power and Rotational Work}
\defnterm{Power} is the physical quantity that measures the rate at which a car's engine does work---the amount of work it does in a certain amount of time, or
$$\text{power} = \frac{\text{work}}{\text{time}}$$
The SI unit of power is the \textbf{joule per second}, also called \textbf{watt}. \\

The car's engine does work on each its powered wheels in the context of \emph{rotational} motion, where work involves torque and angle. \\
The engine does work on a wheel by exerting a torque on the wheel as the wheel rotates through an angle in the direction of the torque. \\

To do work on an object via rotational motion, you must twist it as it rotates through an angle in the direction of your twist. \\
Work can be calculated as:
$$\text{work} = \text{torque} \cdot \text{angle}(\text{in radians})$$
\emph{If you're not twisting or it's not turning, then you're not working}. \\

The engine does work on its wheels via rotational motion, those wheels do work on the car via translational motion. \\
Each wheel grips the pavement with static friction, its contact point with the ground can't move.
Instead, the rotating wheel pushes the axle and car forward as they move forward.
So energy that flows into the wheels via rotational motion returns to the car via translational motion and keeps the car moving. \\

Power can be supplied by translational motion or rotational motion:
$$\text{power} = \text{force} \cdot \text{velocity}$$
$$\text{power} = \text{torque} \cdot \text{angular velocity}$$

\subsection{Kinetic Energy}
Car's brakes are designed to stop the car by turning its kinetic energy into thermal energy.
Brake rubs stationary brake pads against spinning metal discs. \\

A way to calculate kinetic energy is
$$\text{kinetic energy} = \frac{1}{2} \cdot \text{mass} \cdot \text{speed}^{2}$$

\emph{Racing around at twice the speed takes four times the energy}. \\
The dramatic increase in kinetic energy that results from a modest increase in speed explains why high-speed crashes are far deadlier than those at lower speeds. \\

Rotating objects have kinetic energy as well. \\
The kinetic energy of rotational motion depends on inertia and speed;
however, it's the rotational inertia and rotational speed that matter
$$\text{kinetic energy} = \frac{1}{2} \cdot \text{rotational mass} \cdot \text{angular speed}^{2}$$
\emph{It takes a very energetic person to spin his wheels twice as fast}.

\subsection{Coasting Forward: Linear Momentum}
A bumper car's translational inertia makes it hard to start or stop, and its rotational inertia makes it difficult to spin or stop from spinning. \\
There are two other conserved quantities besides energy in nature: \emph{linear momentum} and \emph{angular momentum}. \\

When bumper cars crash into one another, they exchange more than just energy. \\
Energy is direction-less since it's a scalar quantity. \\
The direction of travel after a collision is a conserved vector quantity called linear momentum. \\
\defnterm{Linear momentum}, or \defnterm{momentum}, is the measure of an object's translational impetus or tenacity---its determination to keep moving the way it's currently moving. \\
In other words, momentum is the \emph{conserved quantity of moving};
it can neither be created nor destroyed, only transferred between objects. \\
It can be calculated as such
$$\text{momentum} = \text{mass} \cdot \text{velocity}$$
\emph{It's hard to stop a fast-moving truck}. \\
The faster an object is moving or the more mass it has, the more momentum it has in the direction of its velocity. \\
The SI unit of momentum is \emph{kilogram-meter-per-second} ($kg \cdot m / s$).

\subsection{Exchanging Momentum in a Collision: Impulses}
Momentum is transferred to an object by giving it an \defnterm{impulse}, a force exerted on it for a certain amount of time. \\
When the motor and floor pushes your bumper car forward for a few seconds, they give your car an impulse and transfer momentum to it.
This impulse is the change in your car's momentum \\
Impulse can be calculated as the following
\begin{align*}
\text{impulse} &= \text{force} \cdot \text{time} \\
\Delta p &= F \cdot t
\end{align*}

The more force or the longer that force is exerted, the larger the impulse and the more an object's momentum changes. \\
Like momentum itself, an impulse is an vector quantity. \\

Forces that colliding objects exert on one another as they exchange momentum are called \defnterm{impact forces}. \\
Bumper cars have soft rubber bumpers instead of hard steel bumpers because a collision would last only an instant and would involve enormous impact forces. \\

A moving object carries only momentum, not force. \\
A coasting object is free of any net force.
However, when that object hits an obstacle, the two exchange momentum via impulses, and impulses involve forces.

\subsection{Spinning in Circles: Angular Momentum}
Another conserved vector quantity during a collision is \emph{angular momentum}. \\
\defnterm{Angular momentum} is the measure of an object's rotational impetus or tenacity, its determination to keep rotating the way it's currently rotating. \\
Angular momentum is the \emph{conserved quantity of turning}. \\
It can be calculated as the following
$$\text{angular momentum} = \text{rotational mass} \cdot \text{angular velocity}$$

Angular momentum has the same direction as the angular velocity;
the faster an object is spinning or the larger its rotational mass, the more angular momentum it has in the direction of its angular velocity. \\
The SI unit of angular momentum is \emph{kilogram-meter\textsuperscript{2} per second} ($kg \cdot m^{2} / s$). \\

Angular momentum is also a conserved quantity;
it cannot be created or destroyed, only transferred between objects. \\
For an object to start spinning, something must be transfer angular momentum to it, and the object will then continue to spin until it transfer this angular momentum elsewhere.

\subsection{Glancing Blows: Angular Impulses}
Angular momentum is transferred to an object by giving it an \defnterm{angular impulse}, which is a torque exerted on it for a certain amount of time. \\
Angular impulse is the change in an object's angular momentum and it can be found via
$$\text{angular impulse} = \text{torque} \cdot \text{time}$$
\emph{To get a merry-go-round spinning rapidly, you must twist it hard and for a long time}. \\

An angular impulse is a vector quantity and has the same direction as the torque. \\
As with linear momentum, sudden transfers of angular momentum can break things, so bumper cars are designed to limit their \emph{impact torques} to reasonable levels. \\

Angular momentum is conserved because of Newton's third law of rotational motion. \\
When an object exerts a torque on another object for a certain amount of time, the other object exerts an equal but oppositely directed torque on first object for exactly the same amount of time. \\

By reducing rotational mass, say pulling arms in when spinning, a skater will spin more rapidly since the said skater experiences zero net torque and angular momentum must remain constant. \\
If the skater spreads out his/her arms, there's more rotational mass and thus the spinning slows down.

\subsection{Potential Energy, Acceleration, and Force}
A \defnterm{gradient}, a vector quantity, characterizes how some physical quantity changes gradually with position.
It points in the direction of the fastest increase in its physical quantity, and its magnitude is the rate of that increase. \\

Suppose a car going up a ramp as a potential energy gradient;
that is, the car's potential energy increases as the car moves up the ramp gradually. \\
This gradient, known as \defnterm{potential gradient}, points uphill---in the direction of the quickest potential energy increase---and its magnitude is the increase in the car's potential energy caused by a small uphill step divided by the length of that step. \\
The steeper the ramp, the larger the car's potential gradient. \\

Note that the car's potential gradient and its acceleration point in exactly opposite directions. \\
The car's potential gradient points in the direction that \emph{increases} its potential energy as quickly as possible, whereas car accelerates in the direction that \emph{reduces} its potential energy as quickly as possible. \\
The negative of the car's uphill potential gradient is the downhill force on the car
$$\text{force} = -\text{potential gradient}$$
\emph{An object accelerates in the direction that reduces its potential energy as quickly as possible}.

%%%%%%%%%%%%%%%%%%%%%%%
%% R E S O N A N C E %%
%%%%%%%%%%%%%%%%%%%%%%%

\newpage
\section{Resonance}
\subsection{Stretching a Spring}
There's a relationship between the length of a spring and the forces it is exerting on its ends. \\
The spring scale, which uses a spring, determines how much upward force it is exerting on an object by measuring the length of its spring. \\

If a spring is neither stretched nor compressed, it exerts no forces on its ends. \\
If no other forces act this relaxed spring, each end is in \defnterm{equilibrium}---it experiences zero net force.
Equilibrium occurs whenever the forces acting on an object sum to zero so that the object does not accelerate. \\
So the relaxed spring is at its \defnterm{equilibrium length}, its natural length when no forces are exerted on its ends.
The spring tries to return to this equilibrium length no matter how you distort it. \\

Suppose the left end of a spring is fixed to a wall and its right end to a move-able bar.
When the bar isn't pulling or pushing on the springs right end, that end is in equilibrium at a particular location–its \defnterm{equilibrium position}.
Since the spring's right end naturally returns to this equilibrium position if the bar disturbs it and then lets go, the end is in a \defnterm{stable equilibrium}. \\
Now suppose that the bar moves the spring's right end to the right.
The stretched spring now exerts a leftward force on the right end, trying to restore the spring to its equilibrium position, called \defnterm{restoring force}. \\
The spring exerts restoring forces on both of its end, pulling them equally hard in opposite directions.;
however, the left end is held in place by the wall. \\
The farther the bar stretches the spring to the right, the stronger its restoring forces on the right end becomes.
The restoring force is exactly proportional to how far the bar has stretched the spring. \\
If the bar compresses the spring, the restoring force now exerts a rightward force on its right end. \\

Whenever a spring is distorted away from its equilibrium length, it exerts a restoring force on its move-able end that is proportional to the distortion and points in the opposite direction;
this is known as \defnterm{Hooke's Law} and the law can be presented as
$$\text{restoring force} = - \text{spring constant} \cdot \text{distortion}$$
\emph{The stiffer the spring and the farther you stretch it, the harder it pulls back}. \\
The \defnterm{spring constant} is a measure of the spring's stiffness.
The larger the constant---that is, the firmer the spring---the larger the restoring force the spring exerts for a given distortion. \\

Hooke's Law is not limited to only the behaviour of coil springs. \\
Many objects respond to distortion with restoring forces that are proportional to how far you've distorted them away from their equilibrium lengths---or their equilibrium shapes.
\defnterm{Equilibrium shape} is the shape an object adopts when it's not subject to any outside forces. \\

There's a limit to Hooke's law;
if you distort an object too far, it will usually begin to exert less force than Hooke's law demands. \\
This is due to the object has exceeded its \defnterm{elastic limit} and it will have deformed it permanently in the process. \\
If you stretch a spring too far, it won't return to its equilibrium length when you release it. \\

Distorting a spring requires work;
when it's being pulled, there's work being done as some energy is transferred to the spring. \\
This energy is \defnterm{elastic potential energy}. \\
If you reverse the motion, the spring returns most of this energy to your hand, and the remainder is converted to thermal energy by frictional effects inside the spring itself. \\
A spring that is distorted away from its equilibrium shape always contains elastic potential energy.

\subsection{Natural Resonances}
Practical clocks are based on a particular type of repetitive motion called a \defnterm{natural resonance}.
In a natural resonance, the energy in an isolated object or system of objects causes it to perform a certain motion over and over again. \\
An object that has been displaced from its stable equilibrium accelerates toward that equilibrium but then overshoots;
it coasts right through equilibrium and must turn around to try again.
As long as it has excess energy, this object continues to glide back and forth through its equilibrium and thus exhibits a natural resonance.

\subsection{Pendulums and Harmonic Oscillators}
A pendulum's centre of gravity is directly below its pivot, it is in a stable equilibrium.
Its centre of gravity is then as low as possible, so displacing it raises its gravitational potential energy and a restoring force begins pushing it back toward that equilibrium position. \\
This restoring force is almost exactly proportional to how far the pendulum is from equilibrium. \\
The pendulum swings back and forth about its equilibrium position in a repetitive motion called an \defnterm{oscillation}. \\
As it swings, its energy alternates between potential and kinetic forms.
This repetitive transformation of excess energy from one form to another is part of any oscillation and keeps the oscillator---the system experiencing the oscillation---moving back and forth until that excess energy is either converted into thermal energy or transferred elsewhere. \\

The pendulum is a \defnterm{harmonic oscillator}---the simplest and best understood mechanical system in nature---because its restoring forces is proportional to its displacement from equilibrium. \\
As a harmonic oscillator, the pendulum undergoes \defnterm{simple harmonic motion}, a regular and predictable oscillation that keeps passage of time with remarkable accuracy. \\

The \defnterm{period} of any harmonic oscillator, the time it takes to complete one full cycle of its motion, depends only on how stiffly its restoring force pushes it back and forth, and on how strongly its inertia resists the accelerations of that motion. \\
\defnterm{Stiffness} is the measure of how sharply the restoring force increases as the oscillator is displaced from equilibrium;
stiff restoring forces are associated with firm or hard objects, while less stiff restoring forces are associated with soft objects. \\
The stiffer the restoring force, the more forcefully it pushes the oscillator back and forth and the shorter the oscillator's period. \\
The larger the oscialltor's mass, the less it accelerates and the longer its period. \\

The most important characteristic of a harmonic oscillator is that its period \emph{doesn't} depend on \defnterm{amplitude}, its furthest displacement from equilibrium. \\
The insensitivity to amplitude is a consequence of its special restoring force, a restoring force that is proportional to its displacement from equilibrium. \\
The harmonic oscillator completes a large cycle of motion just as quickly as it completes a small cycle of motion. \\

In a pendulum, increasing its mass doesn't increase its period.
Because increasing the mass also increases its weight and therefore stiffens its restoring force.
These two changes compensate for one another perfectly so that the pendulum's period remains unchanged. \\
The period does depend on its length and on gravity. \\
When a pendulum's length is reduced, the restoring force is stiffed and hence shortens its period.\\
When you strength gravity, you increase the pendulum's weight, stiffen its restoring force, and reduce its period. \\
Note that materials expand with increasing temperature, so a simple pendulum slows down as it heats up.
A better, more accurate pendulum is thermally compensated by using several different materials with different coefficients of loudness expansion to ensure that its centre of mass remains at a fixed distance from its pivot.

\subsection{Pendulum Clocks}
The top of the pendulum has a two-pointed anchor that controls the rotation of a toothed wheel;
this mechanism is called \defnterm{escapement}. \\
A weighted cord wrapped around the toothed wheel's shaft exerts a torque on that wheel, so that the wheel would spin if the anchor weren't holding it in place. \\
Each time the pendulum reaches the end of a swing, one point of the anchor releases the toothed wheel while the other point catches it. \\
The wheel turns slowly as the pendulum rocks back and forth, advancing one tooth for each full cycle of the pendulum.
This wheel turns a series of gears, which slowly advances the clock's hands. \\

The toothed wheel also keeps the pendulum moving by giving the anchor a tiny forward push each time the pendulum completes a swing.
This work on the anchor and pendulum replaces energy lost to friction and air resistance. \\
This energy comes from the weighted cord, which releases gravitational potential energy as its weight descends.
When you wind the clock, you rewind this cord around the shaft, lifting the weight and replenishing its potential energy. \\

A clock works best when its pendulum swings with almost perfect freedom.
Any outside force---even the push from the toothed wheel---will influence the pendulum's period. \\
A good pendulum clock uses an aerodynamic pendulum and low-friction bearings. \\

If the pendulum is displaced too far, it becomes an \defnterm{anharmonic oscillator}---its restoring force ceases to be proportional to its displacement from equilibrium, and its period begins to depend on its amplitude.

\subsection{Balance Clocks}
A small wheel attached to a coil spring is a harmonic oscillator known as a \defnterm{balance ring}, used in most mechanical clocks and watches.
Its period is determined only by the stiffness of the coil spring and the rotational mass of the wheel. \\
It resembles a tiny metal bicycle wheel, supported at its centre of mass/gravity by an axle and a pair of bearings. \\
Any friction in the bearings is exerted so close to the ring's axis of rotation that it produces little torque and the ring turns extremely easily.
The ring pivots about its own center of gravity so that its weight produces no torque on it. \\

The only thing exerting a torque on the balance ring is a tiny coil spring. \\
One end of this spring is attached to the ring, while the other is fixed to the body of the clock.
When the spring is not distorted, it exerts on torque on the ring and the ring is in equilibrium.\\
Since this restoring torque is proportional to the ring's rotation away from a stable equilibrium, the balance ring and coil spring form a harmonic oscillator. \\

The period of the ring depends on the \emph{torsional} stiffness of the coil spring, that is, on how rapidly the spring's torque increases as you twist it, and on the balance ring's \emph{rotational} mass. \\
The balance ring's period doesn't depend on the amplitude of its motion. \\
Gravity exerts no torque on the balance ring, so this timekeeper works anywhere and in any orientation. \\

As the balance ring rocks back and forth, it tips a lever that controls the rotation of a toothed wheel.
An anchor attached to the lever allows the wheel to advance one tooth for each complete cycle of the balance ring's motion. \\
Gears connected to the tooth wheel to the clock's hands, which slowly advance as the wheel turns. \\
The balance clock can't draw energy from a weighted cord.
Instead, it has a main spring that exerts a torque on the toothed wheel. \\
The main spring is a coil of elastic metal that stores energy when you wind the clock.
Its energy keeps the balance ring rocking steadily back and forth and also turns the clock's hands.
The main spring unwinds as the toothed wheel turns, the clock occasionally needs winding.

\subsection{Electronic Clocks}
The potential accuracy of pendulum and balance clocks is limited by friction, air resistance, and thermal expansion to about 10 s per year. \\

Many modern clocks use quartz oscillators as their timekeepers. \\
A quartz oscillator is made from a single crystal of quartz.
A quartz crystal oscillates strongly after being struck. \\
It is a harmonic oscillator because it acts like a spring with a metal cylinder at each end.
The two cylinders oscillate in and out symmetrically about their combined center of mass, with a period determined only by the cylinders' masses and the spring's stiffness. \\
In a quartz crystal, the spring is the crystal itself and the cylinders' are its two halves.
Since their restoring forces are proportional to displacement from equilibrium, both systems are harmonic oscillators. \\

A quartz crystal's restoring forces are extremely stiff due to its hardness.
A tiny distortion leads to large restoring forces. \\
Since the period of a harmonic oscillator decreases as its spring becomes stiffer, a typical quartz oscillator has an extremely short period. \\
Its motion is usually called a \defnterm{vibration} rather than an oscillation because vibration implies a fast oscillation in a mechanical system.
\emph{Oscillation} itself is a more general term for any repetitive process, and it can even apply to such non-mechanical processes as electric or thermal oscillations. \\

A fast oscillator is characterized by its \defnterm{frequency}, the number of cycles it completes in a certain amount of time. \\
The SI unit of frequency is the \emph{cycle per second}, also called the \defnterm{hertz} (Hz)
$$\text{frequency} = \frac{1}{\text{period}}$$

Quartz clocks are normally electronic. \\
The crystal's vibrations are too fast and too small for most mechanical devices to follow. \\
A quartz crystal is intrinsically electronic itself;
it responds mechanically to electrical stress and electrically to mechanical stress. \\
Crystalline quartz is known as a \emph{piezoelectric} material and is ideal for electronic clocks.

\subsection{Sound and Music}
In air, \defnterm{sound} consists of density waves, patterns of compression and rarefaction that travel outward rapidly from their source. \\
When a sound passes by, the air pressure in your ear fluctuates up and down about normal atmospheric pressure. \\

When the fluctuations are repetitive, you hear a \emph{tone} with a pitch equal to the fluctuation's frequency.
\defnterm{Pitch} is the frequency of a sound.

\subsection{A Violin's Vibrating String}
The violin subjects its strings to \defnterm{tension}, outward forces that act to stretch it, and this tension gives each string an equilibrium shape---a straight line. \\
Tension exerts a pair of outward forces on each piece of the string;
its neighbouring pieces are pulling that piece toward them.
Since the string's tension is uniform, these two outward forces sum to zero;
they have equal magnitudes but point in opposite directions. \\
With zero net force on each of its pieces, the straight string is in equilibrium. \\

When the string is curved, the pairs of outward forces no longer sum to zero;
they still have equal magnitudes and pointing in different directions. \\
Each piece experiences a small net force. \\
The net forces on its pieces are restoring forces because they act to straighten the string.
If the string is distorted and then released, these restoring forces will cause the string to vibrate about its straight equilibrium shape in a natural resonance. \\
The more the string is curved, the stronger the restoring forces on its pieces become.
The restoring forces are \defnterm{springlike forces}---they increase in proportion to the string's distortion---so the string is a form of harmonic oscillator. \\

A string can bend and vibrate in many distinct \defnterm{modes}, or basic patterns of distortion, each with its own period of vibration. \\
The period of each vibrational mode is independent of its amplitude. \\

A violin has one simplest vibration---its \defnterm{fundamental vibrational mode}.
The string's kinetic energy peaks as it rushes through its straight equilibrium shape, and its elastic potential energy peaks as it stops to turn around. \\
The string's midpoint travels the farthest (the \defnterm{vibrational antinode}), while its ends remain fixed (the \defnterm{vibrational nodes}). \\
Its vibrational period depends only on the stiffness of its restoring forces and on its inertia.
Either stiffening the violin string or reducing its mass will quicken its fundamental vibration and increase its fundamental pitch. \\

A string's fundamental pitch also depends on its length.
Shortening the string both stiffens it and reduces its mass, so its pitch increases.
A shorter string curves more sharply when it's displaced from equilibrium and therefore subjects its pieces to larger net forces. \\

A \defnterm{mechanical wave}, the natural motions of an extended object about its stable equilibrium shape or situation. \\
An \emph{extended} object is one like a string, stick, or lake surface that has many parts that move with limited independence. \\
The string's fundamental mode is a particularly simple wave, a \defnterm{standing wave}, which is a wave with fixed nodes and antinodes.
A standing wave's basic shape doesn't change with time;
it merely scales up and down rhythmically at a particular frequency and amplitude (its peak extend of motion). \\
An important characteristic is that the standing wave doesn't travel along the string. \\
Not only does a standing wave extends along the string, its associated oscillation is \emph{perpendicular} to the string and therefore \emph{perpendicular} to the wave itself.
A wave in which the underlying oscillation is perpendicular to the wave itself is called a \defnterm{transverse wave}.

\subsection{The Violin String's Harmonics}
A violin string also has \defnterm{higher-order vibrational modes} in which the string vibrates as a chain of shorter strings arcing in alternate directions.
Each of these higher-order modes is another standing wave, with a fixed shape that scales up and down rhythmically at its own frequency and amplitude. \\
A string's vibrational frequency is inversely proportional to its length, so halving its length doubles its frequency. \\
Frequencies that are integer multiples of the fundamental pitches are called \defnterm{harmonics}, so a half-string vibration occurs at the second harmonic pitch and is called the \emph{second harmonic mode}. \\

The string often vibrates in more than one mode at the same time.
For instance, a string vibrating in its fundamental mode can also vibrate in its second harmonic and emit two tones at once. \\
\defnterm{Timbre} is the mixture of the fundamental tones and the harmonics in an instrument. \\
When a violin string is vibrating in several modes at once, its shape and motion are complicated.
The individual standing waves add on top of one another, a process known as \defnterm{superposition}. \\
Each vibrational mode has its own amplitude and therefore its own loudness contribution to the string's timbre. \\

While the waves have virtually no effect on one another, the string's overall distorted shape is the superposition of the individual wave shapes that changes substantially with time. \\
The different harmonic waves vibrate at different frequencies, and their superposition changes as they change. \\
The string's overall wave is not a standing wave, and its features can even move along the string.

\subsection{Bowing and Plucking the Violin String}
The bow exerts frictional forces on the strings as it moves across them. \\
The bow hairs grab the string and push it forward with static friction.
The string's restoring forces eventually overpowers the static friction, and the string starts sliding backward across the hairs. \\
Since the hairs exerts little sliding friction, the string completes half a vibrational cycle with ease.
When it stops to reverse direction, the hairs grab the string again and begin pushing it forward and this process is repeated over and over. \\

Each time the bow pushes the string forward, it does work on the string and adds energy to the string's vibrational modes.
This is called \defnterm{resonant energy transfer}, in which a modest force doing work in synchrony with a natural resonance can transfer a large amount of energy to that resonance. \\
Similar rhythmic pushes can cause other objects to vibrate strongly. \\
A wine glass' response to a certain tone is also an example of \defnterm{sympathetic vibration}, the transfer of vibrational energy between two systems that share a common vibrational frequency. \\

Bowing the string nearer its middle reduces the string's curvature, weakening its harmonic vibrations and giving it a more mellow sound. \\
Bowing the string nearer its end increases the string's curvature, strengthening its harmonic vibrations and giving it a brighter sound. \\

Sometimes a tone from an instrument or sound system will cause some object in the room to begin vibrating loudly. \\
The object has a natural resonance at the tone's frequency, and sympathetic vibration is transferring energy to the object. \\
Energy moves easily between two objects that vibrate at the same frequency. \\

Resonance energy transfer make it possible for sound to shatter a crystal wineglass. \\
When the sound pushes on the glass rhythmically, the sound slowly transfer energy to the glass, until it finally shatters.

\subsection{An Organ Pipe's Vibrating Air}
A pipe organ uses vibrations to create sound. \\
Its vibration takes place in the air itself.
The pipe is open at each end and filled with air.
Since that air is protected by rigid walls of the pipe, its pressure can fluctuate up and down relative to the atmosphere pressure and it can exhibit natural resonances. \\

In its fundamental vibrational mode, air moves alternately toward and away from the pipe's center.
As air moves toward the pipe's center, the density there rises and a pressure imbalance develops.
Since the pressure at the center is higher than its ends, air accelerates \emph{away} from the center. \\
The air eventually stops moving inward and moves outward.
As air moves away from the center, the density in the center drops and a reversed pressure imbalance occurs. \\
Air now accelerates \emph{towards} the center, and the cycle repeats. \\
The air's kinetic energy peaks each time it rushes through that equilibrium and its pressure potential energy peaks each time it stops to turn around. \\

The air column is a harmonic oscillator, where its vibrational frequency depends only on the stiffness of its restoring forces and on its inertia. \\
Stiffening the air column or reducing its mass will quicken its vibration and increase its pitch.
These characteristics depend on the length of the pipe. \\
A shorter pipe holds less air mass and it offers stiffer opposition to any movements of air in and out of that pipe.
With less room in the shorter pipe, the pressure inside it rises and falls more abruptly, hence stiffer restoring forces on the moving air. \\
An organ pipe's vibrational frequency is inversely proportional to its length. \\
The mass of vibrating air in a pipe also increases with the air's average density, so a change in temperature or weather will alter the pipe's pitch. \\

The fundamental vibrational mode of air in the organ is a standing wave. \\
Air in the pipe is an extended object with a stable equilibrium, and the disturbance associated with its fundamental vibrational mode has a basic shape that doesn't change with time;
it merely scales up and down rhythmically. \\

The shape of the wave in the pipe's air has back-and-forth compression and rarefaction, not with side-to-side displacements. \\
All wave's associated oscillation is \emph{along} the pipe and therefore \emph{along} the wave itself.
A wave in which the underlying oscillation is parallel to the wave itself is called a \defnterm{longitudinal wave}. \\
Waves in the air, including those inside organ pipes and other wind instruments and those in the open air, are all longitudinal waves. \\

\subsection{Playing an Organ Pipe}
The organ uses resonant energy transfer to make the air in a pipe vibrate.
As the air flows across the opening, it's easily deflected to one side or the other and tends to follow any air that's already moving into or out of the pipe. \\
If the air inside the pipe is vibrating, the new air will follow it in perfect synchrony and strength the vibration. \\

In its fundamental vibrational mode, the pipe's entire column of air vibrates together. \\
in the higher-order vibrational modes, this air column vibrates as a chain of shorter air columns moving in alternate direction.
If the pipe has a constant width, these vibrations occur at harmonics of the fundamental.
When the air column vibrates as two half-columns, its pitch is exactly twice that of the fundamental mode;
three third-columns lead to three times the fundamental, and so on. \\

The standing waves superpose and the fundamental and harmonic tones are produced together;
the sake of the organ pipe and the place where air is blown across it determine the pipe's harmonic content and thus its timbre.

\subsection{A Drum's Vibrating Surface}
A drumhead has a stable equilibrium and springlike restoring forces, its overtone vibrations \emph{aren't} harmonics. \\
A drumhead is effectively two-dimensional or surface-like, it doesn't divide easily into pieces that resemble the entire drumhead. \\

Since striking a drumhead causes it to vibrate in several modes at once, the drum emits several pitches simultaneously.
The amplitude of each mode, and consequently its loudness, depends on \emph{how hard} you hit the drumhead and also on \emph{where} you hit it.
Circular modes if hit at its centre;
non-circular if hit near its edge. \\

\subsection{Sound in Air}
Air is in a stable equilibrium when its density is uniform (neglecting gravity). \\
If disturbed, the resulting pressure imbalances will provide springlike restoring forces.
These forces, along with air's inertia, lead to rhythmic vibrations---the vibrations of harmonic oscillators. \\

In open air, the most basic vibrations are waves that move steadily in a particular direction and are therefore called \defnterm{travelling waves};
those waves are longitudinal. \\
A basic travelling sound wave consists of an alternating pattern of high-density regions, called \defnterm{crests}, and low-density regions, \defnterm{troughs}.
The highs of any wave is a crest and the lows of any wave is a trough. \\
The shortest distance between two adjacent crests (or troughs) is the \defnterm{wavelength}. \\

A standing wave's crests and troughs flip back and forth in place, crests becoming troughs and troughs becoming crests. \\
A travelling wave's crests and troughs move steadily in a particular direction at a particular speed. \\
The speed and direction of travel is the travelling wave's \defnterm{wave velocity}. \\
The wave speed can be calculated via the following
$$\text{wave speed} = \text{wavelength} \cdot \text{frequency}$$
\emph{Broad waves that vibrate quickly travel fast.} \\

The \defnterm{speed of sound} in air is about $331 m/s$ under standard conditions are sea level ($0^{\circ}C, 101,325~Pa$ pressure). \\

When a marching band steps quickly towards and away from the the listener, the listener hears its music shifted up or down in pitch.
Known as the \defnterm{Doppler effect}, this frequency shift occurs because the listener encounters sound wave crests at a rate that's different from the rate at which those crests were created. \\
If two are moving away from one another, the listener encounters the crests at a decreased rate and the pitch decreases.


%%%%%%%%%%%%%%%%%%%%%
%% P R E S S U R E %%
%%%%%%%%%%%%%%%%%%%%%

\newpage
\section{Pressure}


\clearpage
\printindex
\end{document}

%%%%%%%%%%%%%%%%%%%%%
%% D O C U M E N T %%
%%%%%%%%%%%%%%%%%%%%%
