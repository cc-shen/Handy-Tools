\documentclass[12pt]{article}
\usepackage[margin = 1.5in]{geometry}
\setlength{\parindent}{0in}
\usepackage{amsfonts, amssymb, amsthm, mathtools, tikz, qtree, float}
\usepackage{algpseudocode, algorithm, algorithmicx}
\usepackage[T1]{fontenc}
\usepackage[utf8]{inputenc}
\usepackage{DejaVuSans}
\usepackage{ae, aecompl, color}
\usepackage{wrapfig}
\usepackage{multicol, multicol, array}
\usepackage{imakeidx}
\makeindex[columns=2, title=Indices, intoc]

\usepackage[pdftex, pdfauthor={Charles Shen}, pdftitle={SCI 206: The Physics of How Things Work}, pdfsubject={Lecture Notes from SCI 206: at the University of Waterloo}, pdfkeywords={course notes, notes, Waterloo, University of Waterloo}, pdfproducer={LaTeX}, pdfcreator={pdflatex}]{hyperref}
\usepackage{cleveref}

\DeclarePairedDelimiter{\set}{\lbrace}{\rbrace}
\renewcommand*\familydefault{\sfdefault}
\definecolor{darkish-blue}{RGB}{25,103,185}

\hypersetup{
  colorlinks,
  citecolor=darkish-blue,
  filecolor=darkish-blue,
  linkcolor=darkish-blue,
  urlcolor=darkish-blue
}

\theoremstyle{definition}
\newtheorem{ex}{Example}[subsection]

\crefname{ex}{Example}{Example}

\setlength{\marginparwidth}{1.5in}
\newcommand{\lecture}[1]{
  \marginpar{{
    \footnotesize $\leftarrow$ \underline{#1}}
  }
}

\newcommand{\defnterm}[1]{\textbf{\textcolor{teal}{#1}}\index{#1}}

\newcolumntype{L}[1]{>{\raggedright\let\newline\\\arraybackslash\hspace{0pt}}m{#1}}
\newcolumntype{C}[1]{>{\centering\let\newline\\\arraybackslash\hspace{0pt}}m{#1}}
\newcolumntype{R}[1]{>{\raggedleft\let\newline\\\arraybackslash\hspace{0pt}}m{#1}}

\allowdisplaybreaks

\makeatletter
\def\blfootnote{\gdef\@thefnmark{}\@footnotetext}
\makeatother

%%%%%%%%%%%%%%%%%%%%%
%% D O C U M E N T %%
%%%%%%%%%%%%%%%%%%%%%

\begin{document}
\let\ref\Cref
\pagenumbering{roman}

\title{\bf{SCI 206: The Physics of How Things Work}}
\date{Fall 2016, University of Waterloo \\ \center Notes written from Stefan Idziak's lectures.}
\author{Charles Shen}

\blfootnote{Feel free to email feedback to me at
\href{mailto:ccshen902@gmail.com}{ccshen902@gmail.com}.}

\maketitle
\thispagestyle{empty}
\newpage
\tableofcontents
\newpage
\pagenumbering{arabic}

%%%%%%%%%%%%%%%%%
%% M O T I O N %%
%%%%%%%%%%%%%%%%%

\section{Motion}
\lecture{Sept. 12, 2016}
\subsection{Inertia}
\defnterm{Inertia} states that a body in motion tends to remain in motion; a body at rest tends to remain at rest. \\

At any particular moment, you're located at a \defnterm{position}---that is, it is a specific point in space. \\
Whenever a position is reported, it is always as a \defnterm{distance} and \defnterm{direction} from some reference point. \\

Position is an example of a vector quantity. \\
A \defnterm{vector quantity} consists of both a magnitude and a direction; the \defnterm{magnitude} tells you how much of the quantity there is, while the \defnterm{direction} tells you which way the quantity is moving. \\

\defnterm{Velocity} measures the rate at which your position is changing with time. \\
Its magnitude is your \defnterm{speed}, the distance you travel in a certain amount of time,
$$\text{speed} = \frac{\text{distance}}{\text{time}}$$
and its direction is the direction in which you're heading. \\
A velocity of zero is special, because it has no direction. \\

When your velocity is constant and doesn't changes, you \defnterm{coast}.

\subsection{Newton's Law}
\defnterm{Newton's First Law of Motion}: an object that is not subject to any outside forces moves at a constant velocity, covering equal distances in equal times along a straight-line path. \\

The outside influences referred in the first law are called \defnterm{forces}, a technical term for pushes and pulls. \\
An object that moves in accordance with Newton's first law is said to be inertial. \\

\defnterm{Mass} is the measure of your inertia, your resistance to changes in velocity.
Almost everything in the universe has mass. \\
Mass has no direction, so it's not a vector quantity. \\
It is a \defnterm{scalar quantity}---that is, a quantity that only has an amount. \\

When something pushes on you, your velocity changes; in other words, you accelerate. \\
\defnterm{Acceleration}, a vector quantity, measures the rate at which your velocity is changing with time. \\
Any changes in velocity is acceleration. \\

You accelerate in response to the \defnterm{net force} you experience---the sum of all the individual forces being exerted on you.
\begin{align*}
\text{acceleration} &= \frac{\text{net force}}{\text{mass}} \\
\text{net force} &= \text{mass} \cdot \text{acceleration} \\
F_{net} &= m \cdot a
\end{align*}
This relationship is known as \defnterm{Newton's Second Law of Motion};
the net force exerted on an object is equal to the object's mass times its acceleration.
The acceleration is in the same direction as the net force. \\

\defnterm{Newton's Third Law of Motion}: for every force that one object exerts on a second object, there is an equal but oppositely directed force that the second object exerts on the first object. \\

Forces that are directed exactly away from surfaces are called \defnterm{normal forces}, since the term \defnterm{normal} is used by mathematicians to describe something that points exactly away from a surface---at a right angle or perpendicular to that surface.

\subsection{Weight and Gravity}
\defnterm{Gravity} is the physical phenomenon that produces attractive forces between every pair of objects in the universe. \\

Earth exerts a downward force on any object near its surface.
That object is attracted directly toward the center of Earth with a force we call the object's \defnterm{weight}.
This weight is exactly proportional to the object's mass. \\
Weight is how hard gravity pulls on an object, and mass is how difficult an object is to accelerate.

We have
$$\text{weight} = \text{mass} \cdot \text{acceleration due to gravity}$$
In other words: you can lose weight by either reducing your mass or by going someplace, like a small planet, where gravity is weaker.

The Earth's acceleration due to gravity is $9.8N/kg$ or $9.8m/s^{2}$.

\subsection{The Velocity of a Falling Ball}
The current (present) velocity can be calculated as
\begin{align*}
  \text{present velocity} &= \text{initial velocity} + \text{acceleration} \cdot \text{time} \\
  v_{f} &= v_{i} + a \cdot t
\end{align*}

\subsection{The Position of a Falling Ball}
The average velocity is exactly halfway in between the two individual velocities:
$$\text{average velocity} = \text{initial velocity} + \frac{1}{2}\cdot\text{acceleration}\cdot\text{time}$$

The present position can be calculated as the following:
\begin{align*}
  \text{present position} &= \text{initial position} + \text{initial velocity}\cdot\text{time} +
  \frac{1}{2}\cdot\text{acceleration}\cdot\text{time}^{2} \\
  x_f &= x_i + v_i \cdot t + \frac{1}{2} \cdot a \cdot t^2
\end{align*}

\subsection{Tossing the Ball Upward}
The larger the initial velocity of the ball, the longer it rises and the higher it goes before its velocity is reduced to zero. \\
It then descends for the same amount of time it spent rising. \\
The higher the ball goes before it begins to descent, the longer it takes to return to the ground and the faster it's traveling when it arrives.

\subsection{Energy}
The capacity to make things happen is called \defnterm{energy}, and the process of making them happen is called \defnterm{work}. \\
Energy and work are both physical quantities, meaning that both are measurable. \\

Physical \defnterm{energy} is defined as the capacity to do work. \\
Physical \defnterm{work} refers to the process of transferring energy. \\

Energy is what's transferred, and work does the transferring.
The most important characteristic of energy is that it's conserved.
In physics, a \defnterm{conserved quantity} is one that can't be created or destroyed but that can be transferred between objects or, in the case of energy, be converted from one form to another. \\

When you throw a ball, it picks up speed and undergoes an increase in \defnterm{kinetic energy}, energy of motion that allows the ball to do work on whatever it hits. \\
When you list a rock, it shifts farther from Earth and undergoes an increase in \defnterm{gravitational potential energy}, energy stored in the gravitational forces between the rock and Earth that allows the rock to do work on whatever it falls on. \\
\defnterm{Potential energy} is energy stored in the forces between or within objects.

\subsection{Doing Work}
To do work on an object, you must push on it while it moves in the direction of your push. \\
This relationship can be expressed as the following:
\begin{align*}
  \text{work} &= \text{force} \cdot \text{distance} \\
  W &= F \cdot d
\end{align*}
\emph{If you're not pushing or it's not moving, then you're not working.} \\

Energy has no direction.
It can be hidden as potential energy. \\
\emph{Kinetic energy} is the form of energy contained in an object's motion. \\
\emph{Potential energy} is the form of energy stored in the forces between or within objects.

\subsection{Gravitational Potential Energy}
\begin{align*}
\text{gravitational potential energy} &= \text{mass} \cdot \text{acceleration due to gravity} \cdot \text{height} \\
U &= m \cdot m \cdot h \\
The~higher~it~was,&~the~harder~it~hits.
\end{align*}

A ramp provides \defnterm{mechanical advantage}, the process whereby a mechanical device redistributes the amount of force and distance that go into performing a specific amount of mechanical work.

\subsection{The Seesaw}
Overall movement of an object from one place to another is called the \defnterm{translational motion}. \\
A seesaw does not experience translational motion, it does turn around the pivot in the center, and thus it experiences a different kind of motion. \\
Motion around a fixed amount (which prevents translation) is called \defnterm{rotational motion}. \\

The concepts and laws of rotational motion have many analogies in the concepts and laws of translational motion.

\subsection{The Motion of a Dangling Seesaw}

\defnterm{Rotational inertia} states that a body that's rotating tends to remain rotating; a body thats not rotating tends to remain not rotating. \\
At any particular moment, the seesaw is oriented in a certain way---that is, it has an \defnterm{angular position}.
It describes the seesaw's orientation relative to some reference orientation. \\

\defnterm{Angular velocity} measures the rate at which the seesaw's angular position is changing with time. \\
Its magnitude is the \defnterm{angular speed}, the angle through which the seesaw turns in a certain amount of time,
$$\text{angular speed} = \frac{\text{change in angle}}{\text{time}}$$
and its direction is the axis about which that rotation proceeds. \\
The seesaw's \defnterm{axis of rotation} is the line in space about which the seesaw is rotating.

\defnterm{Torques} is a technical term for twists and spins; its SI unit is \emph{newton-meter} ($N \cdot m$). \\

\defnterm{Newton's First Law of Rotational Motion} \\
A rigid object that is not wobbling and is not subject to any outside torques rotates at a constant angular velocity, turning equal amounts in equal times about a fixed axis of rotation. \\

\subsection{The Seesaw's Center of Mass}
There is a special point in or near a free object about which all of its mass is evenly distributed and about which it naturally spins---its \defnterm{center of mass}. \\
The axis of rotation passes right through this point so that, as the free object rotates, the center of mass doesn't move unless the object has an overall translational velocity. \\

The point around which all the physical quantities of rotation are defined is the \defnterm{center of rotation} or pivot. \\
For a free object, the natural pivot point is its center of mass. \\
For a constrained object, the best pivot point may be determined by the constraints. \\

\subsection{How the Seesaw Responds to Torques}
The \defnterm{rotational mass} is the measure of an object's \emph{rotational} inertia, its resistance to changes in its \emph{angular} velocity;
also known as \textbf{\textcolor{teal}{moment of inertia}}\index{moment of inertia|see {rotational mass}}. \\
An object's rotational mass depends both on its ordinary mass and how that mass is distributed within the object. \\
The SI unit for rotational mass is \emph{kilogram-meter\textsuperscript{2}} ($kg \cdot m^2$). \\

An object's rotational mass depends on how far its mass is from the axis of rotation, so its rotational mass may change when its axis of rotation changes, even if it's rotating about its center of mass. \\
Less torque is required to spin a tennis racket about its handle (racket's mass is fairly close to the axis and the rotational mass is small) than to flip the racket head-over-handle (both the head and the handle are far away fro m the axis and the rotational mass is large). \\

When something exerts a torque on an object, its angular velocity changes; i.e. it undergoes angular acceleration. \\
The \defnterm{Angular acceleration} measures the rate at which the \emph{angular} velocity is changing with time;
The SI unit is \emph{radian per second\textsuperscript{2}} ($1/s^{2}$). \\
It's analogous to acceleration, which measures the rate at which an object's \emph{translational} velocity is changing with time. \\
Just as with acceleration, angular acceleration involves both a magnitude and a direction.
An object undergoes angular acceleration when its angular speed increases or decreases or when its angular velocity changes direction. \\
If an object experiences several torques at once, it undergoes angular acceleration in response to the \defnterm{net torque} it experiences, the sum of all the individual torques being exerted on it.
$$\text{angular acceleration} = \frac{\text{net torque}}{\text{rotational mass}}$$

The relation is also known as \defnterm{Newton's Second Law of Rotational Motion}.
$$\text{net torque} = \text{rotational mass} \cdot \text{angular acceleration}$$
\emph{Spinning a marble is much easier than spinning a merry-go-round}. \\

This law does not apply to nonrigid or wobbling objects, because nonrigid objects can change their rotational masses and wobbling ones are affected by more than one rotational mass simultaneously.

\subsubsection{Summary}
\begin{enumerate}
  \item Your angular position indicates exactly how you're oriented
  \item Your angular velocity measures the rate at which your angular position is changing with time
  \item Your angular acceleration measures the rate at which your angular velocity is changing with time
  \item For you to undergo angular acceleration, you must experience a net torque
  \item The more rotational mass you have, the less angular acceleration you experience for a given torque
\end{enumerate}

\subsection{Forces and Torques}
A force can produce a torque and a torque can produce a force. \\

Nothing happens if you push on the seesaw exactly where the pivot passes through it;
there's no angular acceleration! \\
If you move a little away from the pivot, you can get the seesaw rotating, but you have to push hard. \\
You do much better if you push on the end of the seesaw, where even a small force can start the seesaw rotating. \\
The shortest distance and the direction from the pivot to the place where you push on the seesaw is a vector quantity called the \defnterm{lever arm};
in general, the longer the lever arm, the less force it takes to cause a particular angular acceleration. \\
The torque is proportional to the length of the lever arm;
i.e. you obtain more torque by exerting force farther from the pivot or axis of rotation. \\

When your force is directed parallel to the lever arm, it produces no torque. \\
To produce a torque, your force must have a component that is perpendicular to the lever arm and only that perpendicular component contributes to the torque. \\

The torque produced by a force is equal to the lever arm times that force, where we include on the component of the force that is perpendicular to the lever arm.
$$\text{torque} = \text{lever arm} \cdot \text{force perpendicular to the lever arm}$$
\emph{When twisting an unyielding object, it helps to use a long wrench}.

\subsection{Balancing and Unbalancing the Seesaw}
When a child sits on one end of the seesaw, the child's weight produces a torque on the seesaw about its pivot.
Gravity is ultimately responsible for that torque, so it can be referred to as \emph{gravitational} torque. \\

A \emph{balanced} seesaw experiences no overall gravitational torque about its pivot. \\
If nothing twists it, the balanced seesaw is inertial---the net torque on it its zero. \\
When the children aren't fidgeting, the seesaw is rigid and wobble-free, so it rotates at constant angular velocity, in according to Newton's first law of rotational mass.
If it's motionless, it stays motionless.
It it's running, it continues turning at a steady pace. \\

Two children with unequal weight can also balance the seesaw by sitting at different distances from the pivot.
A heavier child should sit closer to the pivot, and vice versa. \\

With two children on the seesaw, it has a considerable weight but that weight produces no torque on the seesaw about its pivot. \\
Because the seesaw's \defnterm{center of gravity}, the effective location of the seesaw's overall weight, is located at the pivot. \\
With the center of gravity located at the pivot, gravity has no lever arm with which to produce a torque on the seesaw about the pivot. \\

Any object or system of objects has a center of gravity---the effective location of its overall weight. \\
While center of gravity is a gravitational concept and center of mass is an inertial concept, the two conveniently coincide;
the seesaw's center of gravity and its center of mass are located at the same point.
That coincidence stems from the proportionally between an object's weight and its mass near Earth's surface. \\

\subsection{Levers and Mechanical Advantage}
The seesaw's mechanical advantage allows even the tiniest child at its end to list the heaviest adult sitting near its pivot.
Because the child travels much farther than the adult, the child's work on the seesaw equals the seesaw's work on the adult. \\
This effect---a small force exerted for a short distance elsewhere in that system---is an example of the mechanical advantage associated with levers. \\

Also, each child is using the seesaw board to exert a torque on the other child about the pivot, and that torque cancels the other child's gravitational torque. \\
This is \defnterm{Newton's Third Law of Rotational Mass}: if one object exerts a torque on a second object, then the second object will exert an equal but oppositely directed torque on the first object. \\

\emph{Note that} not all equal but oppositely directed torques are Newton's third law pairs. \\
On a balanced seesaw, each child experiences two torques that \emph{happen} to be equal but oppositely directed: a torque due to gravity and a torque due to the other child.
Those two torques form a matched pair because the seesaw is balanced, not because of Newton's third law. \\
If the seesaw were not balanced, those two torques would not be equal but oppositely directed and the children would both be experiencing angular acceleration.

\subsection{Friction}
\defnterm{Friction} is a phenomenon that opposes the relative motion of two surfaces in contact with each one another. \\
Two surfaces that are in \defnterm{relative motion} are traveling with different velocities so that a person standing still on one surface will observe the other surface as moving.
In opposing relative motion, friction exerts forces on both surfaces in directions that tend to bring them to a single velocity. \\

Frictional forces always oppose relative motion, but they vary in strength according to
\begin{enumerate}
  \item how tightly the two surfaces are pressed against one another \\
  i.e. the harder you press two surfaces together, the larger the frictional forces they experience.
  \item how slippery the surfaces are \\
  i.e. roughening the surfaces generally increases friction, while smoothing or lubricating them generally reduces it
  \item whether the surfaces are actually moving relative to one another
\end{enumerate}

\subsection{A Microscopic View of Friction}
All surfaces experience friction as they rub across one another at various contact points. \\
When the surfaces slide across one another, these contact points collide, producing sliding friction and wear. \\

Increasing the size or number of microscopic projections (contact points) by roughening the surfaces generally leads to more friction. \\
Increasing the number of contact points by squeezing the two surfaces more rightly together also leads to more friction.
The microscopic projections simply collide more often. \\

A useful rule of thumb is that the frictional forces between two firm surfaces are proportional to the forces pressing those two surfaces together. \\

Friction also causes wear when the colliding contact points break one another off.
With time, this wear can remove large amounts of material so that even seemingly indestructible stone steps are gradually worn away by foot traffic. \\
The best way to reduce wear between two surfaces is to polish them so that they are extremely smooth.
The smooth surfaces will still touch at contact points and experience friction as they slide across one another, but their contact points will be broad and round and will rarely break one another off during a collision.

\subsection{Static Friction, Sliding Friction, and Traction}
When two surfaces are moving across one another, \defnterm{sliding friction} acts to stop them from sliding. \\

But even when those surfaces have the same velocity, \defnterm{static friction} may act to keep them form starting to slide across one another in the first place. \\
Static friction always exerts a frictional force that exactly balances your push.
However, the force that static friction can exert is limited. \\

By exerting more horizontal force on it than static friction can exert in the other direction;
the net force is then no longer zero, and the object accelerates. \\
Once the object is moving, static friction is replaced by sliding friction.
Because sliding friction acts to bring the object back to reset, and the object must be kept pushing to keep it moving. \\
As the object is being pushed, the contact points between the surfaces no longer have time to settle into one another, and they consequently experience weaker horizontal forces. \\
Thus, force of sliding friction is generally weaker than that of static friction. \\

Both types of frictions (sliding and static) are incorporated in the concept of \defnterm{traction}, the largest amount of frictional force that one surface can obtain from another surface at any given moment. \\
When an object is at rest, its traction is equal to the maximum amount of static friction. \\
When the object is moving, its traction reduces to the amount of force that sliding friction exerts. \\

Another difference between static and sliding friction is that sliding friction wastes energy. \\
It converts useful, \defnterm{ordered energy}, work or energy that can easily do work, into relatively useless \defnterm{thermal energy}, a disordered energy that's associated with temperature. \\
Sliding friction makes things hotter by turning work into thermal energy. \\

In the case of a file cabinet, work becomes thermal energy as your slide the cabinet across the floor at \emph{constant} velocity.
Since the cabinet isn't accelerating, you and the floor are pushing the cabinet equally hard in opposite directions. \\
Since you're pushing the cabinet forward as it moves forward, you're doing positive work on the cabinet.
Sliding friction from the floor, however, is pushing the cabinet backward as it moves forward, so sliding friction is doing negative work on the cabinet. \\
Overall, sliding friction is removing energy from the cabinet just as fast you're adding it. \\

Static friction doesn't waste energy as thermal energy;
it enables objects to do work on one another.

\subsection{Friction and Energy}
Sliding friction converts work into thermal energy. \\

Recall that energy is the capacity to do work, and it is transferred between objects by doing that work. \\
Energy is a conserved physical quantity---it cannot be created or destroyed;
it can only be transferred between objects or converted from one form to another.  \\
Energy can be thought of the \emph{conserved quantity of doing}. \\

Energy's two basic forms are kinetic (energy contained in the motions of objects) and potential (energy stored in the forces between or within those objects). \\
The individual forms of energy are \emph{not} conserved quantities;
only \emph{total} energy---the sum of all the individual forms---is conserved. \\

Thermal energy is a mixture of kinetic and potential energies. \\
The kinetic energy and potential energies in thermal energy are contained entirely within those objects. \\
Any object has \defnterm{internal energy}, energy held entirely within that object by its individual particles and forces between those particles;
\emph{thermal energy} is the portion of internal energy that's associated with temperature. \\
Thermal energy makes particle in the object jiggle randomly and independently;
at any moment, each particle has its own tiny supply of potential and kinetic energies, and this dispersed and disordered energy is collectively referred to as thermal energy. \\

As you push a file cabinet across the floor, you do work on it, but it doesn't pick up speed.
Instead, sliding friction converts your work into thermal energy, and the cabinet and floor become hotter as the energy you transfer to them disperses among their particles. \\
But although sliding friction easily turns work into thermal energy, there's no easy way to turn thermal energy back into work. \\
Disorder not only makes things harder to use, but it also is difficult to undo.
Suppose a mug is dropped on the floor and it shatters into a thousand pieces, the mug is still there;
however, it's disordered and thus much less useful.

\subsection{Wheels}
As a wheel spins, it lowers a portion of its surface onto the surface, leaves it there briefly so that it may experience taking its place. \\
Due to the touch-and-release character of rolling, there is no sliding friction between the wheel and the surface. \\

As the wheel rotates, its hub slides across the stationary axle at its center. \\
This sliding friction wastes energy and causes wear to both hub and axle. \\
While the narrow hub moves relatively slowly across the axle so that the work and wear done each second are small, sliding friction is undesirable. \\

A solution is to insert rollers between the hub and axle. \\
The result is a roller bearing, a mechanical device that eliminates sliding friction between a hub and an axle. \\
A complete bearing consists of two rings separated by rollers that keep those rings from rubbing against one another.
The bearing's outer inner ring is attached to the stationary axle, while its outer ring is attached to the spinning wheel hub.

\subsection{Power and Rotational Work}
\defnterm{Power} is the physical quantity that measures the rate at which a car's engine does work---the amount of work it does in a certain amount of time, or
$$\text{power} = \frac{\text{work}}{\text{time}}$$
The SI unit of power is the \textbf{joule per second}, also called \textbf{watt}. \\

The car's engine does work on each its powered wheels in the context of \emph{rotational} motion, where work involves torque and angle. \\
The engine does work on a wheel by exerting a torque on the wheel as the wheel rotates through an angle in the direction of the torque. \\

To do work on an object via rotational motion, you must twist it as it rotates through an angle in the direction of your twist. \\
Work can be calculated as:
$$\text{work} = \text{torque} \cdot \text{angle}(\text{in radians})$$
\emph{If you're not twisting or it's not turning, then you're not working}. \\

The engine does work on its wheels via rotational motion, those wheels do work on the car via translational motion. \\
Each wheel grips the pavement with static friction, its contact point with the ground can't move.
Instead, the rotating wheel pushes the axle and car forward as they move forward.
So energy that flows into the wheels via rotational motion returns to the car via translational motion and keeps the car moving. \\

Power can be supplied by translational motion or rotational motion:
$$\text{power} = \text{force} \cdot \text{velocity}$$
$$\text{power} = \text{torque} \cdot \text{angular velocity}$$

\subsection{Kinetic Energy}
Car's brakes are designed to stop the car by turning its kinetic energy into thermal energy.
Brake rubs stationary brake pads against spinning metal discs. \\


%%%%%%%%%%%%%%%%%%%%%%%
%% R E S O N A N C E %%
%%%%%%%%%%%%%%%%%%%%%%%

\newpage
\section{Resonance}

\clearpage
\printindex
\end{document}

%%%%%%%%%%%%%%%%%%%%%
%% D O C U M E N T %%
%%%%%%%%%%%%%%%%%%%%%
