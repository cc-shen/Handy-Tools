\documentclass[12pt]{article}
\usepackage[margin = 1.5in]{geometry}
\setlength{\parindent}{0in}
\usepackage{amsfonts, amssymb, amsthm, mathtools, tikz, qtree, float}
\usepackage{algpseudocode, algorithm, algorithmicx}
\usepackage[T1]{fontenc}
\usepackage[utf8]{inputenc}
\usepackage{DejaVuSans}
\usepackage{ae, aecompl, color}
\usepackage{wrapfig}
\usepackage{multicol, multicol, array}
\usepackage{imakeidx}
\makeindex[columns=2, title=Indices, intoc]

\usepackage[pdftex, pdfauthor={Charles Shen}, pdftitle={CS 350: Operating Systems}, pdfsubject={Lecture notes from CS 350: at the University of Waterloo}, pdfkeywords={course notes, notes, Waterloo, University of Waterloo}, pdfproducer={LaTeX}, pdfcreator={pdflatex}]{hyperref}
\usepackage{cleveref}

\DeclarePairedDelimiter{\set}{\lbrace}{\rbrace}
\renewcommand*\familydefault{\sfdefault}
\definecolor{darkish-blue}{RGB}{25,103,185}

\hypersetup{
  colorlinks,
  citecolor=darkish-blue,
  filecolor=darkish-blue,
  linkcolor=darkish-blue,
  urlcolor=darkish-blue
}

\theoremstyle{plain}
\newtheorem{theorem}{Theorem}[subsection]

\theoremstyle{definition}
\newtheorem{proposition}[theorem]{Proposition}
\newtheorem{corollary}[theorem]{Corollary}
\newtheorem{lemma}[theorem]{Lemma}
\newtheorem{ex}[theorem]{Example}
\newtheorem{defn}{Definition}

\crefname{ex}{Example}{Example}

\setlength{\marginparwidth}{1.5in}
\newcommand{\lecture}[1]{
  \marginpar{{
    \footnotesize $\leftarrow$ \underline{#1}}
  }
}

\newcommand{\defnterm}[1]{\textbf{\textcolor{teal}{#1}}\index{#1}}

\newcolumntype{L}[1]{>{\raggedright\let\newline\\\arraybackslash\hspace{0pt}}m{#1}}
\newcolumntype{C}[1]{>{\centering\let\newline\\\arraybackslash\hspace{0pt}}m{#1}}
\newcolumntype{R}[1]{>{\raggedleft\let\newline\\\arraybackslash\hspace{0pt}}m{#1}}

\allowdisplaybreaks

\makeatletter
\def\blfootnote{\gdef\@thefnmark{}\@footnotetext}
\makeatother

%%%%%%%%%%%%%%%%%%%%%
%% D O C U M E N T %%
%%%%%%%%%%%%%%%%%%%%%

\begin{document}
\let\ref\Cref
\pagenumbering{roman}

\title{\bf{CS 350: Operating Systems}}
\date{Fall 2016, University of Waterloo \\ \center Notes written from Gregor Richards's lectures.}
\author{Charles Shen}

\blfootnote{Feel free to email feedback to me at
\href{mailto:ccshen902@gmail.com}{ccshen902@gmail.com}.}

\maketitle
\thispagestyle{empty}
\newpage
\tableofcontents
\newpage
\pagenumbering{arabic}

%%%%%%%%%%%%%%%%%%
%% INTRODUCTION %%
%%%%%%%%%%%%%%%%%%
\section{Introduction}
There are three views of an operating system:
\begin{enumerate}
  \item[1.] \defnterm{Application View} (\ref{app_view_OS}): what service does it provide?
  \item[2.] \defnterm{System View} (\ref{sys_view_OS}): what problems does it solve?
  \item[3.] \defnterm{Implementation View} (\ref{impl_view_OS}): how is it built?
\end{enumerate}
\emph{An operating system is part cop, part facilitator}. \\

\defnterm{kernel}: The operating system kernel is the part of the operating system that responds to system calls, interrupts and exception. \\

\defnterm{operating system} (OS): The operating system as a whole includes the kernel, and may include other related programs that provide services for application such as utility programs, command interpreters, and programming libraries.

\subsection{Application View of an Operating System}\label{app_view_OS}
The OS provides an execution environment for running programs.
\begin{itemize}
  \item The execution environment provides a program with the processor time and memory space that it needs to run.
  \item The execution environment provides interfaces through which a program can use networks, storage, I/O devices, and other system hardware components. \\
  Interfaces provide a simplified, abstract view of hardware to application programs.
  \item The execution environment isolates running programs from one another and prevents undesirable interactions among them.
\end{itemize}

\subsection{System View of an Operating System}\label{sys_view_OS}
The OS manages the hardware resources of a computer system.
\begin{itemize}
  \item Resources include processors, memory, disks and other storage devices, network interfaces, I/O devices such as keyboards, mice and monitors, and so on.
  \item The operating system allocates resources among running programs. \\
  It controls the sharing of resources among programs.
  \item The OS itself also uses resources, which it must share with application programs.
\end{itemize}

\subsection{Implementation View of an Operating System}\label{impl_view_OS}
The OS is a concurrent, real-time program.
\begin{itemize}
  \item Concurrency arises naturally in an OS when it supports concurrent applications, and because it must interact directly with the hardware.
  \item Hardware interactions also impose timing constraints.
\end{itemize}

\subsection{Operating System Abstractions}
The execution environment provided by the OS includes a variety of abstract entities that can be manipulated by a running program. \\
Examples:
\begin{itemize}
  \item \textbf{files and file systems}: abstract view of secondary storage
  \item \textbf{address spaces}: abstract view of primary memory
  \item \textbf{processes, threads}: abstract view of program execution
  \item \textbf{sockets, pipes}: abstract view of network or other message channels
\end{itemize}

%%%%%%%%%%%%%%%%%%%%%%%%%%%%%
%% THREADS AND CONCURRENCY %%
%%%%%%%%%%%%%%%%%%%%%%%%%%%%%
\newpage
\section{Threads and Concurrency}
Threads provide a way for programmers to express \emph{concurrency} in a program. \\
A normal \emph{sequential program} consists of a single thread of execution. \\
In threaded concurrent programs, there are multiple threads of executions that are all occurring at the same time.

\subsection{OS/161's Thread Interface}
Create a new thread:
\begin{verbatim}
int thread_fork(
  const char *name,             // name of new thread
  struct proc *proc,            // thread's process
  void (*func)                  // new thread's function
    (void *, unsigned long),
  void *datat1,                 // function's first param
  unsigned long data2           // function's second param
);
\end{verbatim}

Terminating the calling thread: \\
\begin{verbatim}
void thread_exit(void);
\end{verbatim}

Voluntarily yield execution: \\
\begin{verbatim}
void thread_yield(void);
\end{verbatim}


\newpage
\section{Processes and System Calls}

\newpage
\section{Assignment 2A Review}

\newpage
\section{Virtual Memory}

\newpage
\section{Scheduling}

\newpage
\section{Devices and Device Management}

\newpage
\section{File Systems}

\newpage
\section{Interprocess Communications and Networking}

\clearpage
\printindex
\end{document}

%%%%%%%%%%%%%%%%%%%%%
%% D O C U M E N T %%
%%%%%%%%%%%%%%%%%%%%%
