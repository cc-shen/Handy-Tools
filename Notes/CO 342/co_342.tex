\documentclass[12pt]{article}
\usepackage[margin = 1.5in]{geometry}
\setlength{\parindent}{0in}
\usepackage{amsfonts, amssymb, amsthm, mathtools, tikz, qtree, float}
\usepackage{algpseudocode, algorithm, algorithmicx}
\usepackage[T1]{fontenc}
\usepackage[utf8]{inputenc}
\usepackage{DejaVuSans}
\usepackage{ae, aecompl, color}
\usepackage{wrapfig}
\usepackage{multicol, multicol, array}
\usepackage{imakeidx}
\makeindex[columns=2, title=Indices, intoc]

\usepackage[pdftex, pdfauthor={Charles Shen}, pdftitle={CO 342: Introduction to Graph Theory}, pdfsubject={Theorems and more from CO 342: at the University of Waterloo}, pdfkeywords={course notes, notes, Waterloo, University of Waterloo}, pdfproducer={LaTeX}, pdfcreator={pdflatex}]{hyperref}
\usepackage{cleveref}

\DeclarePairedDelimiter{\set}{\lbrace}{\rbrace}
\renewcommand*\familydefault{\sfdefault}
\definecolor{darkish-blue}{RGB}{25,103,185}

\hypersetup{
  colorlinks,
  citecolor=darkish-blue,
  filecolor=darkish-blue,
  linkcolor=darkish-blue,
  urlcolor=darkish-blue
}

\theoremstyle{plain}
\newtheorem{theorem}{Theorem}[subsection]

\theoremstyle{definition}
\newtheorem{proposition}[theorem]{Proposition}
\newtheorem{corollary}[theorem]{Corollary}
\newtheorem{lemma}[theorem]{Lemma}
\newtheorem{ex}[theorem]{Example}
\newtheorem{defn}{Definition}

\crefname{ex}{Example}{Example}

\setlength{\marginparwidth}{1.5in}
\newcommand{\lecture}[1]{
  \marginpar{{
    \footnotesize $\leftarrow$ \underline{#1}}
  }
}

\newcommand{\defnterm}[1]{\textbf{\textcolor{teal}{#1}}\index{#1}}

\newcolumntype{L}[1]{>{\raggedright\let\newline\\\arraybackslash\hspace{0pt}}m{#1}}
\newcolumntype{C}[1]{>{\centering\let\newline\\\arraybackslash\hspace{0pt}}m{#1}}
\newcolumntype{R}[1]{>{\raggedleft\let\newline\\\arraybackslash\hspace{0pt}}m{#1}}

\allowdisplaybreaks

\makeatletter
\def\blfootnote{\gdef\@thefnmark{}\@footnotetext}
\makeatother

%%%%%%%%%%%%%%%%%%%%%
%% D O C U M E N T %%
%%%%%%%%%%%%%%%%%%%%%

\begin{document}
\let\ref\Cref
\pagenumbering{roman}

\title{\bf{CO 342: Introduction to Graph Theory}}
\date{Fall 2016, University of Waterloo \\ \center Theorems and more reference sheet. \\ \center \textbf{Dropped}}
\author{Charles Shen}

\blfootnote{Feel free to email feedback to me at
\href{mailto:ccshen902@gmail.com}{ccshen902@gmail.com}.}

\maketitle
\thispagestyle{empty}
\newpage
\tableofcontents
\newpage
\pagenumbering{arabic}

%%%%%%%%%%%%%%%
%% B A S I C %%
%%%%%%%%%%%%%%%

\section{The Basics}
\subsection{The Degree of a Vertex}
\begin{proposition}
  The number of vertices in a graph is always even.
\end{proposition}

The number $\delta(G) = min \{d(v) | v \in V\}$ is the \defnterm{minimum degree} of $G$. \\
The number $\Delta(G) = max \{d(V) | v \in V\}$ is the \defnterm{maximum degree} of $G$. \\
The number
$$d(G) = \frac{1}{|V|}\sum_{v \in V}d(v)$$
is the \defnterm{average degree} of $G$. \\
The \defnterm{average degree ratio} of $G$ is expressed as $\epsilon(G) = |E|/|V|$.

\begin{proposition}
  Every graph $G$ with at least one edge has a subgraph $H$ with $\delta(H) > \epsilon(H) \geq \epsilon(G)$.
\end{proposition}

\subsection{Paths and Cycles}
The \emph{minimum} length of a cycle (contained) in a graph $G$ is the \defnterm{girth} $g(G)$ of $G$;
the \emph{maximum} length of a cycle in $G$ is its \defnterm{circumference}.

\begin{proposition}
  Every graph $G$ contains a path of length $\delta(G)$ and a cycle of at least $\delta(G) + 1$ (provided that $\delta(G) \geq 2$).
\end{proposition}

The \defnterm{distance} $d_G(x, y)$ in $G$ of two vertices $x, y$ is the length of a shortest $x$-$y$ path in $G$;
if no such path exists, we exist $d(x, y) = \infty$. \\

The greatest distance between any two vertices in $G$ is the \defnterm{diameter} of $G$, denoted by diam$G$.

\begin{proposition}
  Every graph $G$ containing a cycle satisfies $g(G) \leq 2\text{diam}G + 1$.
\end{proposition}

A vertex is \defnterm{central} in $G$ if its greatest distance from any other vertex is as small as possible.
This distance is the \defnterm{radius} of $G$, denoted by rad$G$.

\begin{proposition}
  A graph $G$ is radius at most $K$ and maximum degree at most $d \geq 3$ has fewer than $\frac{d}{d-2}(d - 1)^{k}$ vertices.
\end{proposition}

\begin{theorem}
  Let $G$ be a graph. If $d(G) \geq d \geq 2$ and $g(G) \geq g \in \mathbb{N}$ then $|G| \geq n_{0}(d, g)$.
\end{theorem}

\begin{corollary}
  If $\delta(G) \geq 3$ then $g(G) < 2\log |G|$.
\end{corollary}

\subsection{Connectivity}
\begin{proposition}
  The vertices of a connected graph $G$ can always be enumerated, say $v_{1}, \dots, v_{n}$, so that $G_{i} = G[v_{1}, \dots, v_{i}]$ is connected for every $i$.
\end{proposition}

$G$ is called \defnterm{k-connected} (for $k \in \mathbb{N}$) if $|G| > k$ and $G - X$ is connected for every set $X \subseteq V$ with $|X| < k$.
That is, no two vertices of $G$ are separated by fewer than $k$ other vertices. \\
The greatest integer $k$ such that $G$ is k-connected is the \defnterm{connectivity} $\kappa(G)$ of G. \\
$\kappa(G) = 0 \iff G$ is disconnected or a $K^{1}$, and $\kappa(K^{n}) = n - 1 \quad \forall n \geq 1$. \\

If $|G| > 1$ and $G - F$ is connected for every set $F \subseteq E$ of fewer than $\ell$ edges, then $G$ is called \defnterm{$\ell$-edge-connected}. \\
The greatest integer $\ell$ such that $G$ is $\ell$-edge connected is the \defnterm{edge-connectivity} $\lambda(G)$ of $G$. \\
$\lambda(G) = 0$ if $G$ is disconnected.

\begin{proposition}
  If $G$ is non-trivial then $\kappa(G) \leq \lambda(G) \leq \delta(G)$.
\end{proposition}

By Proposition 1.4.2, high connectivity requires a large minimum degree. \\
Conversely, large minimum degree does not ensure high connectivity, not even high edge-connectivity. \\
A large minimum degree does imply the existence of a highly connected subgraph.

\begin{theorem}
  Let $0 \neq k \in \mathbb{N}$. Every graph $G$ with $d(G) \geq 4k$ has a $(k+1)$-connected subgraph $H$ such that $\epsilon(H) > \epsilon(G) - k$.
\end{theorem}

\subsection{Trees and Forests}
\begin{theorem}
  The following assertions are equivalent for a graph $T$:
  \begin{enumerate}
    \item[(i)] $T$ is a tree
    \item[(ii)] Any two vertices of $T$ are linked by a unique path in $T$
    \item[(iii)] $T$ is minimally, i.e. $T$ is connected but $T - e$ is disconnected for every edge $e \in T$
    \item[(iv)] $T$ is maximally acyclic, i.e. $T$ contains no cycle but $T + xy$ does for any two non-adjacent vertices $x, y \in T$
  \end{enumerate}
\end{theorem}

\begin{corollary}
  The vertices of a tree can be enumerated, say as $v_{1}, \dots, v_{n}$, so that every $v_{i}$ with $i \geq 2$ has a unique neighbour in $\{v_{1}, \dots, v_{i - 1}\}$.
\end{corollary}

\begin{corollary}
  A connected graph with $n$ vertices is a tree $\iff$ it has $n - 1$ edges.
\end{corollary}

\begin{corollary}
  If $T$ is a tree and $G$ is any graph with $\delta(G) \geq |T| - 1$, then $T \subseteq G$, i.e. G has a subgraph isomorphic to $T$.
\end{corollary}

\begin{lemma}
  Let $T$ be a normal tree in $G$.
  \begin{enumerate}
    \item[(i)] Any two vertices $x, y \in T$ are separated in $G$ by the set $\lceil x\rceil \cap \lceil y \rceil$.
    \item[(ii)] If $S \subseteq V(T) = V(G)$ and $S$ is down-closed, then the components of $G - S$ are spanned by the sets $\lfloor x \rfloor$ with $x$ minimal in $T - S$.
  \end{enumerate}
\end{lemma}

\begin{proposition}
  Every connected graph contains a normal spanning tree, with any specified vertex as its root.
\end{proposition}

\subsection{Bipartite Graphs}
\begin{proposition}
  A graph is bipartite $\iff$ it contains no odd cycle.
\end{proposition}

\subsection{Contraction and Minors}
\begin{proposition}
  The minor relation $\preceq$ and the topological-minor relation are partial orderings on the class of finite graphs, i.e. they are reflexive, antisymmetric and transitive.
\end{proposition}

\begin{proposition}
  A finite graph $G$ is an $IX \iff X$ can be obtained from $G$ by a sequence of edge contractions, i.e. $\iff$ there are graphs $G_{0}, \dots, G_{n}$ and edges $e_{i} \in G_{i}$ such that $G_{0} = G$, $G_{n} \simeq X$, and $G_{i+1} = G_{i}/e_{i} \quad \forall i < n$.
\end{proposition}

\begin{corollary}
  Let $X$ and $Y$ be finite graphs, $X$ is a minor of $Y \iff$ there are graphs $G_{0}, \dots, G_{n}$ such that $G_{0} = Y$ and $G_{n} = X$ and each $G_{i+1}$ arises from $G_{i}$ by deleting an edge, contracting an edge, or deleting a vertex.
\end{corollary}

\begin{proposition}
  \leavevmode
  \begin{enumerate}
    \item[(i)] Every $TX$ is also an $IX$;
    thus, every topological minor of a graph is also its (ordinary) minor
    \item[(ii)] If $\Delta(X) \leq 3$, then every $IX$ contains a $TX$;
    thus, every minor with maximum degree at most 3 of a graph is also its topological minor.
  \end{enumerate}
\end{proposition}

\subsection{Euler Tours}
A closed walk in a graph is an \defnterm{Euler tour} if it traverses every edge of the graph exactly once. \\
A graph is \defnterm{Eulerian} if it admits an Euler tour.

\begin{theorem}
  A connected graph is Eulerian $\iff$ every vertex has even degrees.
\end{theorem}

%%%%%%%%%%%%%%%%%%%%%%%%%%%%%
%% C O N N E C T I V I T Y %%
%%%%%%%%%%%%%%%%%%%%%%%%%%%%%

\section{Connectivity}
\subsection{2-Connected Graphs and Subgraphs}

A relation, $\approx$ say, on a set $E$ is \defnterm{equivalence relation} if
\begin{itemize}
  \item $x \approx x$ for all $x$ in $E$ [reflexive]
  \item whenever $x, y \in E$ and $x \approx y$ we also have $y \approx x$ [symmetric]
  \item If $x \approx y$ and $y \approx z$ then $x \approx z$ [transitive]
\end{itemize}

If $E$ is the vertex set of a graph and $\approx$ means ``is joined by a path to'', then $\approx$ is an equivalence relation on $E$.

\begin{lemma}
  If $G$ is a graph then ``is equal to or lies in a circuit with'' is an equivalence relation on $E(G)$.
\end{lemma}

\begin{proposition}
  A graph is 2-connected $\iff$ it can be constructed from a cycle by successively adding $H$-paths to graphs $H$ already constructed.
\end{proposition}

\begin{lemma}
  Let $G$ be any graph.
  \begin{enumerate}
    \item[(i)] The cycles of $G$ are precisely the cycles of its blocks
    \item[(ii)] The bonds of $G$ are precisely the minimal cuts of its blocks.
  \end{enumerate}
\end{lemma}

\begin{theorem}
  For a connected graph $G$ with at least three vertices, the following properties are equivalent:
  \begin{itemize}
    \item $G$ is 2-connected
    \item any two edges of $G$ lie on a circuit (cycle)
    \item any two vertices of $G$ lie on a circuit
  \end{itemize}
\end{theorem}

\subsection{Blocks}
A connected subgraph $H$ of $G$ is a \defnterm{block} of $G$ if it has no cut-vertex, but any subgraph of $G$ that contains $H$ properly is either not connected or has a cut-vertex.

\begin{lemma}
  Let $G$ be a graph. The equivalence classes of edges of $G$ under $\approx$ are precisely the edge sets of the blocks of $G$.
\end{lemma}

\begin{lemma}
  Any cut vertex of a graph lies in at least two blocks.
\end{lemma}

\begin{lemma}
  The block-cut vertex graph of a graph is a forest.
\end{lemma}

\clearpage
\printindex
\end{document}

%%%%%%%%%%%%%%%%%%%%%
%% D O C U M E N T %%
%%%%%%%%%%%%%%%%%%%%%
