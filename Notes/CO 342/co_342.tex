\documentclass[12pt]{article}
\usepackage[margin = 1.5in]{geometry}
\setlength{\parindent}{0in}
\usepackage{amsfonts, amssymb, amsthm, mathtools, tikz, qtree, float}
\usepackage{algpseudocode, algorithm, algorithmicx}
\usepackage[T1]{fontenc}
\usepackage[utf8]{inputenc}
\usepackage{DejaVuSans}
\usepackage{ae, aecompl, color}
\usepackage{wrapfig}
\usepackage{multicol, multicol, array}
\usepackage{imakeidx}
\makeindex[columns=2, title=Indices, intoc]

\usepackage[pdftex, pdfauthor={Charles Shen}, pdftitle={CO 342: Introduction to Graph Theory}, pdfsubject={Theorems and more from CO 342: at the University of Waterloo}, pdfkeywords={course notes, notes, Waterloo, University of Waterloo}, pdfproducer={LaTeX}, pdfcreator={pdflatex}]{hyperref}
\usepackage{cleveref}

\DeclarePairedDelimiter{\set}{\lbrace}{\rbrace}
\renewcommand*\familydefault{\sfdefault}
\definecolor{darkish-blue}{RGB}{25,103,185}

\hypersetup{
  colorlinks,
  citecolor=darkish-blue,
  filecolor=darkish-blue,
  linkcolor=darkish-blue,
  urlcolor=darkish-blue
}

\theoremstyle{definition}
\newtheorem*{defn}{Definition}
\newtheorem*{theorem}{Theorem}
\newtheorem*{corollary}{Corollary}
\newtheorem{ex}{Example}[section]

\crefname{ex}{Example}{Example}

\setlength{\marginparwidth}{1.5in}
\newcommand{\lecture}[1]{
  \marginpar{{
    \footnotesize $\leftarrow$ \underline{#1}}
  }
}

\newcommand{\defnterm}[1]{
  \textbf{\textcolor{teal}{#1}}\index{#1}
}

\newcolumntype{L}[1]{>{\raggedright\let\newline\\\arraybackslash\hspace{0pt}}m{#1}}
\newcolumntype{C}[1]{>{\centering\let\newline\\\arraybackslash\hspace{0pt}}m{#1}}
\newcolumntype{R}[1]{>{\raggedleft\let\newline\\\arraybackslash\hspace{0pt}}m{#1}}

\allowdisplaybreaks

\makeatletter
\def\blfootnote{\gdef\@thefnmark{}\@footnotetext}
\makeatother

%%%%%%%%%%%%%%%%%%%%%
%% D O C U M E N T %%
%%%%%%%%%%%%%%%%%%%%%

\begin{document}
\let\ref\Cref
\pagenumbering{roman}

\title{\bf{CO 342: Introduction to Graph Theory}}
\date{Fall 2016, University of Waterloo \\ \center Theorems and more reference sheet.}
\author{Charles Shen}

\blfootnote{Feel free to email feedback to me at
\href{mailto:ccshen902@gmail.com}{ccshen902@gmail.com}.}

\maketitle
\thispagestyle{empty}
\newpage
\tableofcontents
\newpage
\pagenumbering{arabic}

\section{The Basics}
\subsection{The Degree of a Vertex}
\begin{theorem}
  The number of vertices in a graph is always even.
\end{theorem}

The number $\delta(G) = min \{d(v) | v \in V\}$ is the \defnterm{minimum degree} of $G$

\begin{theorem}
  Every graph $G$ with at least one edge has a subgraph $H$ with $\delta(H) > \epsilon(H) \geq \epsilon(G)$.
\end{theorem}


\clearpage
\printindex
\end{document}

%%%%%%%%%%%%%%%%%%%%%
%% D O C U M E N T %%
%%%%%%%%%%%%%%%%%%%%%
